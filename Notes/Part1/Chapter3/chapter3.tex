\documentclass[../../notes.tex]{subfiles}

\pagestyle{main}
\renewcommand{\chaptermark}[1]{\markboth{\chaptername\ \thechapter\ (#1)}{}}
\setcounter{chapter}{2}

\begin{document}




\chapter{Determinants}
\begin{itemize}
    \item \marginnote{9/29:}The determinant, geometrically, is the volume of the object (in $\R^3$) you get when you take linear combinations of the vectors.
    \item In 2D:
    \begin{itemize}
        \item Let $v_1,v_2$ be two vectors. Put tail to tail and forming a parallelogram, the determinant of the matrix $(v_1,v_2)$ is the area of said parallelogram.
        \item Linearity 1: $D(av_1,v_2,\dots,v_n)=aD(v_1,\dots,v_n)$ is the same as saying that if you stretch one vector by $a$, you scale up the area by that much, too.
        \item Linearity 2: $D(v_1,\dots,v_{k+}+v_{k-},\dots,v_n)=D(-)+D(+)$.
        \item Antisymmetry: $D(v_1,\dots,v_k,\dots,v_j,\dots,v_n)=-D(v_1,\dots,v_j,\dots,v_k,\dots,v_n)$. Interchanging columns flips the sign of the determinant.
        \item Basis: $D(e_1,\dots,e_n)=1$.
    \end{itemize}
    \item Determinant: Denoted by $D(v_1,\dots,v_n)$, where $(v_1,\dots,v_n)$ is an $n\times n$ matrix.
\end{itemize}




\end{document}