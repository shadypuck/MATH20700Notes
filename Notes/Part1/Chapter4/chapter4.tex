\documentclass[../../notes.tex]{subfiles}

\pagestyle{main}
\renewcommand{\chaptermark}[1]{\markboth{\chaptername\ \thechapter\ (#1)}{}}
\setcounter{chapter}{3}

\begin{document}




\chapter{Introduction to Spectral Theory}
\begin{itemize}
    \item \marginnote{10/1:}\textbf{Difference equation}: Like a differential equation, but instead of writing a differentials, you write differences.
    \item Suppose we want to solve $x_{n+1}=Ax_n$ with $x_0$ given.
    \begin{itemize}
        \item You will find that $x_n=A^nx_0$.
        \item This gets hard to compute, so we want to find a way to simplify the computation.
    \end{itemize}
    \item Thus, we want to diagonalize the matrix, and this concept is inherently linked to eigenvalues and eigenvectors.
    \begin{itemize}
        \item If you can decompose the $x_0$ into a linear combination of eigenvectors, then you can simplify the computation a lot:
        \begin{equation*}
            x_n = \sum\alpha_iA^nv_i = \sum\alpha_i\lambda_i^nv_i
        \end{equation*}
        \item An $n\times n$ matrix will have $n$ eigenvalues. You want $n$ linearly independent eigenvectors, creating an eigenbasis.
    \end{itemize}
    \item To find eigenvalues and eigenvectors, we need to solve $Ax=\lambda x$, i.e., $(A-\lambda I)x=0$. Thus, $\ker(A-\lambda I)\neq\{0\}$, so $\det(A-\lambda I)=0$.
    \item The eigenvalues of $A$ are independent of the choice of basis of the domain of $A$ or the range.
\end{itemize}




\end{document}