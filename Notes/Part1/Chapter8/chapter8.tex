\documentclass[../../notes.tex]{subfiles}

\pagestyle{main}
\renewcommand{\chaptermark}[1]{\markboth{\chaptername\ \thechapter\ (#1)}{}}
\setcounter{chapter}{7}

\begin{document}




\chapter{Dual Spaces and Tensors}
\section{Notes}
\begin{itemize}
    \item \marginnote{10/22:}\textbf{Functional}: A linear bounded map $L:H\to F$, where $H$ is finite dimensional (equivalent to $\R^n$).
    \item \textbf{Dual space}: The set of bounded linear functionals on $H$. \emph{Denoted by} $\bm{H'}$, $\bm{H^*}$.
    \item If $l\leq p<\infty$, then
    \begin{equation*}
        l^p = \left\{ (a_n)_{n\in\N}:\sum_{n=1}^\infty|a_n|^p<\infty \right\}
    \end{equation*}
    \item Back to finite dimensions, $H'\approx\R^n$.
    \item Let $\ab_1,\dots,\ab_n$ be a basis of $H$. Then $L\x=(L\ab_1,\dots,L\ab_n)\approx\R^n$.
    \item Let $L((a_n)_{n\in\N})=\sum_{n=1}^\infty a_nb_n$. Then $L((a_n)_{n\in\N})$ will be bounded if and only if $(b_n)_{n\in\N}\in l^q$ where $1<p<q$ where $\frac{1}{q}+\frac{1}{p}=1$.
    \item \textbf{Young's inequality}: The statement
    \begin{equation*}
        ab \leq \frac{a^p}{p}+\frac{b^q}{q}
    \end{equation*}
    \item We have $|\sum a_nb_n|\leq\norm{a_n}_p\norm{b_n}_p$.
    \item Conclusion:
    \begin{equation*}
        \sum\frac{|a_n||b_n|}{\norm{a_n}_p\norm{b_n}_q} = 1
    \end{equation*}
    \item We can define $H''$, too. This contains linear functionals on $H'$.
    \item We know that $L(x)=\langle x,L\rangle=x(L)$. $x\in H''$.
    \item Riesz representation theorem: Let $H$ have an inner product. $L\in H'$ if and only if there exists a unique $y\in H$ such that $L(x)=(x,y)$.
    \begin{itemize}
        \item Gives us a way to identify all bounded linear functionals on $H$.
        \item In finite dimensions, $L(x)$, where $x=\sum_1^n\alpha_ia_i$ gives us $L(x)=\sum_1^n\alpha_iL(a_i)$.
    \end{itemize}
\end{itemize}



\section{Chapter 8: Dual Spaces and Tensors}
\begin{itemize}
    \item \marginnote{10/28:}Linear functionals are denoted by $L$.
    \begin{itemize}
        \item $L$ is given by a $1\times n$ matrix denoted by $[L]$.
    \end{itemize}
    \item The collection of all $[L]$ (the dual space) is isomorphic to $\R^n$ via $[L]\mapsto[L]^T$.
    \begin{itemize}
        \item However, the objects are different: Let $[I]_{\mathcal{B}\mathcal{A}}$ be the change of coordinates matrix in $\R^n$. We thus have that
        \begin{equation*}
            [\vm]_\mathcal{B} = [I]_{\mathcal{B}\mathcal{A}}[\vm]_\mathcal{A}
        \end{equation*}
        but we also have that
        \begin{equation*}
            [L]_\mathcal{B} = [L]_\mathcal{A}[I]_{\mathcal{A}\mathcal{B}}
        \end{equation*}
        so that
        \begin{equation*}
            [L]_\mathcal{B}^T = ([L]_\mathcal{A}[I]_{\mathcal{A}\mathcal{B}})^T
            = [I]_{\mathcal{A}\mathcal{B}}^T[L]_\mathcal{A}^T
        \end{equation*}
        \item Essentially, "if $S$ is the change of coordinate matrix in $X$\dots then the change of coordinate matrix in the dual space $X'$ is $(S^{-1})^T$" \parencite[219]{bib:Treil}.
    \end{itemize}
    \item Lemma 8.1.3: Let $\vm\in V$. If $L(\vm)=0$ for all $L\in V'$, then $\vm=\bm{0}$. As a corollary, if $L(\vm_1)=L(\vm_2)$ for all $L\in V'$, then $\vm_1=\vm_2$.
    \item The second dual $V''$ is canonically (i.e., in a natural way) isomorphic to $V$.
    \item \textbf{Dual basis} (to $\bb_1,\dots,\bb_n\in V$): The system of vectors $\bb_1',\dots,\bb_n'\in V'$ uniquely defined by the following equation. \emph{Also known as} \textbf{biorthogonal basis}.
    \begin{equation*}
        \bb_k'(\bb_j) = \delta_{kj}
    \end{equation*}
    \begin{itemize}
        \item The $k^\text{th}$ coordinate of a vector $\vm$ in a basis $\bb_1,\dots,\bb_n$ is $\bb_k'(\vm)$.
        \begin{itemize}
            \item This is a baby version of the \textbf{abstract non-orthogonal Fourier decomposition} of $\vm$.
        \end{itemize}
    \end{itemize}
    \item Theorem 8.2.1 (Riesz representation theorem): Let $H$ be an inner product space. Given a linear functional $L$ on $H$, there exists a unique vector $\y\in H$ such that
    \begin{equation*}
        L(\vm) = (\vm,\y)
    \end{equation*}
    for all $\vm\in H$.
    \item If $V$ is a real inner product space, we can define an isomorphism from $V$ to $V'$ by $\y\mapsto L_\y=(\vm,\y)$.
    \begin{itemize}
        \item If $V$ is complex, this function is not linear since if $\alpha$ is complex,
        \begin{equation*}
            L_{\alpha\y}(\vm) = (\vm,\alpha\y) = \bar{\alpha}(\vm,\y) = \bar{\alpha}L_\y(\vm)
        \end{equation*}
        \item It follows by such a mapping that $\bb_k'=\bb_k$ for each $k$.
    \end{itemize}
    \item \textbf{Conjugate linear} (transformation): A transformation $T$ such that
    \begin{equation*}
        T(\alpha\x+\beta\y) = \bar{\alpha}T\x+\bar{\beta}T\y
    \end{equation*}
    \item It is customary to write outputs of linear functionals $L(\vm)$ in the form $\langle\vm, L\rangle$.
    \begin{itemize}
        \item This expression is linear in both arguments, unlike the inner product.
    \end{itemize}
    \item Defines the dual transformation as the unique transformation such that
    \begin{equation*}
        \langle A\x,\y'\rangle = \langle\x,A'\y\rangle
    \end{equation*}
    for all $\x\in X$, $\y'\in Y'$.
    \begin{itemize}
        \item It's matrix in the standard bases equals $A^T$.
    \end{itemize}
    \item Annihilators are denoted by $E^\perp$ here.
    \item Proposition 8.3.6: The annihilator of the annihilator of $E$ equals $E$.
    \item Let $A:X\to Y$ be an operator acting from one vector space to another. Then
    \begin{enumerate}
        \item $\ker A'=(\range A)^\perp$.
        \item $\ker A=(\range A')^\perp$.
        \item $\range A = (\ker A')^\perp$.
        \item $\range A'=(\ker A)^\perp$.
    \end{enumerate}
\end{itemize}




\end{document}