\documentclass[../../notes.tex]{subfiles}

\pagestyle{main}
\renewcommand{\chaptermark}[1]{\markboth{\chaptername\ \thechapter\ (#1)}{}}
\setcounter{chapter}{8}

\begin{document}




\chapter{Advanced Spectral Theory}
\begin{itemize}
    \item \marginnote{10/22:}Let $p(z)=\sum_{i=0}^na_iz^i$ be a polynomial. Let $A$ be an $n\times n$ matrix. We let $p(A)=\sum_{i=0}^na_iA^i$.
    \item Theorem: If $A$ is an $n\times n$ and $p(\lambda)=\det(A-\lambda I)$, then $p(A)=0$.
    \begin{itemize}
        \item We know that $p(\lambda)=a(z-\lambda_1)\cdots(z-\lambda_n)$ where $\lambda_1,\dots,\lambda_n$ are the eigenvalues.
        \item Thus $p(A)=a(A-\lambda_1I)\cdots(A-\lambda_nI)$.
        \item If you are in $\R^n$ and have this property, you can factorize your matrix.
        \item Thus, $p(A)\x=\bm{0}$ since $\x$ can be decomposed into a linear combination of eigenvectors of $A$, which will be taken to 0 one by one by the terms of $p(A)$.
    \end{itemize}
    \item $\sigma(B)=\{\text{eigenvalues of }B\}$ is known as the \textbf{spectrum} of $B$.
    \item If $p$ is an arbitrary polynomial and $A$ is $n\times n$, then $\mu$ is an eigenvalue of $p(A)$ if and only if $\mu=p(\lambda)$ where $\lambda$ is an eigenvalue of $A$. In essence, $\sigma(p(A))=p(\sigma(A))$.
    \item Chapter 9 will not be on the exam. We don't have to know the generalization to infinite dimensional spaces.
\end{itemize}




\end{document}