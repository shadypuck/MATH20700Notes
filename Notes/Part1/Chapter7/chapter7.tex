\documentclass[../../notes.tex]{subfiles}

\pagestyle{main}
\renewcommand{\chaptermark}[1]{\markboth{\chaptername\ \thechapter\ (#1)}{}}
\setcounter{chapter}{6}

\begin{document}




\chapter{Bilinear and Quadratic Forms}
\begin{itemize}
    \item \marginnote{10/18:}\textbf{Bilinear form}: A function $L:\R^n\times\R^n\to\R$ such that
    \begin{align*}
        L(\alpha\x_1+\beta\x_2,\y) &= \alpha L(\x_1,\y)+\beta L(\x_2,\y)&
        L(\x,\alpha\y_1+\beta\y_2) &= \alpha L(\x,\y_1)+\beta L(\x,\y_2)
    \end{align*}
    \begin{itemize}
        \item $L(\x,\y)=(A\x,\y)$.
    \end{itemize}
    \item \textbf{Quadratic form}: A bilinear form $L(\x,\x)$.
    \begin{itemize}
        \item $(\x,\x)$ is a polynomial of degree 2 in $\x_1,\dots,\x_n$:
        \begin{equation*}
            L(\lambda\x,\lambda\x) = (\lambda\x,\lambda\x) = \lambda^2(\x,\x)
        \end{equation*}
    \end{itemize}
    \item We have that
    \begin{equation*}
        (A\x,\x) = (A\lambda\x,\lambda\x) = \lambda^2(A\x,\x) = \sum_{j,i=1}^n\alpha_{j,i}\x_i\x_j
    \end{equation*}
    \item The general form of a quadratic form:
    \begin{itemize}
        \item Can any quadratic form on $\R^n$ be written as $(A\x,\x)$?
    \end{itemize}
\end{itemize}




\end{document}