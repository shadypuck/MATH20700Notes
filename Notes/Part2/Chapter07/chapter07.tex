\documentclass[../../notes.tex]{subfiles}

\pagestyle{main}
\renewcommand{\chaptermark}[1]{\markboth{\chaptername\ \thechapter\ (#1)}{}}
\setcounter{chapter}{6}

\begin{document}




\chapter{Sequences and Series of Functions}
\section{Notes}
\begin{itemize}
    \item \marginnote{11/15:}Soug will not test on differentiation/integration assuming that we know them already.
    \item \textbf{Pointwise convergent} (sequence $(f_n)_{n\in\N}$ of functions): A sequence of functions $f_n:E\to\R$ such that $\lim_{n\to\infty}f_n(x)=f(x)$ for all $x\in E$.
    \item Can we interchange "limit" in the above definition with continuity, convergence of series, integration, differentiation, etc.?
    \item Examples with negative answer:
    \begin{enumerate}
        \item Interchanging limits: Let $S_{mn}=\frac{m}{m+n}$. $S_{mn}\to 1$ as $m\to\infty$, and $S_{mn}\to 0$ as $n\to\infty$.
        \item $f_n(x)=x^2/(1+x)^n$. $f(x)=\sum_{n=1}^\infty f_n(x)$. If $x=0$, then $f_n(x)=0$ for all $n$ and $f(x)=0$. If $x\neq 0$, we have
        \begin{equation*}
            f(x) = \sum_{n=1}^\infty\frac{x^2}{(1+x^2)^n} = x^2\sum_{n=1}^\infty X^n = \frac{x^2}{1-X} = \frac{x^2}{1-(1/(1+x^2))} = 1+x^2
        \end{equation*}
        \item Consider $f_m(x)=\lim_{n\to\infty}(\cos(m\pi x))^2n$. $\lim_{m\to\infty}f_m(x)$ goes to 0 if $x\notin\Q$ and goes to 1 if $x\in\Q$. $f_m\to\chi_\Q$, where $\chi_\Q$ is the characteristic function of the rationals which is not Riemann integrable (partitions, upper and lower integrals, etc.).
        \item $f_n(x)=\sin nx/\sqrt{n}\to f(x)=0$ for all $x$. However, $f_n'(x)=\sqrt{n}\cos nx\nrightarrow 0$
        \item If $0\leq x\leq 1$, define $f_n(x)=n^2x(1-x^2)^n$. We know that $f_n(0)=0$. $\lim_{n\to\infty}f_n(x)=0$ for all $x\in(0,1]$. We can show that $\int_0^1x(1-x^2)^n\dd{x}=1/(2n+2)$. Thus, $\int_0^1f_n(x)\dd{x}=n^2/(2n+2)$. Limit of the functions is zero, but their integrals diverge to infinity.
    \end{enumerate}
    \item \textbf{Uniformly convergent} (sequence $(f_n)_{n\in\N}$ of functions on $E$): A sequence of functions $f_n:E\to\R$ such that for all $\epsilon>0$, there exists $N$ such that if $n\geq N$, then $|f_n(x)-f(x)|<\epsilon$ for all $x\in E$. \emph{Denoted by} $\bm{f_n\rightrightarrows f}$.
    \item Theorem: $f_n\rightrightarrows f$ iff $(f_n)_{n\in\N}$ is uniformly Cauchy (i.e., for all $\epsilon>0$, there exists $N$ such that if $n,m\geq N$ then $|f_n(x)-f_m(x)|<\epsilon$ for all $x\in E$).
    \begin{itemize}
        \item Let $M_n=\sup_{x\in E}|f_n(x)-f(x)|$. If $f_n\to f$ pointwise, then $f_n\rightrightarrows f$ if $M_n\to 0$.
    \end{itemize}
    \item Theorem: If $(f_n)_{n\in\N}$ and $|f_n(x)|\leq M_n$, then $\sum f_n\rightrightarrows f$ if $\sum M_n<\infty$.
    \item Theorem: If $E$ is a compact metric space, $f_n\rightrightarrows f$ in $E$, $x$ is a limit point of $E$, and $\lim_{t\to x}f_n(t)=A_n$ exists, then $(A_n)_{n\in\N}$ converges and $\lim_{t\to x}f(t)=\lim_{n\to\infty}A_n$.
    \item Corollary: $\lim_{t\to x}\lim_{n\to\infty}f_n(t)=\lim_{n\to\infty}\lim_{t\to x}f_n(t)$.
    \begin{itemize}
        \item \marginnote{11/16:}Fix $\epsilon>0$. Then $f_n\rightrightarrows f$ implies that there exists some $N$ such that $n,m\geq N$ implies $|f_n(t)-f_m(t)|<\epsilon$ for all $t\in E$.
        \item $x$ is a limit point of $E$ and $t\to x$ implies $|A_n-A_m|<\epsilon$. Thus, $(A_n)_{n\in\N}$ is cauchy, so there exists $A$ such that $A_n\to A$.
        \item WTS: $|f(t)-A|\leq|f(t)-f_n(t)|+|f_n(t)-A_n|+|A_n-A|$, so we WTS the three terms on the right are small.
        \item There exists $n$ such that $|f(t)-f_n(t)|<\epsilon/3$ for all $t$ since $f_n\rightrightarrows f$ by hypothesis.
        \item Since $t$ is in a small neighborhood of $x$, there exists $n$ such that $|A_n-A|<\epsilon/3$.
        \item We also have $|f_n(t)-A_n|<\epsilon/3$ by hypothesis.
        \item This is a very important proof to understand, because proofs like this pop up often.
    \end{itemize}
    \item Corollary: $f_n$ continuous and $f_n\rightrightarrows f$ implies $f$ is continuous.
    \item Theorem: Let $K$ be compact. Assume
    \begin{enumerate}[label={(\alph*)}]
        \item $(f_n)_{n\in\N}\subset C(K)=\{f:K\to\R\mid f\text{ continuous}\}$.
        \item $f_n\to f$ pointwise in $K$ and $f\in C(K)$.
        \item $f_n(x)\geq f_{n+1}(x)$ for all $x\in K$.
    \end{enumerate}
    Then $f_n\rightrightarrows f$.
    \begin{itemize}
        \item WLOG $f=0$, $g_n=f_n-f\to 0$, $g_n\geq g_{n+1}\geq 0$.
        \item For all $\epsilon>0$, there exists $N$ such that $n\geq N$ and $0\leq g_n(x)\leq\epsilon$ for all $x\in K$.
        \item $K_n=\{x\in K:g_n(x)\geq\epsilon\}$.
        \item $g_n$ continuous implies $K_n$ closed. This combined with $K$ compact implies $K_n$ is compact.
        \item $g_n$ decreasing implies $K_n\supset K_{n+1}$. Thus, $K_n$ is a nested family of compact sets, so $\bigcap K_n\neq\emptyset$.
        \item This implies that each $K_n$ is nonempty, contradicting the fact that each $g_n\to 0$ for all $x$.
        \item Thus, there exists an $N$ such that $K_n$ is empty for all $n\geq N$. Thus $g_n(x)\leq\epsilon$ for all $x\in K$, $n\geq N$.
        \item Note that the compactness of $K$ is important. If $f:(0,1)\to\R$ is defined by $f(x)=1/(nx+1)$, then $f_n\to 0$, but $f_n\not\rightrightarrows f$.
    \end{itemize}
    \item Let $C(X)=\{f:X\to\R\mid f\text{ continuous, bounded}\}$ for $X$ a metric space.
    \item If we define $\norm{f}=\sum_{x\in X}|f(x)|$, for $f,g\in C(X)$, we may define $d(f,g)=\norm{f-g}$. This definition satisfies the properties of a distance function, and $\norm{\cdot}$ is a norm.
    \begin{itemize}
        \item Thus, $C(X)$ is a complete metric space, a normed space, or, specifically, a \textbf{Banach space}.
    \end{itemize}
    \item Theorem: $(f_n)_{n\in\N}\subset C(X)$ such that $\norm{f_n-f_m}_{n,m\to\infty}\to 0$. Then there exists $f\in C(X)$ such that $\norm{f_n-f}_{n\to\infty}\to 0$.
    \begin{itemize}
        \item We get such a strong statement using properties of the image, not properties of the domain.
        \item For all $\epsilon>0$, there exists $N$ such that $n,m\geq N$.
        \item $|f_n(x)-f_m(x)|\leq\norm{f_n-f_m}<\epsilon$ for all $x$.
        \item Then there exists $f$ such that $f_m(x)\to f(x)$. It follows that $|f_n(x)-f_m(x)|<\epsilon$
    \end{itemize}
    \item Uniform convergence and integration.
    \item Stieltjes integral.
    \begin{itemize}
        \item Define $\alpha:\R\to\R$ nondecreasing.
        \item If we sum over the minimums/maximums of a partition times $\alpha(x_{i+1})-\alpha(x_i)$ instead of $x_{i+1}-x_i$, we obtain said integral as the upper/lower limits just like the Riemann integral.
        \item We write $\int_a^bf(x)\dd{\alpha(x)}$ where $\dd{\alpha(x)}=\alpha(x)\dd{x}$.
    \end{itemize}
    \item Theorem: If $\alpha$ is nondecreasing on $[a,b]$, $f_n\in R(\alpha)$ such that $f_n\rightrightarrows f$ on $[a,b]$
    \begin{itemize}
        \item We have
        \begin{align*}
            \left| \int f_n(x)\dd{\alpha(x)}-\int f(x)\dd{\alpha(x)} \right| &= \left| \int (f_n-f)(x)\dd{\alpha(x)}\right|\\
            &\leq \norm{f_n-f}(\alpha(b)-\alpha(a))\\
            &\leq \int|f_n-f|\dd{\alpha(x)}\\
            &\leq \int\norm{f_n-f}\dd{\alpha(x)}
            &\leq \norm{f_n-f}\int_a^b\dd{\alpha(x)}
            &= \norm{f_n-f}(\alpha(b)-\alpha(a))
        \end{align*}
    \end{itemize}
    \item \marginnote{11/19:}Suppose $f_n\to f$ and $f_n'\to g$. When does $f'=g$?
    \item Theorem: If $f_n:[a,b]\to\R$ is differentiable, $f_n(x_0)$ converges for some $x_0\in[a,b]$, and $f_n'$ converges uniformly on $[a,b]$, then there exists $f$ differentiable on $[a,b]$ such that $f_n\rightrightarrows f$ and $f_n'\rightrightarrows f'$.
    \begin{itemize}
        \item Assume the $f_n'$ are continuous. Then $f_n(x)-f_n(x_0)=\int_{x_0}^xf_n'(y)\dd{y}$.
        \item Since $f_n'\rightrightarrows g$, $\int_{x_0}^xf_n'(y)\dd{y}\to\int_{x_0}^xg(y)\dd{y}$.
        \item It follows since $f_n(x_0)\to f(x_0)$ that $f_n\rightrightarrows f$.
        \item By the previous theorem, if
        \begin{equation*}
            f_n'(x) = \lim_{h\to 0}\frac{f_n(x+h)-f_n(x)}{h}
        \end{equation*}
        then
        \begin{equation*}
            \lim_{n\to\infty}f_n'(x) = \lim_{h\to 0}\lim_{n\to\infty}\frac{f_n(x+h)-f_n(x)}{h} = \lim_{h\to 0}\frac{f(x+h)-f(x)}{h}
        \end{equation*}
        \item Fix $\epsilon>0$. Then there exists $N$ such that $n,m\geq N$ such that $|f_n(x_0)-f_m(x_0)|<\epsilon/2$ and $|f_n'(t)-f_m'(t)|<\epsilon/2$ for all $t\in[a,b]$.
        \item We know that $f_n(t)-f_n(x_0)=\int_{x_0}^tf_n'(y)\dd{y}$ and $f_m(t)-f_m(x_0)=\int_{x_0}^tf_m'(y)\dd{y}$.
        \item Thus,
        \begin{equation*}
            |f_n(t)-f_n(x_0)| \leq |f_n(t)-f_n(x_0)|+|f_m(t)-f_m(x_0)|-|f_m(t)-f_m(x_0)|
        \end{equation*}
        \item Let $f_n(t)-f_n(x_0)=c_n(t-x_0)$ and $f_m(t)-f_m(x_0)=c_m(t-x_0)$.
        \item ...
    \end{itemize}
    \item Let $f:[a,b]\to\R$ be continuous. What conditions on $f$ imply that $f'$ exists?
    \item Suppose $f$ is Lipschitz continuous (equivalent to saying there exists $L>0$ such that $|f(x)-f(y)|\leq L|x-y|$); then $f'$ exists \textbf{almost everywhere}.
    \begin{itemize}
        \item If $f$ differentiable, this is equivalent to saying $f$ bounded.
    \end{itemize}
    \item \textbf{Almost everywhere}: Something happens almost everywhere if the set of places where it doesn't happen has measure zero.
    \item Suppose $f$ is \textbf{H\"{o}lder continuous}, then $f'$ does not exist?
    \item \textbf{H\"{o}lder continuous} (function $f$): There exists $L>0$ such that $|f(y)-f(x)|<L|x-y|^\alpha$ where $\alpha\in(0,1)$
    \item Suppose $f$ exists such that $f$ is H\"{o}lder continuous in a neighborhood of every point in the domain. This function is not anywhere differentiable. Such a function does indeed exist (and it's Brownian motion). The construction of such a function is the essence of Stochastic analysis.
    \begin{itemize}
        \item Probabilistically: Has mean zero, distributed as a normal function like the Gaussian, and the increments are independent of each other.
        \item Analytically: It's a function that is H\"{o}lder continuous at half plus $\epsilon$ for every $\epsilon$ and it is nowhere differentiable.
    \end{itemize}
    \item Theorem: There exists $f:\R\to\R$ continuous but nowhere differentiable.
    \begin{itemize}
        \item This theorem is due to Weierstrass and as such, such functions are typically called Weierstrass functions.
    \end{itemize}
    \item A general class of functions that are nowhere differentiable (not in \textcite{bib:Rudin}; we don't have to prove this).
    \begin{itemize}
        \item Example 1:
        \begin{equation*}
            f(x) = \sum_{n=0}^\infty a^n\cos(b^n\pi x)
        \end{equation*}
        where $0<a<1$, $b$ positive odd integer greater than 1, and $ab>1+\frac{3}{2}\pi$.
        \begin{itemize}
            \item This function at every point oscillates more and more and more.
        \end{itemize}
    \end{itemize}
    \item \textcite{bib:Rudin}'s simple example.
    \begin{itemize}
        \item $\phi:[-1,1]\to\R$ defined by $\phi(x)=|x|$ is not differentiable at zero.
        \item Takes $\phi$ extends it periodically with period $2$, creating a sawtooth function.
        \item Repeat the behavior so that the nondifferentiability becomes more and more frequent to get
        \begin{equation*}
            f(x) = \sum_1^\infty\left( \frac{3}{4} \right)^n\phi(4^nx)
        \end{equation*}
        \item This is continuous.
        \item Fix any $x\in\R$, $m\in\N$. Then $\delta_m=\pm\frac{1}{2}\cdot 4^{-m}$.
        \item Then consider $4^mx$, $4^m(x+\delta_m)$.
        \item Rudin asserts
        \begin{equation*}
            \left| \frac{f(x+\delta_m)-f(x)}{\delta_m} \right| \to\infty
        \end{equation*}
        as $m\to\infty$ for all $x$.
    \end{itemize}
\end{itemize}




\end{document}