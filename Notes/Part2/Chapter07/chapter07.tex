\documentclass[../../notes.tex]{subfiles}

\pagestyle{main}
\renewcommand{\chaptermark}[1]{\markboth{\chaptername\ \thechapter\ (#1)}{}}
\setcounter{chapter}{6}

\begin{document}




\chapter{Sequences and Series of Functions}
\section{Notes}
\begin{itemize}
    \item \marginnote{11/15:}Soug will not test on differentiation/integration assuming that we know them already.
    \item \textbf{Pointwise convergent} (sequence $(f_n)_{n\in\N}$ of functions): A sequence of functions $f_n:E\to\R$ such that $\lim_{n\to\infty}f_n(x)=f(x)$ for all $x\in E$.
    \item Can we interchange "limit" in the above definition with continuity, convergence of series, integration, differentiation, etc.?
    \item Examples with negative answer:
    \begin{enumerate}
        \item Interchanging limits: Let $S_{mn}=\frac{m}{m+n}$. $S_{mn}\to 1$ as $m\to\infty$, and $S_{mn}\to 0$ as $n\to\infty$.
        \item $f_n(x)=x^2/(1+x)^n$. $f(x)=\sum_{n=1}^\infty f_n(x)$. If $x=0$, then $f_n(x)=0$ for all $n$ and $f(x)=0$. If $x\neq 0$, we have
        \begin{equation*}
            f(x) = \sum_{n=1}^\infty\frac{x^2}{(1+x^2)^n} = x^2\sum_{n=1}^\infty X^n = \frac{x^2}{1-X} = \frac{x^2}{1-(1/(1+x^2))} = 1+x^2
        \end{equation*}
        \item Consider $f_m(x)=\lim_{n\to\infty}(\cos(m\pi x))^2n$. $\lim_{m\to\infty}f_m(x)$ goes to 0 if $x\notin\Q$ and goes to 1 if $x\in\Q$. $f_m\to\chi_\Q$, where $\chi_\Q$ is the characteristic function of the rationals which is not Riemann integrable (partitions, upper and lower integrals, etc.).
        \item $f_n(x)=\sin nx/\sqrt{n}\to f(x)=0$ for all $x$. However, $f_n'(x)=\sqrt{n}\cos nx\nrightarrow 0$
        \item If $0\leq x\leq 1$, define $f_n(x)=n^2x(1-x^2)^n$. We know that $f_n(0)=0$. $\lim_{n\to\infty}f_n(x)=0$ for all $x\in(0,1]$. We can show that $\int_0^1x(1-x^2)^n\dd{x}=1/(2n+2)$. Thus, $\int_0^1f_n(x)\dd{x}=n^2/(2n+2)$. Limit of the functions is zero, but their integrals diverge to infinity.
    \end{enumerate}
    \item \textbf{Uniformly convergent} (sequence $(f_n)_{n\in\N}$ of functions on $E$): A sequence of functions $f_n:E\to\R$ such that for all $\epsilon>0$, there exists $N$ such that if $n\geq N$, then $|f_n(x)-f(x)|<\epsilon$ for all $x\in E$. \emph{Denoted by} $\bm{f_n\rightrightarrows f}$.
    \item Theorem: $f_n\rightrightarrows f$ iff $(f_n)_{n\in\N}$ is uniformly Cauchy (i.e., for all $\epsilon>0$, there exists $N$ such that if $n,m\geq N$ then $|f_n(x)-f_m(x)|<\epsilon$ for all $x\in E$).
    \begin{itemize}
        \item Let $M_n=\sup_{x\in E}|f_n(x)-f(x)|$. If $f_n\to f$ pointwise, then $f_n\rightrightarrows f$ if $M_n\to 0$.
    \end{itemize}
    \item Theorem: If $(f_n)_{n\in\N}$ and $|f_n(x)|\leq M_n$, then $\sum f_n\rightrightarrows f$ if $\sum M_n<\infty$.
    \item Theorem: If $E$ is a compact metric space, $f_n\rightrightarrows f$ in $E$, $x$ is a limit point of $E$, and $\lim_{t\to x}f_n(t)=A_n$ exists, then $(A_n)_{n\in\N}$ converges and $\lim_{t\to x}f(t)=\lim_{n\to\infty}A_n$.
    \item Corollary: $\lim_{t\to x}\lim_{n\to\infty}f_n(t)=\lim_{n\to\infty}\lim_{t\to x}f_n(t)$.
\end{itemize}




\end{document}