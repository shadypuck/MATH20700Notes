\documentclass[../../notes.tex]{subfiles}

\pagestyle{main}
\renewcommand{\chaptermark}[1]{\markboth{\chaptername\ \thechapter\ (#1)}{}}
\setcounter{chapter}{6}

\begin{document}




\chapter{Sequences and Series of Functions}
\section{Notes}
\begin{itemize}
    \item \marginnote{11/15:}Soug will not test on differentiation/integration assuming that we know them already.
    \item \textbf{Pointwise convergent} (sequence $(f_n)_{n\in\N}$ of functions): A sequence of functions $f_n:E\to\R$ such that $\lim_{n\to\infty}f_n(x)=f(x)$ for all $x\in E$.
    \item Can we interchange "limit" in the above definition with continuity, convergence of series, integration, differentiation, etc.?
    \item Examples with negative answer:
    \begin{enumerate}
        \item Interchanging limits: Let $S_{mn}=\frac{m}{m+n}$. $S_{mn}\to 1$ as $m\to\infty$, and $S_{mn}\to 0$ as $n\to\infty$.
        \item $f_n(x)=x^2/(1+x)^n$. $f(x)=\sum_{n=1}^\infty f_n(x)$. If $x=0$, then $f_n(x)=0$ for all $n$ and $f(x)=0$. If $x\neq 0$, we have
        \begin{equation*}
            f(x) = \sum_{n=1}^\infty\frac{x^2}{(1+x^2)^n} = x^2\sum_{n=1}^\infty X^n = \frac{x^2}{1-X} = \frac{x^2}{1-(1/(1+x^2))} = 1+x^2
        \end{equation*}
        \item Consider $f_m(x)=\lim_{n\to\infty}(\cos(m\pi x))^2n$. $\lim_{m\to\infty}f_m(x)$ goes to 0 if $x\notin\Q$ and goes to 1 if $x\in\Q$. $f_m\to\chi_\Q$, where $\chi_\Q$ is the characteristic function of the rationals which is not Riemann integrable (partitions, upper and lower integrals, etc.).
        \item $f_n(x)=\sin nx/\sqrt{n}\to f(x)=0$ for all $x$. However, $f_n'(x)=\sqrt{n}\cos nx\nrightarrow 0$
        \item If $0\leq x\leq 1$, define $f_n(x)=n^2x(1-x^2)^n$. We know that $f_n(0)=0$. $\lim_{n\to\infty}f_n(x)=0$ for all $x\in(0,1]$. We can show that $\int_0^1x(1-x^2)^n\dd{x}=1/(2n+2)$. Thus, $\int_0^1f_n(x)\dd{x}=n^2/(2n+2)$. Limit of the functions is zero, but their integrals diverge to infinity.
    \end{enumerate}
    \item \textbf{Uniformly convergent} (sequence $(f_n)_{n\in\N}$ of functions on $E$): A sequence of functions $f_n:E\to\R$ such that for all $\epsilon>0$, there exists $N$ such that if $n\geq N$, then $|f_n(x)-f(x)|<\epsilon$ for all $x\in E$. \emph{Denoted by} $\bm{f_n\rightrightarrows f}$.
    \item Theorem: $f_n\rightrightarrows f$ iff $(f_n)_{n\in\N}$ is uniformly Cauchy (i.e., for all $\epsilon>0$, there exists $N$ such that if $n,m\geq N$ then $|f_n(x)-f_m(x)|<\epsilon$ for all $x\in E$).
    \begin{itemize}
        \item Let $M_n=\sup_{x\in E}|f_n(x)-f(x)|$. If $f_n\to f$ pointwise, then $f_n\rightrightarrows f$ if $M_n\to 0$.
    \end{itemize}
    \item Theorem: If $(f_n)_{n\in\N}$ and $|f_n(x)|\leq M_n$, then $\sum f_n\rightrightarrows f$ if $\sum M_n<\infty$.
    \item Theorem: If $E$ is a compact metric space, $f_n\rightrightarrows f$ in $E$, $x$ is a limit point of $E$, and $\lim_{t\to x}f_n(t)=A_n$ exists, then $(A_n)_{n\in\N}$ converges and $\lim_{t\to x}f(t)=\lim_{n\to\infty}A_n$.
    \item Corollary: $\lim_{t\to x}\lim_{n\to\infty}f_n(t)=\lim_{n\to\infty}\lim_{t\to x}f_n(t)$.
    \begin{itemize}
        \item \marginnote{11/16:}Fix $\epsilon>0$. Then $f_n\rightrightarrows f$ implies that there exists some $N$ such that $n,m\geq N$ implies $|f_n(t)-f_m(t)|<\epsilon$ for all $t\in E$.
        \item $x$ is a limit point of $E$ and $t\to x$ implies $|A_n-A_m|<\epsilon$. Thus, $(A_n)_{n\in\N}$ is cauchy, so there exists $A$ such that $A_n\to A$.
        \item WTS: $|f(t)-A|\leq|f(t)-f_n(t)|+|f_n(t)-A_n|+|A_n-A|$, so we WTS the three terms on the right are small.
        \item There exists $n$ such that $|f(t)-f_n(t)|<\epsilon/3$ for all $t$ since $f_n\rightrightarrows f$ by hypothesis.
        \item Since $t$ is in a small neighborhood of $x$, there exists $n$ such that $|A_n-A|<\epsilon/3$.
        \item We also have $|f_n(t)-A_n|<\epsilon/3$ by hypothesis.
        \item This is a very important proof to understand, because proofs like this pop up often.
    \end{itemize}
    \item Corollary: $f_n$ continuous and $f_n\rightrightarrows f$ implies $f$ is continuous.
    \item Theorem: Let $K$ be compact. Assume
    \begin{enumerate}[label={(\alph*)}]
        \item $(f_n)_{n\in\N}\subset C(K)=\{f:K\to\R\mid f\text{ continuous}\}$.
        \item $f_n\to f$ pointwise in $K$ and $f\in C(K)$.
        \item $f_n(x)\geq f_{n+1}(x)$ for all $x\in K$.
    \end{enumerate}
    Then $f_n\rightrightarrows f$.
    \begin{itemize}
        \item WLOG $f=0$, $g_n=f_n-f\to 0$, $g_n\geq g_{n+1}\geq 0$.
        \item For all $\epsilon>0$, there exists $N$ such that $n\geq N$ and $0\leq g_n(x)\leq\epsilon$ for all $x\in K$.
        \item $K_n=\{x\in K:g_n(x)\geq\epsilon\}$.
        \item $g_n$ continuous implies $K_n$ closed. This combined with $K$ compact implies $K_n$ is compact.
        \item $g_n$ decreasing implies $K_n\supset K_{n+1}$. Thus, $K_n$ is a nested family of compact sets, so $\bigcap K_n\neq\emptyset$.
        \item This implies that each $K_n$ is nonempty, contradicting the fact that each $g_n\to 0$ for all $x$.
        \item Thus, there exists an $N$ such that $K_n$ is empty for all $n\geq N$. Thus $g_n(x)\leq\epsilon$ for all $x\in K$, $n\geq N$.
        \item Note that the compactness of $K$ is important. If $f:(0,1)\to\R$ is defined by $f(x)=1/(nx+1)$, then $f_n\to 0$, but $f_n\not\rightrightarrows f$.
    \end{itemize}
    \item Let $C(X)=\{f:X\to\R\mid f\text{ continuous, bounded}\}$ for $X$ a metric space.
    \item If we define $\norm{f}=\sum_{x\in X}|f(x)|$, for $f,g\in C(X)$, we may define $d(f,g)=\norm{f-g}$. This definition satisfies the properties of a distance function, and $\norm{\cdot}$ is a norm.
    \begin{itemize}
        \item Thus, $C(X)$ is a complete metric space, a normed space, or, specifically, a \textbf{Banach space}.
    \end{itemize}
    \item Theorem: $(f_n)_{n\in\N}\subset C(X)$ such that $\norm{f_n-f_m}_{n,m\to\infty}\to 0$. Then there exists $f\in C(X)$ such that $\norm{f_n-f}_{n\to\infty}\to 0$.
    \begin{itemize}
        \item We get such a strong statement using properties of the image, not properties of the domain.
        \item For all $\epsilon>0$, there exists $N$ such that $n,m\geq N$.
        \item $|f_n(x)-f_m(x)|\leq\norm{f_n-f_m}<\epsilon$ for all $x$.
        \item Then there exists $f$ such that $f_m(x)\to f(x)$. It follows that $|f_n(x)-f_m(x)|<\epsilon$
    \end{itemize}
    \item Uniform convergence and integration.
    \item Stieltjes integral.
    \begin{itemize}
        \item Define $\alpha:\R\to\R$ nondecreasing.
        \item If we sum over the minimums/maximums of a partition times $\alpha(x_{i+1})-\alpha(x_i)$ instead of $x_{i+1}-x_i$, we obtain said integral as the upper/lower limits just like the Riemann integral.
        \item We write $\int_a^bf(x)\dd{\alpha(x)}$ where $\dd{\alpha(x)}=\alpha(x)\dd{x}$.
    \end{itemize}
    \item Theorem: If $\alpha$ is nondecreasing on $[a,b]$, $f_n\in R(\alpha)$ such that $f_n\rightrightarrows f$ on $[a,b]$
    \begin{itemize}
        \item We have
        \begin{align*}
            \left| \int f_n(x)\dd{\alpha(x)}-\int f(x)\dd{\alpha(x)} \right| &= \left| \int (f_n-f)(x)\dd{\alpha(x)}\right|\\
            &\leq \norm{f_n-f}(\alpha(b)-\alpha(a))\\
            &\leq \int|f_n-f|\dd{\alpha(x)}\\
            &\leq \int\norm{f_n-f}\dd{\alpha(x)}
            &\leq \norm{f_n-f}\int_a^b\dd{\alpha(x)}
            &= \norm{f_n-f}(\alpha(b)-\alpha(a))
        \end{align*}
    \end{itemize}
    \item \marginnote{11/19:}Suppose $f_n\to f$ and $f_n'\to g$. When does $f'=g$?
    \item Theorem: If $f_n:[a,b]\to\R$ is differentiable, $f_n(x_0)$ converges for some $x_0\in[a,b]$, and $f_n'$ converges uniformly on $[a,b]$, then there exists $f$ differentiable on $[a,b]$ such that $f_n\rightrightarrows f$ and $f_n'\rightrightarrows f'$.
    \begin{itemize}
        \item Assume the $f_n'$ are continuous. Then $f_n(x)-f_n(x_0)=\int_{x_0}^xf_n'(y)\dd{y}$.
        \item Since $f_n'\rightrightarrows g$, $\int_{x_0}^xf_n'(y)\dd{y}\to\int_{x_0}^xg(y)\dd{y}$.
        \item It follows since $f_n(x_0)\to f(x_0)$ that $f_n\rightrightarrows f$.
        \item By the previous theorem, if
        \begin{equation*}
            f_n'(x) = \lim_{h\to 0}\frac{f_n(x+h)-f_n(x)}{h}
        \end{equation*}
        then
        \begin{equation*}
            \lim_{n\to\infty}f_n'(x) = \lim_{h\to 0}\lim_{n\to\infty}\frac{f_n(x+h)-f_n(x)}{h} = \lim_{h\to 0}\frac{f(x+h)-f(x)}{h}
        \end{equation*}
        \item Fix $\epsilon>0$. Then there exists $N$ such that $n,m\geq N$ such that $|f_n(x_0)-f_m(x_0)|<\epsilon/2$ and $|f_n'(t)-f_m'(t)|<\epsilon/2$ for all $t\in[a,b]$.
        \item We know that $f_n(t)-f_n(x_0)=\int_{x_0}^tf_n'(y)\dd{y}$ and $f_m(t)-f_m(x_0)=\int_{x_0}^tf_m'(y)\dd{y}$.
        \item Thus,
        \begin{equation*}
            |f_n(t)-f_n(x_0)| \leq |f_n(t)-f_n(x_0)|+|f_m(t)-f_m(x_0)|-|f_m(t)-f_m(x_0)|
        \end{equation*}
        \item Let $f_n(t)-f_n(x_0)=c_n(t-x_0)$ and $f_m(t)-f_m(x_0)=c_m(t-x_0)$.
        \item ...
    \end{itemize}
    \item Let $f:[a,b]\to\R$ be continuous. What conditions on $f$ imply that $f'$ exists?
    \item Suppose $f$ is Lipschitz continuous (equivalent to saying there exists $L>0$ such that $|f(x)-f(y)|\leq L|x-y|$); then $f'$ exists \textbf{almost everywhere}.
    \begin{itemize}
        \item If $f$ differentiable, this is equivalent to saying $f$ bounded.
    \end{itemize}
    \item \textbf{Almost everywhere}: Something happens almost everywhere if the set of places where it doesn't happen has measure zero.
    \item Suppose $f$ is \textbf{H\"{o}lder continuous}, then $f'$ does not exist?
    \item \textbf{H\"{o}lder continuous} (function $f$): There exists $L>0$ such that $|f(y)-f(x)|<L|x-y|^\alpha$ where $\alpha\in(0,1)$
    \item Suppose $f$ exists such that $f$ is H\"{o}lder continuous in a neighborhood of every point in the domain. This function is not anywhere differentiable. Such a function does indeed exist (and it's Brownian motion). The construction of such a function is the essence of Stochastic analysis.
    \begin{itemize}
        \item Probabilistically: Has mean zero, distributed as a normal function like the Gaussian, and the increments are independent of each other.
        \item Analytically: It's a function that is H\"{o}lder continuous at half plus $\epsilon$ for every $\epsilon$ and it is nowhere differentiable.
    \end{itemize}
    \item Theorem: There exists $f:\R\to\R$ continuous but nowhere differentiable.
    \begin{itemize}
        \item This theorem is due to Weierstrass and as such, such functions are typically called Weierstrass functions.
    \end{itemize}
    \item A general class of functions that are nowhere differentiable (not in \textcite{bib:Rudin}; we don't have to prove this).
    \begin{itemize}
        \item Example 1:
        \begin{equation*}
            f(x) = \sum_{n=0}^\infty a^n\cos(b^n\pi x)
        \end{equation*}
        where $0<a<1$, $b$ positive odd integer greater than 1, and $ab>1+\frac{3}{2}\pi$.
        \begin{itemize}
            \item This function at every point oscillates more and more and more.
        \end{itemize}
    \end{itemize}
    \item \textcite{bib:Rudin}'s simple example.
    \begin{itemize}
        \item $\phi:[-1,1]\to\R$ defined by $\phi(x)=|x|$ is not differentiable at zero.
        \item Takes $\phi$ extends it periodically with period $2$, creating a sawtooth function.
        \item Repeat the behavior so that the nondifferentiability becomes more and more frequent to get
        \begin{equation*}
            f(x) = \sum_1^\infty\left( \frac{3}{4} \right)^n\phi(4^nx)
        \end{equation*}
        \item This is continuous.
        \item Fix any $x\in\R$, $m\in\N$. Then $\delta_m=\pm\frac{1}{2}\cdot 4^{-m}$.
        \item Then consider $4^mx$, $4^m(x+\delta_m)$.
        \item Rudin asserts
        \begin{equation*}
            \left| \frac{f(x+\delta_m)-f(x)}{\delta_m} \right| \to\infty
        \end{equation*}
        as $m\to\infty$ for all $x$.
    \end{itemize}
    \item \marginnote{11/29:}Finding a uniformly convergent subsequence of a sequence of functions.
    \begin{itemize}
        \item Pointwise, uniformly, bounded if there exist $M_x$ such that $|f_n(x)|\leq M_x$ for all $n,x$. Uniformly bounded if there exists $M$ such that $|f_n(x)|\leq M$ for all $n,x$.
    \end{itemize}
    \item Theorem: If $(f_n)_{n\in\N}$ is pointwise bounded and $E\subset X$ is countable, then there exists a subsequence $f_{n_k}$ which converges for every $x\in E$.
    \begin{itemize}
        \item Let $E=\{x_i:i\in\N\}$. Consider $f_n(x_1)$. $f_{1,k}(x_1)$ converges.
        \item $S_1:f_{1,1}(x_1),f_{1,2}(x_1),f_{1,3}(x_1),f_{1,4}(x_1),\dots$.
        \item $S_2:f_{2,1}(x_2),f_{2,2}(x_2),f_{2,3}(x_2),f_{2,4}(x_2),\dots$.
        \item Now consider $f_{2,k}(x_3)$.
        \item $S_3:f_{3,1},f_{3,2},f_{3,3},f_{3,4},\dots$.
        \item Continue on and on to $S_4,S_5,\dots$. We know that each of these sequences converges pointwise by hypothesis.
        \item Now consider the diagonal sequence $f_{1,1},f_{2,2},f_{3,3},f_{4,4}$.
        \begin{itemize}
            \item This subsequence of the original sequence we may call $g_k$.
            \item We posit that $g_k$ converges for every $x\in E$. 
        \end{itemize}
    \end{itemize}
    \item Theorem: There exists $f_n$ which is uniformly bounded but does not converge uniformly.
    \begin{itemize}
        \item Let $f_n(x)=\sin(2\pi x)$ for $0\leq x\leq 2\pi$.
        \item Let $f_n(x)=x^2/(x^2+(1-nx)^2)$ on $0\leq x\leq 1$. This sequence is uniformly bounded, converges pointwise, but $f_n(1/n)=1$ so $f_n$ cannot converge uniformly to zero.
    \end{itemize}
    \item What does it mean that $f_n:[0,1]\to\R$ does not converge uniformly?
    \begin{itemize}
        \item It means that there exists a subsequence of the functions evaluated at certain points that is always greater than or equal to some fixed distance away from the limit.
    \end{itemize}
    \item Equicontinuity: $\mathcal{F}\{f:X\to\R\}$ for $(X,d)$ a metric space is equicontinuous iff for all $\epsilon>0$, there exists a $\delta>0$ such that $d(x,y)<\delta$ implies $|f(x)-f(y)|<\epsilon$ for all $x,y\in X$, $f\in\mathcal{F}$.
    \item Modulus of continuity: $f:X\to\R$ is continuous at $x$. A modulus of continuity is a function $\omega_X:[0,1]\to[0,1]$ such that $|f(y)-f(x)|\leq\omega_X|y-x|$.
    \item The final result we'll prove: \textbf{Arzel\`{a}-Ascoli theorem}: If we have a family of functions on a compact set and we have a dense subset of that set, then if we have a sequence of functions that are equicontinuous, then they converge uniformly.
    \item \marginnote{12/1:}The final will be in this room.
    \item The last PSet will not be graded, but there will be similar questions on the final.
    \item No class on Friday.
    \item Review from last time:
    \begin{itemize}
        \item Equiboundedness and equicontinuity.
        \item If $E$ is a dense subset of $X$, then any pointwise bounded sequence has a subsequence that converges on $E$ (diagonal argument.)
    \end{itemize}
    \item \textbf{Equicontinuous} (sequence $\{f_n\}$): For all $\epsilon>0$, there exists a $\delta>0$ such that $d(x,y)<\delta$ implies $|f_n(x)-f_n(y)|<\epsilon$ for all $n$.
    \item Theorem: If $K$ is a compact set and $\{f_n\}\in C(K)$ converges uniformly on $K$, then the $f_n$'s are equicontinuous on $K$.
    \begin{itemize}
        \item The $f_n$ are uniformly Cauchy: For all $\epsilon>0$, there exists $N$ such that $n,m\geq N$ imply $\norm{f_n-f_m}<\epsilon$ where $\norm{f_n-f_m}=\sup_{x\in K}(f_n-f_m)(x)$.
        \item If $n\geq N$, then $|f_n(x)-f_n(y)|\leq|f_N(x)-f_N(y)|+2\norm{f_n-f_N}$ (since $f_n(x)-f_n(y)=f_n(x)-f_N(x)+f_N(x)-f_N(y)+f_N(y)-f_n(y)$).
        \item Thus $|f_n(x)-f_n(y)|\leq|f_N(x)-f_N(y)|+2\norm{f_n-f_N}<3\epsilon$ implies $|f_i(x)-f_i(y)|<\epsilon$ if $|x-y|<\delta$ for $i=1,\dots,N$.
    \end{itemize}
    \item \textbf{Arzel\`{a}-Ascoli theorem}: If $K$ is compact, $(f_n)_{n\geq 1}\subset C(K)$ which are pointwise bounded and equicontinuous, then
    \begin{enumerate}[label={(\alph*)}]
        \item $(f_n)_{n\geq 1}$ are uniformly bounded (equicontinuous).
        \item There exists $(f_{n_k})_{k\geq 1}$ which converges uniformly on $K$.
    \end{enumerate}
    \begin{itemize}
        \item Since $K$ is compact, you can cover it by finitely many balls of radius $\delta$.
        \item Thus $|f_n(p_k)|\leq M=\max(M_{p_1},\dots,M_{p_k})$ where $K\subset\bigcup_{k\in K}B(p_{n_k},\delta)$.
        \item $K$ has a countable dense subset $E$ (Exercise 2.25).
        \item $|f_n(x)|\leq M+\epsilon$ for all $x$.
        \item $\{B(x,\delta)\}_{x\in E}$ is an open cover of $K$.
        \item Thus has a finite subcover.
        \item ...
    \end{itemize}
    \item If $f_n:\R^n\to\R$ are continuous, equibounded, equicontinuous, then there exists $f_{n_k}$ which converges locally uniformly to some $f:\R^n\to\R$.
    \item How do you learn math?
    \begin{itemize}
        \item In an ideal world, study by looking at theorems, thinking that you should be able to prove it, and etc.
        \item Since we don't have the time to do everything ourselves, don't just get stuck in a place; move on and continue thinking if you have to.
    \end{itemize}
    \item Let $\dot{\phi}=f(x,t)$ and $x(0)=c$. Let $\phi:\R\times[0,1]\to\R$. Assume $\phi$ is bounded and continuous. Then there exists a solution of the differential equation and initial condition.
    \begin{itemize}
        \item We need to find a function $x:[0,1]\to\R$ continuous such that $x(t)=c+\int_0^t\phi(x(s),s)\dd{s}$.
        \item Let $t_i=i/N$. Then $x_n(t)=\phi(x_i,t_i)$ on $t_i<t<t_{i+1}$.
        \item $x_n(t)=x_n(t_i)+\phi(x_i,t_i)(t-t_i)$.
        \item $\frac{x_{i+1}-x_i}{1/N}=\phi(x_i,t_i)$.
        \item $\Delta_n(t)=x_n'(t)-\phi(x_n(t),t)$ for $\phi(x_i,t_i)-\phi(x_n(t),t)$ measures how close our solution is.
        \item All of these things imply that our final formula is
        \begin{equation*}
            x_n(t) = c+\int_0^t[\phi(x_n(s),s)+\Delta_n(s)]\dd{s}
        \end{equation*}
        \item If we know that $x_n\rightrightarrows x$, then $\Delta_n\rightrightarrows 0$.
        \item We then use the A-A theorem to imply convergence.
    \end{itemize}
    \item When we get to MATH 208, say we didn't do any multivariable calculus in MATH 207.
    \begin{itemize}
        \item We didn't do how to integrate in $\R^n$, how to integrate by parts (Stoke's theorem), Lagrange multipliers (constraint minimization).
    \end{itemize}
    \item Problem 4.23: Show the inequalities at the bottom first and then use those to show continuity.
    \begin{itemize}
        \item Consider $\lim_{t\to u}f(t)$. Approach from two sides separately and show cancellation??? Chloe will write a solution.
        \item This is a particular trick for convex functions; it's not exactly recyclable.
    \end{itemize}
    \item The trick for Problem 4.26 is recyclable.
    \item Linear algebra questions on the final are easier than the midterm.
    \begin{itemize}
        \item The last question will be the hard one this time.
    \end{itemize}
\end{itemize}



\section{Sequences and Series of Functions}
\emph{From \textcite{bib:Rudin}.}
\begin{itemize}
    \item \marginnote{12/6:}Let $f$ be a complex-valued function.
    \item \textbf{Limit} (of $\{f_n\}$): The function $f:E\to\C$ defined by
    \begin{equation*}
        f(x) = \lim_{n\to\infty}f_n(x)
    \end{equation*}
    for all $x\in E$, where $\{f_n\}$ is a sequence of functions such that $\lim_{n\to\infty}f_n(x)$ exists for all $x\in E$. \emph{Also known as} \textbf{limit function}.
    \item \textbf{Sum} (of $\{f_n\}$): The function $f:E\to\C$ defined by
    \begin{equation*}
        f(x) = \sum_{n=1}^\infty f_n(x)
    \end{equation*}
    for all $x\in E$, where $\{f_n\}$ is a sequence of functions such that $\sum_{n=1}^\infty f_n(x)$ exists for all $x\in E$.
    \item Motivation for this chapter: Which properties of functions are preserved under the limit and summation operations?
    \item Continuity example:
    \begin{itemize}
        \item A function is continuous at $x$ if $\lim_{t\to x}f(t)=f(x)$.
        \item Hence, the limit of a sequence of continuous functions is continuous at $x$ if
        \begin{equation*}
            \lim_{t\to x}\lim_{n\to\infty}f_n(t) = \lim_{n\to\infty}\lim_{t\to x}f_n(t)
        \end{equation*}
        \item This switching of the limits is not always possible: If $s_{m,n}=m/(m+n)$, then
        \begin{equation*}
            \lim_{n\to\infty}\lim_{m\to\infty}s_{m,n} = 1
            \neq 0
            = \lim_{m\to\infty}\lim_{n\to\infty}s_{m,n}
        \end{equation*}
        \item For an example of a sequence of continuous functions converging to a discontinuous function, see the example 2 (11/15 class notes).
    \end{itemize}
    \item \textbf{Pointwise convergent} (sequence $\{f_n\}$): A sequence of functions $\{f_n\}$ for which there exists a function $f$ such that for every $\epsilon>0$ and for every $x\in E$, there exists an integer $N$ such that if $n\geq N$, then
    \begin{equation*}
        |f_n(x)-f(x)| < \epsilon
    \end{equation*}
    \item \textbf{Uniformly convergent} (sequence $\{f_n\}$): A sequence of functions $\{f_n\}$ for which there exists a function $f$ such that for every $\epsilon>0$, there exists an integer $N$ such that if $n\geq N$, then
    \begin{equation*}
        |f_n(x)-f(x)| < \epsilon
    \end{equation*}
    for all $x\in E$.
    \item Theorem 7.8: Cauchy criterion for uniform convergence.
    \item Theorem 7.9: If $\lim_{n\to\infty}f_n(x)=f(x)$ for all $x\in E$ and $M_n=\sup_{x\in E}|f_n(x)-f(x)|$, then $f_n\to f$ uniformly on $E$ iff $M_n\to 0$ as $n\to\infty$.
    \item Theorem 7.10: $|f_n(x)|\leq M_n$ for all $x\in E$ and $\sum M_n$ converges implies $\sum f_n$ converges uniformly.
    \item Theorem 7.11: Suppose $f_n\to f$ uniformly on a set $E$ in a metric space. Let $x$ be a limit point of $E$, and suppose that
    \begin{equation*}
        \lim_{t\to x}f_n(t) = A_n
    \end{equation*}
    for each $n\in\N$. Then $\{A_n\}$ converges and
    \begin{equation*}
        \lim_{t\to x}f(t) = \lim_{n\to\infty}A_n
    \end{equation*}
    \begin{proof}
        See IBL Theorem 17.6.
    \end{proof}
    \begin{itemize}
        \item It follows that in this case,
        \begin{equation*}
            \lim_{t\to x}\lim_{n\to\infty}f_n(t) = \lim_{n\to\infty}\lim_{t\to x}f_n(t)
        \end{equation*}
    \end{itemize}
    \item Theorem 7.12: A uniformly convergent sequence of continuous functions converges to a continuous function.
    \item Theorem 7.13: $K$ compact, $\{f_n\}$ a sequence of continuous functions on $K$ that converges pointwise to a continuous function $f$ on $K$, and $f_n(x)\geq f_{n+1}(x)$ for all $x\in K$, $n\in\N$ implies $f_n\to f$ uniformly on $K$.
    \item $\pmb{\mathscr{C}}\bm{(X)}$: The set of all complex-valued, continuous, bounded functions with domain $X$ a metric space.
    \item \textbf{Supremum norm} (of $f\in\mathscr{C}(X)$): The following value. \emph{Denoted by} $\pmb{\norm{f}}$. \emph{Given by}
    \begin{equation*}
        \norm{f} = \sup_{x\in X}|f(x)|
    \end{equation*}
    \item Properties of the supremum norm.
    \begin{itemize}
        \item $\norm{f}<\infty$ ($f$ is bounded).
        \item $\norm{f}=0$ iff $f=0$.
        \item $\norm{f+g}\leq\norm{f}+\norm{g}$.
    \end{itemize}
    \item The above properties plus the definition $d(f,g)=\norm{f-g}$ for any $f,g\in\mathscr{C}(X)$ makes $\mathscr{C}(X)$ a metric space!
    \item Rephrasing Theorem 7.9: A sequence $\{f_n\}$ converges to $f$ with respect to the metric of $\mathscr{C}(X)$ if and only if $f_n\to f$ uniformly on $X$.
    \begin{itemize}
        \item Thus, closed subsets $\mathscr{A}\subset\mathscr{C}(X)$ are sometimes called \textbf{uniformly closed}.
        \item The closure of a subset $\mathscr{A}\subset\mathscr{C}(X)$ can similarly be called the \textbf{uniform closure}.
    \end{itemize}
    \item Theorem 7.15: The above metric makes $\mathscr{C}(X)$ into a complete metric space.
    \item Theorem 7.16: $f_n\in\mathscr{R}(\alpha)$ on $[a,b]$ and $f_n\to f$ uniformly on $[a,b]$ imply $f\in\mathscr{R}(\alpha)$ on $[a,b]$ and
    \begin{equation*}
        \int_a^bf\dd{\alpha} = \lim_{n\to\infty}\int_a^bf_n\dd{\alpha}
    \end{equation*}
    \begin{proof}
        See IBL Theorem 17.7; \textcite{bib:Rudin}'s is much slicker though.
    \end{proof}
    \item Theorem 7.17: Suppose $\{f_n\}$ are differentiable on $[a,b]$, $\{f_n(x_0)\}$ converges for some $x_0\in[a,b]$, and $\{f_n'\}$ converges uniformly on $[a,b]$. Then $\{f_n\}$ converges uniformly on $[a,b]$ to a function $f$ such that
    \begin{equation*}
        f'(x) = \lim_{n\to\infty}f_n'(x)
    \end{equation*}
    for all $x\in[a,b]$.
    \begin{proof}
        Long and complicated, and given in class.
    \end{proof}
    \begin{itemize}
        \item Assuming the continuity of the $\{f_n'\}$ admits a proof like IBL Theorem 17.8.
    \end{itemize}
    \item Theorem 7.18: There exists a continuous function $f:\R\to\R$ which is nowhere differentiable.
    \begin{proof}
        Long and complicated, and given in class.
    \end{proof}
    \item \textbf{Pointwise bounded} (sequence $\{f_n\}$): A sequence of functions $\{f_n\}$ for which there exists a finite-valued function $\phi$ defined on $E$ such that $|f_n(x)|<\phi(x)$ for all $x\in E$.
    \item \textbf{Uniformly bounded} (sequence $\{f_n\}$): A sequence of functions $\{f_n\}$ for which there exists a number $M$ such that $|f_n(x)|<M$ for all $x\in E$.
    \item \textbf{Equicontinuous} (family of functions $\mathscr{F}$): A family of functions $\mathscr{F}$ such that for every $\epsilon>0$, there exists a $\delta>0$ such that
    \begin{equation*}
        |f(x)-f(y)| < \epsilon
    \end{equation*}
    whenever $d(x,y)<\delta$, $x,y\in E$, and $f\in\mathscr{F}$.
    \begin{itemize}
        \item Every member of an equicontinuous family is uniformly continuous.
    \end{itemize}
    \item Theorem 7.23: $\{f_n\}$ pointwise bounded and defined on a countable set $E$ implies $\{f_n\}$ has a pointwise convergent subsequence $\{f_{n_k}\}$.
    \begin{proof}
        Diagonalization argument from class.
    \end{proof}
    \item Theorem 7.24: $K$ compact and $\{f_n\}\subset\mathscr{C}(K)$ uniformly convergent implies $\{f_n\}$ equicontinuous on $K$.
    \item Theorem 7.25 (Arzel\`{a}-Ascoli Theorem): If $K$ is compact and $\{f_n\}\subset\mathscr{C}(K)$ is pointwise bounded and equicontinuous on $K$, then
    \begin{enumerate}
        \item $\{f_n\}$ is uniformly bounded on $K$;
        \item $\{f_n\}$ contains a uniformly convergent subsequence.
    \end{enumerate}
    \item Theorem 7.26 (Weierstrass Approximation Theorem): $f$ continuous on $[a,b]$ implies there exists a sequence of polynomials $P_n$ such that
    \begin{equation*}
        \lim_{n\to\infty}P_n(x) = f(x)
    \end{equation*}
    uniformly on $[a,b]$.
    \item Corollary 7.27: For every interval $[-a,a]$, there is a sequence of real polynomials $\{P_n\}$ such that $P_n(0)=0$ and such that
    \begin{equation*}
        \lim_{n\to\infty}P_n(x) = |x|
    \end{equation*}
    uniformly on $[-a,a]$.
    \item \textbf{Algebra}: A family $\mathscr{A}$ of complex functions defined on a set $E$ such that for all $f,g\in\mathscr{A}$ and $c\in\C$,
    \begin{enumerate}[label={(\roman*)}]
        \item $f+g\in\mathscr{A}$;
        \item $fg\in\mathscr{A}$;
        \item $cf\in\mathscr{A}$.
    \end{enumerate}
    \item \textbf{Uniformly closed} (algebra $\mathscr{A}$): An algebra $\mathscr{A}$ such that if $\{f_n\}\subset\mathscr{A}$ and $f_n\to f$ uniformly on $E$, then $f\in\mathscr{A}$.
    \item \textbf{Uniform closure} (of an algebra $\mathscr{A}$): The set $\mathscr{B}$ of all functions which are limits of uniformly convergent sequences of members of $\mathscr{A}$.
    \item Rephrasing Theorem 7.26: The set of all continuous functions on $[a,b]$ is the uniform closure of the set of polynomials on $[a,b]$.
    \item Theorem 7.29: Let $\mathscr{B}$ be the uniform closure of an algebra $\mathscr{A}$ of bounded functions. Then $\mathscr{B}$ is a uniformly closed algebra.
    \item \textbf{Separating points} (family $\mathscr{A}$ on $E$): A family of functions $\mathscr{A}$ on a set $E$ such that to every pair of distinct points $x_1,x_2\in E$, there corresponds a function $f\in\mathscr{A}$ such that $f(x_1)\neq f(x_2)$.
    \begin{itemize}
        \item Example of a family that separates points on $\R^1$: the algebra of all polynomials in one variable.
        \item Example of a family that does not separate points on $[-1,1]$: the set of all even polynomials in one variable (since $f(x)=f(-x)$ for every even function $f$).
    \end{itemize}
    \item \textbf{Vanishing at no point} (family $\mathscr{A}$ on $E$): A family of functions $\mathscr{A}$ such that to each $x\in E$, there corresponds a function $g\in\mathscr{A}$ such that $g(x)\neq 0$.
    \item Theorem 7.31: $\mathscr{A}$ an algebra on $E$ that separates points and vanishes at no point, $x_1,x_2\in E$ distinct, and $c_1,c_2\in\C$ imply $\mathscr{A}$ contains a function $f$ such that
    \begin{align*}
        f(x_1) &= c_1&
        f(x_2) &= c_2
    \end{align*}
    \item Theorem 7.32 (Stone-Weierstrass Theorem): $\mathscr{A}$ an algebra of real continuous functions on $K$ compact that separates points of $K$ and vanishes at no point of $K$ implies the uniform closure $\mathscr{B}$ of $\mathscr{A}$ consists of all real continuous functions on $K$.
    \begin{itemize}
        \item Theorem 7.32 holds on complex algebras with the additional hypothesis that $\mathscr{A}$ is \textbf{self-adjoint} (see Theorem 7.33).
    \end{itemize}
    \item \textbf{Self-adjoint} (algebra $\mathscr{A}$): An algebra $\mathscr{A}$ such that if $f\in\mathscr{A}$, then its \textbf{complex conjugate} $\bar{f}\in\mathscr{A}$.
    \item \textbf{Complex conjugate} (of $f$): The function $\bar{f}$ defined by $\bar{f}(x)=\overline{f(x)}$.
    \item Theorem 7.33: $\mathscr{A}$ a self-adjoint algebra of complex continuous functions on $K$ compact that separates points of $K$ and vanishes at no point of $K$ implies the uniform closure $\mathscr{B}$ of $\mathscr{A}$ consists of all complex continuous functions on $K$.
    \begin{itemize}
        \item In other words, $\mathscr{A}$ is dense in $\mathscr{C}(K)$.
    \end{itemize}
\end{itemize}




\end{document}