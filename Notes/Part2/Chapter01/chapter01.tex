\documentclass[../../notes.tex]{subfiles}

\pagestyle{main}
\renewcommand{\chaptermark}[1]{\markboth{\chaptername\ \thechapter\ (#1)}{}}

\begin{document}




\chapter{The Real and Complex Number Systems}
\section{Notes}
\begin{itemize}
    \item \marginnote{11/1:}Spent a lot of time trying to cheer us up regarding the midterm.
    \item There may be some true/false on linear algebra on the final.
    \item Facts:
    \begin{enumerate}
        \item $\sqrt{2}$ is irrational.
        \item Archimedes principle: If $x>0$ and $y\in\R$, then there exists $n$ such that $nx>y$.
        \item If $x>y$, then there exists $q\in\Q$ such that $x>q>y$.
    \end{enumerate}
\end{itemize}



\section{Chapter 1: The Real and Complex Number Systems}
\emph{From \textcite{bib:Rudin}.}
\begin{itemize}
    \item \marginnote{11/6:}\textcite{bib:Rudin} presents several interesting proofs throughout this section that may be of interest later by means of their divergence from the ones with which I am familiar.
    \item \textbf{Least-upper-bound property}: The property pertaining to a set $S$ that if $E\subset S$, $E\neq\emptyset$, and $E$ is bounded above, then $\sup E\in S$.
    \begin{itemize}
        \item For example, $\Q$ does not have the least-upper-bound property.
        \item The \textbf{greatest-lower-bound property} is analogously defined.
    \end{itemize}
    \item Theorem: Suppose $S$ is an ordered set with the least-upper-bound property, $B\subset S$ is nonempty, and $B$ is bounded below. Let $L$ be the set of all lower bounds of $B$. Then $\alpha=\sup L$ exists in $S$, and $\alpha=\inf B$. In particular, $\inf B$ exists in $S$.
    \begin{itemize}
        \item Essentially, this theorem states that any set that satisfies the least-upper-bound property satisfies the greatest lower bound property.
    \end{itemize}
    \item \textbf{Existence theorem}: There exists an ordered field $\R$ which has the least-upper-bound property. Moreover, $\R$ contains $\Q$ as a subfield.
    \begin{itemize}
        \item The second statement implies that the operations of addition and multiplication on $\R$, when applied to $\Q$, coincide with the operations of addition and multiplication on $\Q$.
    \end{itemize}
    \item \textbf{Archimedean property} (of $\R$): If $x\in\R$, $y\in\R$, and $x>0$, then there is a positive integer $n$ such that $nx>y$.
    \item \textcite{bib:Rudin} proves several theorems about the real numbers from the least-upper-bound property as opposed to the traditional construction of the real numbers.
    \item Introduces the decimal system.
    \item \textbf{Finite real number system}: That which has been defined thus far.
    \item \textbf{Extended real number system}: The set $\R\cup\{+\infty,-\infty\}$ where $+\infty,-\infty$ obey the expected properties (supremum [resp. infimum] of every set, $x+\infty=\infty$, etc.).
    \item Defines the complex field axiomatically with complex numbers in the form $(a,b)$ for $a,b\in\R$.
    \begin{itemize}
        \item Notes that the real numbers form a subfield of the complex field.
        \item Defines $i=(0,1)$, proves $i^2=-1$, proves $a+bi=(a,b)$.
    \end{itemize}
    \item \textbf{Schwarz inequality}: If $a_1,\dots,a_n$ and $b_1,\dots,b_n$ are complex numbers, then
    \begin{equation*}
        \left| \sum_{j=1}^na_j\bar{b}_j \right|^2 \leq \sum_{j=1}^n|a_j|^2\sum_{j=1}^n|b_j|^2
    \end{equation*}
    \item \textbf{Euclidean $\bm{k}$-space}: The vector space $\R^k$ over the real field.
\end{itemize}




\end{document}