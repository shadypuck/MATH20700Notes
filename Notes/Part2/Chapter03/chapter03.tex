\documentclass[../../notes.tex]{subfiles}

\pagestyle{main}
\renewcommand{\chaptermark}[1]{\markboth{\chaptername\ \thechapter\ (#1)}{}}
\setcounter{chapter}{2}

\begin{document}




\chapter{Numerical Sequences and Series}
\section{Notes}
\begin{itemize}
    \item \marginnote{11/8:}Any bounded sequence in $\R^k$ has a convergent subsequence.
    \item \marginnote{11/10:}Read and understand the section about Cauchy sequences converging and the sup/inf.
\end{itemize}



\section{Chapter 3: Numerical Sequences and Series}
\emph{From \textcite{bib:Rudin}.}
\begin{itemize}
    \item \marginnote{11/7:}Convergence of sequences is relative.
    \begin{itemize}
        \item For example, the sequence $1/n$ for $n=1,2,\dots$ converges in $\R$, but not in $(0,\infty)$.
    \end{itemize}
    \item \textbf{Range} (of $\{p_n\}$): The set of all points $p_n$.
    \begin{itemize}
        \item This definition squares nicely with the formal definition of a sequence as a function $p$ defined on $\N$.
    \end{itemize}
    \item Theorem 3.6a: If $\{p_n\}$ is a sequence in a compact metric space $X$, then some subsequence of $\{p_n\}$ converges to a point of $X$.
    \item Theorem 3.7: The subsequential limits of a sequence $\{p_n\}$ in a metric space $X$ form a closed subset of $X$.
    \item \textbf{Diameter} (of $E$): The supremum of the set
    \begin{equation*}
        S = \{d(p,q):p,q\in E\}
    \end{equation*}
    where $E$ is a nonempty subset of a metric space $X$. \emph{Denoted by} $\bm{\diam E}$.
    \item Theorem 3.10:
    \begin{enumerate}[label={(\alph*)}]
        \item If $\bar{E}$ is the closure of a set $E$ in a metric space $X$, then
        \begin{equation*}
            \diam\bar{E} = \diam E
        \end{equation*}
        \item If $K_n$ is a sequence of compact sets in $X$ such that $K_n\supset K_{n+1}$ ($n=1,2,3,\dots$) and if $\lim_{n\to\infty}\diam K_n=0$, then $\bigcap_1^\infty K_n$ consists of exactly one point.
    \end{enumerate}
    \item \textbf{Complete} (metric space): A metric space in which every Cauchy sequence converges.
    \item All compact metric spaces and all Euclidean spaces are complete.
    \begin{itemize}
        \item The metric space $(\Q,|x-y|)$ is not complete.
    \end{itemize}
    \item \textbf{Monotonically increasing} (sequence $\{s_n\}$): A sequence $\{s_n\}$ of real numbers such that $s_n\leq s_{n+1}$ for each $n\in\N$.
    \item \textbf{Monotonically decreasing} (sequence $\{s_n\}$): A sequence $\{s_n\}$ of real numbers such that $s_n\geq s_{n+1}$ for each $n\in\N$.
    \item \textbf{Monotonic sequences}: The class of all sequences that are either monotonically increasing or monotonically decreasing.
    \item \textbf{Upper limit} (of $\{s_n\}$): The supremum of the set $E$ of all subsequential limits of $\{s_n\}$. \emph{Denoted by} $\bm{s^*}$, $\bm{\limsup_{n\to\infty}s_n}$.
    \item \textbf{Lower limit} (of $\{s_n\}$): The infimum of the set $E$ of all subsequential limits of $\{s_n\}$. \emph{Denoted by} $\bm{s_*}$, $\bm{\liminf_{n\to\infty}s_n}$.
    \item Theorem 3.17: Let $\{s_n\}$ be a sequence of real numbers. Then $s^*$ has (and is the only number to have both of) the following two properties.
    \begin{enumerate}[label={(\alph*)}]
        \item $s^*\in E$.
        \item If $x>s^*$, then there is an integer $N$ such that $n\geq N$ implies $s_n<x$.
    \end{enumerate}
    An analogous result holds for $s_*$.
    \item \marginnote{11/8:}Series are defined in terms of sequences. Moreover, sequences can be defined in terms of series: Let $a_1=s_1$, $a_n=s_n-s_{n-1}$ ($n\in\N+1$). Thus, every theorem about sequences can be stated in terms of series and vice versa, but it is nevertheless useful to consider both concepts \parencite[59]{bib:Rudin}.
    \item Theorem 3.27: Suppose $\{a_n\}$ is a monotonically decreasing sequence of nonnegative terms. Then the series $\sum_{n=1}^\infty a_n$ converges if and only if the series
    \begin{equation*}
        \sum_{k=0}^\infty 2^ka_{2^k} = a_1+2a_2+4a_4+8a_8+\cdots
    \end{equation*}
    converges.
    \item Theorem 3.29: If $p>1$,
    \begin{equation*}
        \sum_{n=2}^\infty\frac{1}{n(\log n)^p}
    \end{equation*}
    converges; if $p\leq 1$, the series diverges.
    \begin{itemize}
        \item Note that $\log n=\ln n$.
        \item Note that we sum from $n=2$ since $\log 1=0$.
    \end{itemize}
    \item \textbf{e}: The number
    \begin{equation*}
        \e = \sum_{n=0}^\infty\frac{1}{n!}
    \end{equation*}
    \item Theorem 3.31: $\lim_{n\to\infty}(1+1/n)^n=\e$.
    \item Theorem 3.32: $\e$ is irrational.
    \item Theorem 3.39: Given the power series $\sum c_nz^n$, put
    \begin{align*}
        \alpha &= \limsup_{n\to\infty}\sqrt[n]{|c_n|}&
        R &= \frac{1}{\alpha}
    \end{align*}
    (If $\alpha=0$, let $R=+\infty$; if $\alpha=+\infty$, let $R=0$.) Then $\sum c_nz^n$ converges if $|z|<R$ and diverges if $|z|>R$.
    \item \textbf{Radius of convergence} (of a power series): The number $R$ defined by Theorem 3.39.
    \item Theorem 3.41 (partial summation formula): Given two sequence $\{a_n\},\{b_n\}$, put
    \begin{equation*}
        A_n =
        \begin{cases}
            \sum_{k=0}^na_k & n\geq 0\\
            0 & n=-1
        \end{cases}
    \end{equation*}
    Then if $0\leq p\leq q$, we have
    \begin{equation*}
        \sum_{n=p}^qa_nb_n = \sum_{n=p}^{q-1}A_n(b_n-b_{n+1})+A_qb_q-A_{p-1}b_p
    \end{equation*}
    \item \textbf{Product} (of $\sum a_n,\sum b_n$): The series $\sum c_n$ defined by
    \begin{equation*}
        c_n = \sum_{k=0}^na_kb_{n-k}
    \end{equation*}
    for each $n=0,1,2,\dots$.
    \begin{itemize}
        \item We motivate this definition by noting that if $\sum c_n$ is the product of $\sum a_n,\sum b_n$, then
        \begin{equation*}
            \sum_{n=0}^\infty a_nz^n\cdot\sum_{n=0}^\infty b_nz^n = \sum_{n=0}^\infty c_nz^n
        \end{equation*}
        \item Setting $z=1$ then yields the given definition.
    \end{itemize}
    \item The product of two convergent series may diverge. However\dots
    \item Theorem 3.50 (by Mertens): Suppose (a) $\sum_{n=0}^\infty a_n$ converges absolutely, (b) $\sum_{n=0}^\infty a_n=A$, (c) $\sum_{n=0}^\infty b_n=B$, and (d) $\sum_{n=0}^\infty c_n$ is the product of $\sum_{n=0}^\infty a_n$ and $\sum_{n=0}^\infty b_n$. Then
    \begin{equation*}
        \sum_{n=0}^\infty c_n = AB
    \end{equation*}
    \item Theorem 3.51 (by Abel): If $\sum a_n,\sum b_n,\sum c_n$ converge to $A,B,C$, respectively, and $\sum c_n$ is the product of $\sum a_n,\sum b_n$, then $C=AB$.
    \item \textbf{Rearrangement} (of $\sum a_n$): A series $\sum a_n'$ defined by $a_n'=a_{k_n}$ for each $n\in\N$, where $\{k_n\}$ is a sequence in which every positive integer appears once and only once (that is, $\{k_n\}$ is a 1-1 function from $\N$ onto $\N$).
    \item Theorem 3.54: Let $\sum a_n$ be a series of real number which converges, but not absolutely. Suppose $-\infty\leq\alpha\leq\beta\leq\infty$. Then there exists a rearrangement $\sum a_n'$ with partial sums $s_n'$ such that
    \begin{align*}
        \liminf_{n\to\infty}s_n' &= \alpha&
        \limsup_{n\to\infty}s_n' &= \beta
    \end{align*}
    \item Theorem 3.55: If $\sum a_n$ is a series of complex numbers which converges absolutely, then every rearrangement of $\sum a_n$ converges, and they all converge to the same sum.
\end{itemize}




\end{document}