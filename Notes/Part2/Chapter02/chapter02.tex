\documentclass[../../notes.tex]{subfiles}

\pagestyle{main}
\renewcommand{\chaptermark}[1]{\markboth{\chaptername\ \thechapter\ (#1)}{}}
\stepcounter{chapter}

\begin{document}




\chapter{Basic Topology}
\section{Notes}
\begin{itemize}
    \item \marginnote{11/1:}Equivalence relationships are denoted $A\sim B$.
    \begin{itemize}
        \item These are\dots
        \begin{itemize}
            \item Reflexive ($A\sim A$).
            \item Symmetric ($A\sim B\Longleftrightarrow B\sim A$).
            \item Transitive ($A\sim B\ \&\ B\sim C\Longrightarrow A\sim C$).
        \end{itemize}
        \item Equivalence relations give rise to equivalence classes.
    \end{itemize}
    \item \textbf{Countable} (set $A$): A set $A$ such that $A\sim\N$, in the sense that there exists a one-to-one and onto map from $\N\to A$.
    \begin{itemize}
        \item Alternatively, $A$ can be written in the form $A=\{f(n):n\in\N\}$.
    \end{itemize}
    \item \textbf{Finite countable} vs. \textbf{infinite countable} (see \textcite{bib:Rudin}).
    \item $\N$ denotes the natural numbers.
    \item $\N_0$ denotes the natural numbers including 0.
    \item $\Z$ denotes the integers.
    \item We know that $\N\sim\Z$: Let $f:\N\to\Z$ be defined by
    \begin{equation*}
        f(n) =
        \begin{cases}
            \frac{n}{2} & n\text{ even}\\
            \frac{n-1}{2} & n\text{ odd}
        \end{cases}
    \end{equation*}
    \item More facts.
    \begin{enumerate}
        \item Every subset of a countable set is countable.
        \item Unions of countable sets are countable.
        \begin{itemize}
            \item If the sets $E_n$ for some finite list of numbers are countable, then $\bigcup_nE_n$ is countable.
            \item Soug goes over the diagonalization method of counting.
        \end{itemize}
        \item $n$-fold Cartesian products of countable sets are countable (we induct on $n$).
        \begin{itemize}
            \item If $A$ is countable and $B$ is countable, then $A\times B$ is countable.
            \item If $A$ is finite and to each $\alpha\in A$ we assign a countable set $E_\alpha$, $\otimes_{\alpha\in A}E_\alpha$ is countable.
        \end{itemize}
    \end{enumerate}
    \item \textbf{Metric space}: A space $X$ along with a matrix $d:X\times X\to[0,\infty)$ such that
    \begin{itemize}
        \item $d(x,y)>0$ iff $x\neq y$, and $d(x,x)=0$ iff $x=0$.
        \item $d(x,y)=d(y,x)$.
        \item $d(x,y)\leq d(x,z)+d(z,y)$.
    \end{itemize}
    \item Example ($\R^n$):
    \begin{itemize}
        \item We may define $d$ by
        \begin{equation*}
            d(x,y) = \sqrt{\sum(x_i-y_i)^2}
        \end{equation*}
        \item We can also define the $p$-metrics (recall normed spaces) with $p$ where 2 is.
    \end{itemize}
    \item Example ($X_p=\{f:Y\to\R:1\leq p<\infty,\int_Y|f|^p\dd{y}<\infty\}$):
    \begin{itemize}
        \item This is $\ell_p$.
        \item Define
        \begin{equation*}
            \norm{f-g}_p = \left[ \int_Y|f-g|^p\dd{y} \right]^{1/p}
        \end{equation*}
    \end{itemize}
    \item Convergence: $x_n\to x\Longleftrightarrow d(x_n,x)\to 0$.
    \item \textbf{Neighborhood}: The set of all points a distance less than $r$ away from $p$. \emph{Denoted by} $\bm{N_r(p)}$. \emph{Given by}
    \begin{equation*}
        N_r(p) = \{q\in X:d(p,q)<r\}
    \end{equation*}
    \item \textbf{Limit point} (of $E$): A point $p$ such that every neighborhood of $p$ intersects $E$ at a point other than $p$. \emph{Also known as} \textbf{accumulation point}.
    \begin{itemize}
        \item Symbolically,
        \begin{equation*}
            N_r(p)\cap(E\setminus\{p\}) \neq \emptyset
        \end{equation*}
        for all $r>0$.
    \end{itemize}
    \item \textbf{Isolated point} (of $E$): A point $p$ such that $p\in E$ and $p$ is not a limit point of $E$.
    \item \textbf{Closed} (set $E$): A set $E$ that contains all of its limit points.
    \item \textbf{Interior} (point $p$): A point $p$ such that there exists $N_r(p)\subset E$.
    \item \textbf{Open} (set $E$): A set $E$, all points of which are interior points.
    \item \textbf{Perfect} (set $E$): A set $E$ that is closed and every point of $E$ is a limit point of $E$.
    \item \textbf{Bounded} (set $E$): There exists a number $M$ and a $y\in X$ such that $E\subset\{p:d(p,y)\leq M\}$.
    \item \textbf{Dense} (set $E$ in $X$): A set $E$ such that every point of $X$ is a limit point of $E$ or a point of $E$, itself.
    \item \marginnote{11/3:}Every neighborhood is an open set.
    \item If $p$ is a limit point of $E$, every neighborhood of $p$ contains infinitely many points of $E$.
    \begin{itemize}
        \item Thus, a finite set cannot have a limit point.
        \item Prove by contradiction: Suppose there is a neighborhood that contains only finitely many points of $E$. Then the neighborhood with radius smaller than the distance to the closest point does not contain any points of $E$, a contradiction.
    \end{itemize}
    \item $E$ is open iff $E^C$\footnote{The complement of $E$.} is closed.
    \begin{itemize}
        \item Assume $E^C$ closed. If $p\in E$, then $p$ is not a limit point of $E^C$. It follows that there exists a neighborhood of $p$ that is entirely contained within $E$, so $p$ is interior, as desired.
        \item Suppose $E$ is open. Let $p$ be any limit point of $E^C$. Then $p\in E^C$.
    \end{itemize}
    \item $F$ is closed iff $F^C$ is open.
    \item If $(G_\alpha)_{\alpha\in A}$ is a family of open sets in $X$, then the union is open.
    \begin{itemize}
        \item Let $p\in\bigcup_{\alpha\in A}G_\alpha$. Then $p\in G_\alpha$ for some $\alpha\in A$. It follows that $p$ is an interior point of $G_\alpha$, so thus an interior point of the union of $G_\alpha$ with everything else.
    \end{itemize}
    \item Finite intersections of open sets are open.
    \begin{itemize}
        \item In the infinite case $\bigcap_{n\in\N}(-1/n,1/n)=\{0\}$, an intersection of infinitely many open sets is closed.
        \item However, in the finite case, just consider the neighborhood with the smallest radius and take this one.
    \end{itemize}
    \item The intersection of closed sets is closed.
    \item The union of finitely many closed sets is closed.
    \begin{itemize}
        \item These follow from the previous two by De Morgan's rule.
    \end{itemize}
    \item Let $\bar{E}=E\cup E'$ where $E'$ is the set of limit points of $E$.
    \item Let $X$ be a metric space and $E\subset X$. Then
    \begin{enumerate}
        \item $\bar{E}$ is closed.
        \begin{itemize}
            \item WTS: $\bar{E}^C$ is open. Let $p\in\bar{E}^C$. Then $p$ is neither in $E$ nor is it a limit point of $E$. Thus, there exists a neighborhood of $\bar{E}^C$ containing entirely points of $\bar{E}^C$. Therefore, $\bar{E}^C$ is open, so $\bar{E}$ is closed.
        \end{itemize}
        \item $E=\bar{E}$ iff $E$ is closed.
        \begin{itemize}
            \item Think $p\in\bigcap G_\alpha$?
        \end{itemize}
        \item $\bar{E}\subset F$ for any closed $F\supset E$.
        \begin{itemize}
            \item If $E\subset F$, then any limit point of $E$ will be a limit point of $F$. Thus, $E'\subset F'$. Then $\bar{E}=E\cup E'\subset F\cup F'=\bar{F}=F$ where the last equality holds because $F$ is closed.
        \end{itemize}
    \end{enumerate}
    \item Types of sets.
    \begin{table}[h!]
        \centering
        \small
        \renewcommand{\arraystretch}{1.4}
        \begin{tabular}{l|c|c|c|c}
             & Closed & Open & Perfect & Bounded\\ \hline
            $\{z\in\Q:|z|<1\}$ & N & Y & N & Y\\ \hline
            $\{z\in\Q:|z|\leq 1\}$ & Y & N & Y & Y\\ \hline
            Nonempty finite set & Y & N & N & Y\\ \hline
            $\Z$ & Y & N & N & N\\ \hline
            $\{1/n:n\in\N\}$ & N & N & N & Y\\ \hline
            $\R^2$ & Y & Y & Y & N\\ \hline
            $(a,b)$ & N & ? & N & Y\\
        \end{tabular}
        \caption{Types of sets.}
        \label{tab:typesSets}
    \end{table}
    \item \textbf{Relatively open} (set $E$ to $Y$): A set $E\subset Y\subset X$ such that if $p\in E$, then there exists a $Y$-neighborhood of $E$ contained in $E$.
    \item Let $N_r^X(p)=\{y\in X:d(y,p)<r\}$ be a neighborhood of $p$ in $X$, and let $N_r^Y(p)=\{y\in Y:d(y,p)<r\}$ be a neighborhood of $p$ in $Y$. Then $N_r^Y(p)=N_r^X(p)\cap Y$.
    \item $E$ is open relative to $Y$ iff $E=G\cap Y$ where $G$ is open relative to $X$.
    \item Introduces the supremum.
    \item If $E\subset\R$, $E\neq\emptyset$, and $E$ is bounded above, $\sup E<\infty$.
    \item Let $y=\sup E$. Then $y\in\bar{E}$.
    \item There exists a sequence $a_n\in A$ such that $a_n\to x=\sup A$.
    \item $A$ is compact iff any open cover of the set has a finite subcover.
    \item Study and \emph{know} all of these proofs.
\end{itemize}




\end{document}