\documentclass[../../notes.tex]{subfiles}

\pagestyle{main}
\renewcommand{\chaptermark}[1]{\markboth{\chaptername\ \thechapter\ (#1)}{}}
\setcounter{chapter}{4}

\begin{document}




\chapter{Differentiation}
\section{Chapter 5: Differentiation}
\emph{From \textcite{bib:Rudin}.}
\begin{itemize}
    \item \marginnote{12/5:}Let $f$ be a real-valued function defined on $[a,b]$.
    \item \textbf{Derivative} (of $f$ at $x$): The limit $\lim_{t\to x}\phi(t)$, provided that said limit exists, where $\phi:(a,b)\setminus\{x\}\to\R$ is defined by
    \begin{equation*}
        \phi(t) = \frac{f(t)-f(x)}{t-x}
    \end{equation*}
    \emph{Denoted by} $\bm{f'(x)}$.
    \item \textbf{Derivative} (of $f$): The real function defined on $X$ that evaluates to $f'(x)$ everywhere on its domain, where
    \begin{equation*}
        X = \{x\in[a,b]:f'(x)\text{ exists}\}
    \end{equation*}
    \emph{Denoted by} $\bm{f'}$.
    \item Theorem 5.2: Differentiability at $x$ implies continuity at $x$.
    \begin{itemize}
        \item The converse is not true.
    \end{itemize}
    \item Theorem 5.3: Sum, product, and quotient rules of derivatives.
    \item Theorem 5.4 (Chain Rule): Suppose $f$ is continuous on $[a,b]$, $f'(x)$ exists at some point $x\in[a,b]$, $g$ is defined on an interval $I$ which contains the range of $f$, and $g$ is differentiable at the point $f(x)$. If $h(t)=g(f(t))$ for all $t\in[a,b]$, then $h$ is differentiable at $x$ and
    \begin{equation*}
        h'(x) = g'(f(x))f'(x)
    \end{equation*}
    \begin{proof}
        Let $y=f(x)$. Since $f$ is differentiable at $x$ and $g$ is differentiable at $f(x)$, we have that
        \begin{align*}
            \frac{f(t)-f(x)}{t-x} &= f'(x)+u(t)&
                \frac{g(s)-g(y)}{s-y} &= g'(y)+v(s)\\
            f(t)-f(x) &= (t-x)[f'(x)+u(t)]&
                g(s)-g(y) &= (s-y)[g'(y)+v(s)]
        \end{align*}
        where $t\in[a,b]$, $s\in I$, $u(t)\to 0$ as $t\to x$, and $v(s)\to 0$ as $s\to y$. Let $s=f(t)$. Then
        \begin{align*}
            h(t)-h(x) &= g(f(t))-g(f(x))\\
            &= [f(t)-f(x)]\cdot[g'(f(x))+v(s)]\\
            &= (t-x)\cdot[f'(x)+u(t)]\cdot[g'(f(x))+v(s)]\\
            \frac{h(t)-h(x)}{t-x} &= [f'(x)+u(t)]\cdot[g'(f(x))+v(s)]
        \end{align*}
        Thus, since as $t\to x$, $s=f(t)\to f(x)=y$ by the continuity of $f$, we have that
        \begin{align*}
            h'(x) &= \lim_{t\to x}\frac{h(t)-h(x)}{t-x}\\
            &= \lim_{t\to x}[f'(x)+u(t)]\cdot[g'(f(x))+v(s)]\\
            &= [f'(x)+0]\cdot[g'(f(x))+0]\\
            &= g'(f(x))f'(x)
        \end{align*}
        as desired.
    \end{proof}
    \item \textbf{Local maximum} (of $f:X\to\R$): A point $p\in X$ for which there exists a $\delta>0$ such that $f(q)\leq f(p)$ for all $q\in X$ with $d(p,q)<\delta$.
    \item Theorem 5.8: $f(x)$ a local maximum and $f'$ exists implies $f'(x)=0$.
    \item Theorem 5.9 (Generalized or Cauchy Mean Value Theorem): $f,g$ continuous on $[a,b]$, differentiable on $(a,b)$ imply there exists $x\in(a,b)$ such that
    \begin{equation*}
        [f(b)-f(a)]g'(x) = [g(b)-g(a)]f'(x)
    \end{equation*}
    \item Theorem 5.10 (Mean Value Theorem): $f$ continuous on $[a,b]$, differentiable on $(a,b)$ implies there exists $x\in(a,b)$ such that
    \begin{equation*}
        f(b)-f(a) = (b-a)f'(x)
    \end{equation*}
    \begin{proof}
        Take $g(x)=x$ in Theorem 5.9.
    \end{proof}
    \item Theorem 5.11: Suppose $f$ is differentiable in $(a,b)$.
    \begin{enumerate}[label={(\alph*)}]
        \item If $f'(x)\geq 0$ for all $x\in(a,b)$, then $f$ is monotonically increasing.
        \item If $f'(x)=0$ for all $x\in(a,b)$, then $f$ is constant.
        \item If $f'(x)\leq 0$ for all $x\in(a,b)$, then $f$ is monotonically decreasing.
    \end{enumerate}
    \item Theorem 5.12: $f$ differentiable on $[a,b]$ and $f'(a)<\lambda<f'(b)$ implies there exists $x\in(a,b)$ such that $f'(x)=\lambda$.
    \item Corollary: $f$ differentiable on $[a,b]$ implies $f'$ has no simple discontinuities on $[a,b]$.
    \begin{itemize}
        \item But it may have discontinuities of the second kind.
    \end{itemize}
    \item Theorem 5.13 (L'H\^{o}pital's Rule): $f,g$ differentiable on $(a,b)$, $g'(x)\neq 0$ for all $x\in(a,b)$, $f'(x)/g'(x)\to A$, and $f(x)\to 0$ and $g(x)\to 0$ as $x\to a$ or $g(x)\to +\infty$ as $x\to a$ implies $f(x)/g(x)\to A$ as $x\to a$, where $-\infty\leq a<b\leq +\infty$.
    \item \textbf{\emph{n}\textsuperscript{th} derivative} (of $f$ at $x$): The derivative of the $(n-1)^\text{th}$ derivative of $f$ at $x$, if it exists. \emph{Denoted by} $\bm{f^{(n)}(x)}$.
    \begin{itemize}
        \item $f^{(n)}(x)$ exists iff $f^{(n-1)}$ exists in some $N_r(x)$ and $f^{(n-1)'}(x)$ exists.
        \item We customarily denote the first few higher order derivatives with repeated primes, e.g., $f''(x)$ is the second derivative of $f$.
    \end{itemize}
    \item Theorem 5.15 (Taylor's Theorem): $f$ defined on $[a,b]$, $n\in\N$, $f^{(n-1)}$ continuous on $[a,b]$, $f^{(n)}(t)$ defined on $(a,b)$, $\alpha,\beta\in[a,b]$ such that $\alpha\neq\beta$, and
    \begin{equation*}
        P(t) = \sum_{k=0}^{n-1}\frac{f^{(k)}(\alpha)}{k!}(t-\alpha)^k
    \end{equation*}
    implies there exists $x\in(\alpha,\beta)$ such that
    \begin{equation*}
        f(\beta) = P(\beta)+\frac{f^{(n)}(x)}{n!}(\beta-\alpha)^n
    \end{equation*}
    \begin{itemize}
        \item For $n=1$, this is the mean value theorem.
        \item "In general, the theorem shows that $f$ can be approximated by a polynomial of degree $n-1$ and that the last equation above allows us to estimate the error, if we know bounds on $|f^{(n)}(x)|$" \parencite[111]{bib:Rudin}.
    \end{itemize}
    \item \textbf{Derivative} (of $\fb$ at $x$): The point $\fb'(x)\in\R^k$, if it exists, such that
    \begin{equation*}
        \lim_{t\to x}\norm{\frac{\fb(t)-\fb(x)}{t-x}-\fb'(x)} = 0
    \end{equation*}
    \item Theorems 5.2-5.3 remain valid for vector-valued functions.
    \item If $\fb=(f_1,\dots,f_k)$, then $\fb'(x)$ exists iff $f_i'(x)$ ($i=1,\dots,k$) exists and
    \begin{equation*}
        \fb' = (f_1',\dots,f_k')
    \end{equation*}
    \item Theorem 5.19: $\fb:[a,b]\to\R^k$ continuous and $\fb$ differentiable on $(a,b)$ implies there exists $x\in(a,b)$ such that
    \begin{equation*}
        \norm{\fb(b)-\fb(a)} \leq (b-a)\norm{\fb'(x)}
    \end{equation*}
\end{itemize}




\end{document}