\documentclass[../psets.tex]{subfiles}

\pagestyle{main}
\renewcommand{\leftmark}{Problem Set \thesection}
\setcounter{section}{8}

\begin{document}




\section{Sequences and Series of Functions}
\emph{From \textcite{bib:Rudin}.}
\subsection*{Chapter 7}
\begin{enumerate}[label={\textbf{\arabic*.}}]
    \item \marginnote{12/10:}Prove that every uniformly convergent sequence of bounded functions is uniformly bounded.
    \stepcounter{enumi}
    \item Construct sequences $\{f_n\},\{g_n\}$ which converge uniformly on some set $E$, but such that $\{f_ng_n\}$ does not converge uniformly on $E$ (of course, $\{f_ng_n\}$ must converge on $E$).
    \item Consider
    \begin{equation*}
        f(x) = \sum_{n=1}^\infty\frac{1}{1+n^2x}
    \end{equation*}
    For what values of $x$ does the series converge absolutely? On what intervals does it converge uniformly? On what intervals does it fail to converge uniformly? Is $f$ continuous wherever the series converges? Is $f$ bounded?
    \item Let
    \begin{equation*}
        f_n(x) =
        \begin{cases}
            0 & x<\frac{1}{n+1}\\
            \sin^2\frac{\pi}{x} & \frac{1}{n+1}\leq x\leq\frac{1}{n}\\
            0 & \frac{1}{n}<x
        \end{cases}
    \end{equation*}
    Show that $\{f_n\}$ converges to a continuous function, but not uniformly. Use the series $\sum f_n$ to show that absolute convergence, even for all $x$, does not imply uniform convergence.
    \item Prove that the series
    \begin{equation*}
        \sum_{n=1}^\infty(-1)^n\frac{x^2+n}{n^2}
    \end{equation*}
    converges uniformly in every bounded interval, but does not converge absolutely for any value of $x$.
    \item For $n=1,2,3,\dots$ and $x$ real, put
    \begin{equation*}
        f_n(x) = \frac{x}{1+nx^2}
    \end{equation*}
    Show that $\{f_n\}$ converges uniformly to a function $f$ and that the equation
    \begin{equation*}
        f'(x) = \lim_{n\to\infty}f_n'(x)
    \end{equation*}
    is correct if $x\neq 0$ but false if $x=0$.
    \item If
    \begin{equation*}
        I(x) =
        \begin{cases}
            0 & x\leq 0\\
            1 & x>0
        \end{cases}
    \end{equation*}
    if $\{x_n\}$ is a sequence of distinct points of $(a,b)$, and if $\sum|c_n|$ converges, prove that the series
    \begin{equation*}
        f(x) = \sum_{n=1}^\infty c_nI(x-x_n)
    \end{equation*}
    converges uniformly on $[a,b]$, and that $f$ is continuous for every $x\neq x_n$.
    \item Let $\{f_n\}$ be a sequence of continuous functions which converges uniformly to a function $f$ on a set $E$. Prove that
    \begin{equation*}
        \lim_{n\to\infty}f_n(x_n) = f(x)
    \end{equation*}
    for every sequence of points $x_n\in E$ such that $x_n\to x$ and $x\in E$. Is the converse of this true?
    \item Letting $(x)$ denote the fractional part of the real number $x$ (see Exercise 4.16 for the definition), consider the function
    \begin{equation*}
        f(x) = \sum_{n=1}^\infty\frac{(nx)}{n^2}
    \end{equation*}
    defined on $\R$. Find all discontinuities of $f$, and show that they form a countable dense set. Show that $f$ is nevertheless Riemann-integrable on every bounded interval.
    \item Suppose $\{f_n\},\{g_n\}$ are defined on $E$ and that
    \begin{enumerate}
        \item $\sum f_n$ has uniformly bounded partial sums;
        \item $g_n\to 0$ uniformly on $E$;
        \item $g_1(x)\geq g_2(x)\geq g_3(x)\geq\cdots$ for every $x\in E$.
    \end{enumerate}
    Prove that $\sum f_ng_n$ converges uniformly on $E$. (Hint: Compare with Theorem 3.42.)
    \item Suppose $g$ and $f_n$ ($n\in\N$) are defined on $(0,\infty)$, are Riemann-integrable on $[t,T]$ whenever $0<t<T<\infty$, $|f_n|\leq g$, $f_n\to f$ uniformly on every compact subset of $(0,\infty)$, and
    \begin{equation*}
        \int_0^\infty g(x)\dd{x} < \infty
    \end{equation*}
    Prove that
    \begin{equation*}
        \lim_{n\to\infty}\int_0^\infty f_n(x)\dd{x} = \int_0^\infty f(x)\dd{x}
    \end{equation*}
    (See Exercises 6.7-6.8 for the relevant definitions.) This is a rather weak form of Lebesgue's dominated convergence theorem (Theorem 11.32). Even in the context of the Riemann integral, uniform convergence can be replaced by pointwise convergence if it is assumed that $f\in\mathscr{R}$. (See \textcite{bib:Cunningham} and \textcite{bib:Kestelman}.)
    \item Assume that $\{f_n\}$ is a sequence of monotonically increasing functions on $\R^1$ with $0\leq f_n(x)\leq 1$ for all $x$ and all $n$.
    \begin{enumerate}
        \item Prove that there is a function $f$ and a sequence $\{n_k\}$ such that
        \begin{equation*}
            f(x) = \lim_{k\to\infty}f_{n_k}(x)
        \end{equation*}
        for every $x\in\R^1$. The existence of such a pointwise convergent subsequence is usually called \textbf{Helly's selection theorem}. (Hint: (i) Some subsequence $\{f_{n_i}\}$ converges at all rational points $r$, say, to $f(r)$. (ii) Define $f(x)$ for any $x\in\R^1$ to be $\sup f(r)$, the $\sup$ being taken over all $r\leq x$. (iii) Show that $f_{n_i}(x)\to f(x)$ at every $x$ at which $f$ is continuous. [This is where monotonicity is strongly used.] (iv) A subsequence of $\{f_{n_i}\}$ converges at every point of discontinuity of $f$ since there are at most countably many such points.)
        \item If, moreover, $f$ is continuous, prove that $f_{n_k}\to f$ uniformly on compact sets. (Hint: Modify your proof of (iii) appropriately.)
    \end{enumerate}
    \item Let $f$ be a continuous real function on $\R^1$ with the following properties: $0\leq f(t)\leq 1$, $f(t+2)=f(t)$ for every $t$, and
    \begin{equation*}
        f(t) =
        \begin{cases}
            0 & 0\leq t\leq\frac{1}{3}\\
            1 & \frac{2}{3}\leq t\leq 1
        \end{cases}
    \end{equation*}
    Put $\Phi(t)=(x(t),y(t))$, where
    \begin{align*}
        x(t) &= \sum_{n=1}^\infty 2^{-n}f(3^{2n-1}t)&
        y(t) &= \sum_{n=1}^\infty 2^{-n}f(3^{2n}t)
    \end{align*}
    Prove that $\Phi$ is continuous and that $\Phi$ maps $I=[0,1]$ onto the unit square $I^2\subset\R^2$. In fact, show that $\Phi$ maps the Cantor set onto $I^2$. (Hint: Each $(x_0,y_0)\in I^2$ has the form
    \begin{align*}
        x_0 &= \sum_{n=1}^\infty 2^{-n}a_{2n-1}&
        y_0 &= \sum_{n=1}^\infty 2^{-n}a_{2n}
    \end{align*}
    where each $a_i$ is 0 or 1. If
    \begin{equation*}
        t_0 = \sum_{i=1}^\infty 3^{-i-1}(2a_i)
    \end{equation*}
    show that $f(3^kt_0)=a_k$, and hence that $x(t_0)=x_0$, $y_0(t_0)=y_0$.) This simple example of a so-called \textbf{space-filling curve} is due to \textcite{bib:Schoenberg}.
    \item Suppose $f$ is a real continuous function on $\R^1$, $f_n(t)=f(nt)$ for $n\in\N$, and $\{f_n\}$ is equicontinuous on $[0,1]$. What conclusion can you draw about $f$?
    \item Suppose $\{f_n\}$ is an equicontinuous sequence of functions on a compact set $K$, and $\{f_n\}$ converges pointwise on $K$. Prove that $\{f_n\}$ converges uniformly on $K$.
    \stepcounter{enumi}
    \item Let $\{f_n\}$ be a uniformly bounded sequence of functions which are Riemann-integrable on $[a,b]$ and put
    \begin{equation*}
        F_n(x) = \int_a^xf_n(t)\dd{t}
    \end{equation*}
    for $x\in[a,b]$. Prove that there exists a subsequence $\{F_{n_k}\}$ which converges uniformly on $[a,b]$.
    \item Let $K$ be a compact metric space, let $S$ be a subset of $\mathscr{C}(K)$. Prove that $S$ is compact (with respect to the metric defined in Section 7.14) if and only if $S$ is uniformly closed, pointwise bounded, and equicontinuous. (If $S$ is not equicontinuous, then $S$ contains a sequence which has no equicontinuous subsequence, hence has no subsequence that converges uniformly on $K$.)
    \item If $f$ is continuous on $[0,1]$ and if
    \begin{equation*}
        \int_0^1f(x)x^n\dd{x} = 0
    \end{equation*}
    for $n=0,1,2,\dots$, prove that $f(x)=0$ on $[0,1]$. (Hint: The integral of the product of $f$ with any polynomial is zero. Use the Weierstrass theorem to show that $\int_0^1f^2(x)\dd{x}=0$.)
    \stepcounter{enumi}
    \item Assume $f\in\mathscr{R}(\alpha)$ on $[a,b]$, and prove that there are polynomials $P_n$ such that
    \begin{equation*}
        \lim_{n\to\infty}\int_a^b|f-P_n|^2\dd{\alpha} = 0
    \end{equation*}
    (Compare with Exercise 6.12.)
    \item Put $P_0=0$, and define for $n=0,1,2,\dots$
    \begin{equation*}
        P_{n+1}(x) = P_n(x)+\frac{x^2-P_n^2(x)}{2}
    \end{equation*}
    Prove that
    \begin{equation*}
        \lim_{n\to\infty}P_n(x) = |x|
    \end{equation*}
    uniformly on $[-1,1]$. This makes it possible to prove the Stone-Weierstrass theorem without first proving Theorem 7.26. (Hint: Use the identity
    \begin{equation*}
        |x|-P_{n+1}(x) = [|x|-P_n(x)]\left[ 1-\frac{|x|+P_n(x)}{2} \right]
    \end{equation*}
    to prove that $0\leq P_n(x)\leq P_{n+1}(x)\leq|x|$ if $|x|\leq 1$ and that
    \begin{equation*}
        |x|-P_n(x) \leq |x|\left( 1-\frac{|x|}{2} \right)^n < \frac{2}{n+1}
    \end{equation*}
    if $|x|<1$.)
    \stepcounter{enumi}
    \item Suppose $\phi$ is a continuous bounded real function in the strip defined by $0\leq x\leq 1$, $-\infty <y<\infty$. Prove that the initial-value problem
    \begin{align*}
        y' &= \phi(x,y)&
        y(0) &= c
    \end{align*}
    has a solution. Note that the hypotheses of this existence theorem are less stringent than those of the corresponding uniqueness theorem; see Exercise 5.27. (Hint: Fix $n$. For $i=0,\dots,n$, put $x_i=i/n$. Let $f_n$ be a continuous function on $[0,1]$ such that $f_n(0)=c$, let
    \begin{equation*}
        f_n'(t) = \phi(x_i,f_n(x_i))
    \end{equation*}
    if $x_i<t<x_{i+1}$, and put
    \begin{equation*}
        \Delta_n(t) = f_n'(t)-\phi(t,f_n(t))
    \end{equation*}
    except at points $x_i$, where $\Delta_n(t)=0$. Then
    \begin{equation*}
        f_n(x) = c+\int_0^x[\phi(t,f_n(t))+\Delta_n(t)]\dd{t}
    \end{equation*}
    Choose $M<\infty$ so that $|\phi|\leq M$. Verify the following assertions.
    \begin{enumerate}
        \item $|f_n'|\leq M$, $|\Delta_n|\leq 2M$, $\Delta_n\in\mathscr{R}$, and $|f_n|\leq|c|+M=M_1$ say, on $[0,1]$, for all $n$.
        \item $\{f_n\}$ is equicontinuous on $[0,1]$ since $|f_n'|\leq M$.
        \item Some $\{f_{n_k}\}$ converges to some $f$, uniformly on $[0,1]$.
        \item Since $\phi$ is uniformly continuous on the rectangle $0\leq x\leq 1$, $|y|\leq M_1$,
        \begin{equation*}
            \phi(t,f_{n_k}(t)) \to \phi(t,f(t))
        \end{equation*}
        uniformly on $[0,1]$.
        \item $\Delta_n(t)\to 0$ uniformly on $[0,1]$ since
        \begin{equation*}
            \Delta_n(t) = \phi(x_i,f_n(x_i))-\phi(t,f_n(t))
        \end{equation*}
        in $(x_i,x_{i+1})$.
        \item Hence
        \begin{equation*}
            f(x) = c+\int_0^x\phi(t,f(t))\dd{t}
        \end{equation*}
    \end{enumerate}
    This $f$ is a solution of the given problem.)
    \item Prove an analogous existence theorem for the initial-value problem
    \begin{align*}
        \y' &= \bm{\Phi}(\x,\y)&
        \y(0) &= \cb
    \end{align*}
    where now $\cb\in\R^k$, $\y\in\R^k$, and $\bm{\Phi}$ is a continuous bounded mapping of the part of $\R^{k+1}$ defined by $0\leq x\leq 1$, $\y\in\R^k$ into $\R^k$. Compare Exercise 5.28. (Hint: Use the vector-valued version of Theorem 7.25.)
\end{enumerate}




\end{document}