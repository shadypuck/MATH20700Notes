\documentclass[../psets.tex]{subfiles}

\pagestyle{main}
\renewcommand{\leftmark}{Problem Set \thesection}
\setcounter{section}{7}

\begin{document}




\section{Continuity}
\emph{From \textcite{bib:Rudin}.}
\subsection*{Chapter 4}
\begin{enumerate}[label={\textbf{\arabic*.}}]
    \item \marginnote{11/29:}Suppose $f$ is a real function defined on $\R^1$ which satisfies
    \begin{equation*}
        \lim_{h\to 0}[f(x+h)-f(x-h)] = 0
    \end{equation*}
    for every $x\in\R^1$. Does this imply that $f$ is continuous?
    \begin{proof}
        No.\par
        Consider the function $f:\R^1\to\R$ defined by
        \begin{equation*}
            f(x) =
            \begin{cases}
                1 & x=0\\
                0 & x\neq 0
            \end{cases}
        \end{equation*}
        We first show that $f$ satisfies the desired property. We divide into two cases ($x=0$ and $x\neq 0$). If $x=0$, then since $f(0+h)=0$ for all $h\neq 0$ by the definition of $f$, we have that for every $\epsilon>0$, there exists $\delta$ (arbitrarily choose $\delta=1$) such that $|[f(x+h)-f(x-h)]-0|=|[0-0]-0|=0<\epsilon$ for all $h\in\R^1$ satisfying $0<|h-0|<\delta$. It follows that $\lim_{h\to 0}[f(0+h)-f(0-h)]=0$, as desired. On the other hand, if $x\neq 0$, assume WLOG that $x>0$ (the argument is symmetric if $x<0$). Choose $\delta$ such that $0<\delta<x$. Then $\delta<|x-0|$, so $x\pm h\neq 0$ for any $h$ satisfying $0<|h|<\delta$ (for otherwise, we would have $h=\pm x$ and thus $|x|<\delta$, contradicting $\delta<|x|$). It follows that for every $\epsilon>0$, there exists $\delta$ (this chosen $\delta$) such that $|[f(x+h)-f(x-h)]-0|=0<\epsilon$ for all $h\in\R^1$ satisfying $0<|h-0|<\delta$.\par
        Second, we show that $f$ is not continuous. Specifically, $f$ is not continuous at 0 since the fact that $f(y)=0$ for any $y\neq 0$ implies that $\lim_{y\to 0}f(y)=0\neq 1=f(x)$.
    \end{proof}
    \item If $f$ is a continuous mapping of a metric space $X$ into a metric space $Y$, prove that
    \begin{equation*}
        f(\bar{E}) \subset \overline{f(E)}
    \end{equation*}
    for every set $E\subset X$ ($\bar{E}$ denotes the closure of $E$). Show, by an example, that $f(\bar{E})$ can be a proper subset of $\overline{f(E)}$.
    \begin{proof}
        Let $f(x)\in f(\bar{E})$ be arbitrary. It follows by the definition of images that $x\in\bar{E}$. We now divide into two cases ($x\in E$ and $x\in E'$). If $x\in E$, then $f(x)\in f(E)\subset f(E)\cup f(E)'=\overline{f(E)}$ as desired. On the other hand, if $x\in E'$, then every neighborhood of $x$ contains some element of $E$ other than $x$. We now look to show that every neighborhood of $f(x)$ contains some element of $f(E)$ other than $f(x)$. Let $N_\epsilon(f(x))$ be an arbitrary neighborhood of $f(x)$. Since $f(x)$ is continuous, we have that $\lim_{y\to x}f(y)=f(x)$. It follows that there exists $\delta>0$ such that $d_X(y,x)<\delta$ implies $d_Y(f(y),f(x))<\epsilon$. By the hypothesis that $x\in E'$, $N_\delta(x)$ contains some $y\in E$ such that $y\neq x$. Moreover, since $y\in E$, $f(y)\in f(E)$. Additionally, the previous statement implies that $d_Y(f(y),f(x))<\epsilon$. If $f(x)=f(y)$, then $f(x)\in f(E)\subset\overline{f(E)}$, and if $f(x)\neq f(y)$, then our neighborhood $N_\epsilon(f(x))$ contains an element of $f(E)$ other than $f(x)$, as desired.\par
        Let $f:(0,1)\to\R$ be defined by $f(x)=x$. Then
        \begin{equation*}
            f[\overline{(0,1)}] = f[(0,1)]
            = (0,1)
            \subsetneq [0,1]
            = \overline{f[(0,1)]}
        \end{equation*}
        as desired.
    \end{proof}
    \item Let $f$ be a continuous real function on a metric space $X$. Let $Z(f)$ (the \textbf{zero set} of $f$) be the set of all $p\in X$ at which $f(p)=0$. Prove that $Z(f)$ is closed.
    \begin{proof}
        % Let $p$ be a limit point of $Z(f)$, and suppose for the sake of contradiction that $f(p)\neq 0$. For the sake of definiteness, let $f(p)>0$ (the proof is symmetric if $f(p)<0$). Choose $\epsilon$ such that $0<\epsilon<f(p)$. It follows that $0\notin(f(p)-\epsilon,f(p)+\epsilon)$. Since $(f(p)-\epsilon,f(p)+\epsilon)$ is open in $\R$ as a segment, the continuity of $f$ and Theorem 4.8 imply that $f^{-1}[(f(p)-\epsilon,f(p)+\epsilon)]$ is open in $X$. Thus, we may choose $N_r(p)\subset f^{-1}[(f(p)-\epsilon,f(p)+\epsilon)]$. It follows since every $x\in N_r(p)\subset f^{-1}[(f(p)-\epsilon,f(p)+\epsilon)]$ must satisfy $f(x)\in(f(p)-\epsilon,f(p)+\epsilon)$ that $f(x)\neq 0$ for any $x\in N_r(p)$. Thus, $N_r(p)\cap Z(f)=\emptyset$, contradicting the hypothesis that $p$ is a limit point of $Z(f)$.

        By definition, $Z(f)=f^{-1}(\{0\})$. Thus, since $\{0\}$ is closed as a finite set and $f$ is continuous, the Corollary to Theorem 4.8 implies that $Z(f)$ is closed.
    \end{proof}
    \item Let $f$ and $g$ be continuous mappings of a metric space $X$ into a metric space $Y$, and let $E$ be a dense subset of $X$. Prove that $f(E)$ is dense in $f(X)$. If $g(p)=f(p)$ for all $p\in E$, prove that $g(p)=f(p)$ for all $p\in X$. (In other words, a continuous mapping is determined by its values on a dense subset of its domain.)
    \begin{proof}
        % \begin{itemize}
        %     \item Suppose (contradiction): There exists $p\in X$ such that $g(p)\neq f(p)$.
        %     % \item Hypothesis: $p\notin E$
        %     \item Let $d_Y(g(p),f(p))=2\epsilon$.
        %     % \item Consider $N_\epsilon(g(p)),N_\epsilon(f(p))$. By definition, disjoint.
        %     \item Continuity of $g$: There exists $\delta_1>0$ such that $d_X(x,p)<\delta_1$ implies $d_Y(g(x),g(p))<\epsilon$.
        %     \item Continuity of $f$: There exists $\delta_2>0$ such that $d_X(x,p)<\delta_2$ implies $d_Y(f(x),f(p))<\epsilon$.
        %     \item Choose $\delta=\min(\delta_1,\delta_2)$.
        %     \item Consider $N_\delta(p)$.
        %     \item Density of $E$ in $X$: There exists $x\in E$ such that $x\in N_\delta(p)$.
        %     % \item This and $p\notin E$: $x\neq p$.
        %     \item $d_X(x,p)<\delta\leq\delta_1$: $d_Y(g(x),g(p))<\epsilon$.
        %     \item $d_X(x,p)<\delta\leq\delta_2$: $d_Y(f(x),f(p))<\epsilon$.
        %     \item $x\in E$: $f(x)=g(x)$.
        %     \item Triangle inequality of metrics:
        %     \begin{equation*}
        %         d_Y(g(p),f(p)) \leq d_Y(g(p),g(x))+d_Y(f(x),f(p))
        %         < \epsilon+\epsilon = 2\epsilon
        %     \end{equation*}
        %     \item Contradicts $d_Y(g(p),f(p))=2\epsilon$.
        % \end{itemize}


        To prove that $f(E)$ is dense in $f(X)$, it will suffice to show that every $f(x)\in f(X)$ is either an element or a limit point of $f(E)$. Let $f(x)\in f(X)$ be arbitrary. If $f(x)\in f(E)$, we are done. If $f(x)\notin f(E)$, then $x\notin E$. It follows however by the density of $E$ in $X$ that $x$ is a limit point of $E$. Therefore,
        \begin{align*}
            f(x) &\in f(\bar{E})\\
            &\subset \overline{f(E)}\tag*{Exercise 4.2}\\
            &= f(E)\cup f(E)'
        \end{align*}
        so since $f(x)\notin f(E)$, we must have $f(x)\in f(E)'$, as desired.\par
        As to the other part of the proof, suppose for the sake of contradiction that there exists $p\in X$ such that $g(p)\neq f(p)$. Now since $g(p)\neq f(p)$, $d_Y(g(p),f(p))\neq 0$. In particular, we may let $d_Y(g(p),f(p))=2\epsilon$ where $\epsilon>0$. It follows by the continuity of $g$ that there exists $\delta_1>0$ such that $d_X(x,p)<\delta_1$ implies $d_Y(g(x),g(p))<\epsilon$, and symmetrically by the continuity of $f$ that there exists $\delta_2>0$ such that $d_X(x,p)<\delta_2$ implies $d_Y(f(x),f(p))<\epsilon$. Choose $\delta=\min(\delta_1,\delta_2)$. Since $E$ is dense in $X$, there exists $x\in E$ such that $x\in N_\delta(p)$. Consequently, since $d_X(x,p)<\delta\leq\delta_1$, we have that $d_Y(g(x),g(p))<\epsilon$, and symmetrically that $d_Y(f(x),f(p))<\epsilon$. But since $x\in E$, $f(x)=g(x)$. Therefore, we have that
        \begin{equation*}
            d_Y(g(p),f(p)) \leq d_Y(g(p),g(x))+d_Y(f(x),f(p))
            < \epsilon+\epsilon = 2\epsilon
        \end{equation*}
        contradicting the previously proven fact that $d_Y(g(p),f(p))=2\epsilon$.
    \end{proof}
    \item If $f$ is a real continuous function defined on a closed set $E\subset\R^1$, prove that there exist continuous real functions $g$ on $\R^1$ such that $g(x)=f(x)$ for all $x\in E$. (Such functions $g$ are called \textbf{continuous extensions} of $f$ from $E$ to $\R^1$.) Show that the result becomes false if the word "closed" is omitted. Extend the result to vector-valued functions. (Hint: Let the graph of $g$ be a straight line on each of the segments which constitute the complement of $E$ [compare Exercise 2.29]. The result remains true if $\R^1$ is replaced by any metric space, but the proof is not so simple.)
    \begin{proof}
        Since $E$ is closed, $E^c$ is open. Thus, by Exercise 2.29, $E^c$ is the union of an at most countable collection of disjoint segments. In particular, we may let
        \begin{equation*}
            E^c = \bigcup_{i=1}^n(a_i,b_i)
        \end{equation*}
        where $n\in[1,\infty]$ and $(a_i,b_i)\cap(a_j,b_j)=\emptyset$ for any $i\neq j$. Thus, we may let $g:\R^1\to\R$ be defined by
        \begin{equation*}
            g(x) =
            \begin{cases}
                f(x) & x\in E\\
                f(a_i)+\frac{f(b_i)-f(a_i)}{b_i-a_i}(x-a_i) & x\in (a_i,b_i)\subset E^c
            \end{cases}
        \end{equation*}
        Clearly $g(x)=f(x)$ for all $x\in E$. All that remains is to show that $g$ is continuous. To do so, we divide into three cases ($x\in E^\circ$, $x\in E^c$, and $x\in E\setminus E^\circ$).\par\smallskip
        First, suppose $x\in E^\circ$. To prove that $g$ is continuous at $x$, it will suffice to show that for every $\epsilon>0$, there exists a $\delta>0$ such that $|g(y)-g(x)|<\epsilon$ for all $y\in\R^1$ for which $|y-x|<\delta$. Since $x\in E^\circ$, there exists $N_{\delta_1}(x)\subset E$. Additionally, since $f$ is continuous at $x$, there exists a $\delta_2>0$ such that $|f(y)-f(x)|<\epsilon$ for all $y\in E$ for which $|y-x|<\delta_2$. Choose $\delta=\min(\delta_1,\delta_2)$. Now suppose $y\in\R^1$ and $|y-x|<\delta$. Since $|y-x|<\delta\leq\delta_1$, $y\in N_{\delta_1}(x)\subset E$. This combined with the fact that $x\in E$ by hypothesis implies that $g(y)=f(y)$ and $g(x)=f(x)$. It follows since $|y-x|<\delta\leq\delta_2$ that
        \begin{align*}
            |g(y)-g(x)| &= |f(y)-f(x)|\\
            &< \epsilon
        \end{align*}
        as desired.\par\smallskip
        Second, suppose $x\in E^c$. Then $x\in(a_i,b_i)$ for some $i\in\N$. To prove that $g$ is continuous at $x$, it will suffice to show that for every $\epsilon>0$, there exists a $\delta>0$ such that $|g(y)-g(x)|<\epsilon$ for all $y\in\R^1$ for which $|y-x|<\delta$. Let $\epsilon>0$ be arbitrary. Since $x\in(a_i,b_i)$ open, there exists $N_{\delta_1}(x)\subset(a_i,b_i)$. Additionally, let $\delta_2=\epsilon\cdot|(b_i-a_i)/(f(b_i)-f(a_i))|$. Let $\delta=\min(\delta_1,\delta_2)$. Now suppose $y\in\R^1$ and $|y-x|<\delta$. Since $|y-x|<\delta\leq\delta_1$, $y\in N_{\delta_1}(x)\subset(a_i,b_i)$. This combined with the fact that $x\in(a_i,b_i)$ by hypothesis implies that
        \begin{align*}
            g(y) &= f(a_i)+\frac{f(b_i)-f(a_i)}{b_i-a_i}(y-a_i)&
            g(x) &= f(a_i)+\frac{f(b_i)-f(a_i)}{b_i-a_i}(x-a_i)
        \end{align*}
        It follows since $|y-x|<\delta\leq\delta_2$
        \begin{align*}
            |g(y)-g(x)| &= \left| \left[ f(a_i)+\frac{f(b_i)-f(a_i)}{b_i-a_i}(y-a_i) \right]-\left[ f(a_i)+\frac{f(b_i)-f(a_i)}{b_i-a_i}(x-a_i) \right] \right|\\
            &= \left| \frac{f(b_i)-f(a_i)}{b_i-a_i}(y-x) \right|\\
            &= \left| \frac{f(b_i)-f(a_i)}{b_i-a_i} \right|\cdot|y-x|\\
            &< \left| \frac{f(b_i)-f(a_i)}{b_i-a_i} \right|\cdot\epsilon\cdot\left| \frac{b_i-a_i}{f(b_i)-f(a_i)} \right|\\
            &= \epsilon
        \end{align*}
        as desired.\par\smallskip
        Third, suppose $x\in E\setminus E^\circ$.
        % Then $x\in E$ but every neighborhood of $x$ contains some point of $E^c$. It follows that $x\in{E^c}'$. 
        We now show that this means that $x=a_i$ or $x=b_j$ for some $i,j$, or that $x\in E'$. If $x=a_i$ or $x=b_j$ for some $i,j$, then clearly $x\notin E^c$ (for otherwise there would be a segment $(a_k,b_k)$ containing it that is not disjoint from $(a_i,b_i)$ [resp. $(a_j,b_j)$]), i.e., $x\in E$. Additionally, $x\notin E^\circ$ since it's status as the endpoint of a segment means that there are points of $E^c$ arbitrarily close to it. On the other hand, if $x\in E\setminus E^\circ$ and $x\neq a_i,b_j$ for any $i,j$, then we can show that $x\in E'$. Indeed, consider $N_r(x)$. Since $x\notin E^\circ$, there exists a point $y\in E^c$ such that $y\in N_r(x)$. Suppose for the sake of definiteness that $y>x$ (the proof is symmetric if $y<x$). It follows since $y\in E^c$ and $x\notin E^c$ that there exits a segment $(a_k,b_k)$ containing $y$ but not containing $x$. Naturally, this must imply that $x<a_k<y<b_k$. But since each $a_k\in E$ as previously established, and $|a_k-x|<|y-x|<r$, $a_k\in N_r(x)$, as desired. Having established that $x=a_i$ or $x=b_j$ for some $i,j$, or that $x\in E'$, we now divide into these two subcases. In particular, for the first subcase, we divide into three subsubcases ($x=a_i$ and $x\neq b_j$, $x\neq a_i$ and $x=b_j$, and $x=a_i=b_j$).\par
        Suppose first that $x=a_i$ for some $i$ and $x\neq b_j$ for any $j$. To prove that $g$ is continuous at $x$, it will suffice to show that $g(x+)=g(x-)=g(x)$. Since $x=a_i$, we can show by an argument analogous to that used in the second case that $g(x+)=f(a_i)=g(x)$. Additionally, since $x\neq b_j$ for any $j$, there exists $(y,x]\subset E$ for some $y<x$. Thus, we can show by an argument analogous to that used in the first case that $g(x-)=f(x-)=f(x)=g(x)$.\par
        The proof of the second subsubcase is symmetric to that of the first.\par
        For the third subsubcase, simply apply the first part of the proof of the first subsubcase twice, once to each "side" of $x$.\par
        For the second subcase, we know that if $x\in E\setminus E^\circ$, then $x\in E$, so $\lim_{y\to x}f(y)=f(x)$. This combined with the fact that $x\in E'$ implies that $y$ can actually approach $x$. Moreover, we know by the definition of $g$ that if $f$ is bounded in sufficiently small regions of $x$ (as it is), $g$ will be bounded in sufficiently small regions of $x$ by the same bounds (if points of $g$ exceeded the minimal bounds of $f$, then there would have to be some points of $f$ not extended/connected to each other via a straight line). Therefore, $g$ is continuous at $x$, as desired.
    \end{proof}
    \item Let $(X,d_X)$ and $(Y,d_Y)$ be metric spaces, and let $f:E\to Y$ where $E$ is a compact subset of $X$. Consider the \textbf{graph} $G\subset X\times Y$ of $f$, where the metric on $X\times Y$ is $d=d_X+d_Y$, i.e., $d[(x_1,y_1),(x_2,y_2)]=d_X(x_1,x_2)+d_Y(y_1,y_2)$. Show that $f$ is continuous if and only if $G$ is compact. (Hint: There are several ways of doing this. There is a "topological" proof that only uses the fact that compact sets are closed in a metric space, and the fact that a function is continuous if and only if pre-images of closed sets are closed. Another way to go about it is to use sequential compactness [i.e., any sequence contained in a compact set has a convergent subsequence].)
    \begin{proof}
        Suppose first that $f$ is continuous. Define $\gb:E\to X\times Y$ by $\gb(x)=(x,f(x))$ for all $x\in E$. Then since the component functions $g_1,g_2$ are continuous, Theorem 4.10 implies that $\gb$ is continuous. Therefore, since $\gb$ is continuous and $E$ is compact, Theorem 4.14 implies that $\gb(E)=G$ is compact.\par
        Now suppose that $G$ is compact. To prove that $f$ is continuous, the Corollary to Theorem 4.8 tells us that it will suffice to show that for every closed set $C$ in $Y$, $f^{-1}(C)$ is closed in $X$. Let $C$ be an arbitrary closed set in $Y$. Define $\pi_1:G\to X$ and $\pi_2:G\to Y$ by
        \begin{align*}
            \pi_1(x,f(x)) &= x&
            \pi_2(x,f(x)) &= f(x)
        \end{align*}
        We can prove that $\pi_1,\pi_2$ are continuous by choosing $\delta=\epsilon$ in the definition of continuity. It follows by the Corollary to Theorem 4.8 that $\pi_2^{-1}(C)$ is closed in $G$. This combined with the fact that $G$ is compact implies that $\pi_2^{-1}(C)$ is compact. This combined with the fact that $\pi_1$ is continuous implies by Theorem 4.14 that $\pi_1(\pi_2^{-1}(C))=f^{-1}(C)$ is compact. Thus, by Theorem 2.34, $f^{-1}(C)$ is closed, as desired.
    \end{proof}
    \item If $E\subset X$ and if $f$ is a function defined on $X$, the \textbf{restriction} of $f$ to $E$ is the function $g$ whose domain of definition is $E$ such that $g(p)=f(p)$ for $p\in E$. Define $f$ and $g$ on $\R^2$ by
    \begin{align*}
        f(x,y) &=
        \begin{cases}
            0 & (x,y)=(0,0)\\
            \frac{xy^2}{x^2+y^4} & (x,y)\neq(0,0)
        \end{cases}&
        g(x,y) &=
        \begin{cases}
            0 & (x,y)=(0,0)\\
            \frac{xy^2}{x^2+y^6} & (x,y)\neq(0,0)
        \end{cases}
    \end{align*}
    Prove that $f$ is bounded on $\R^2$, that $g$ is unbounded in every neighborhood of $(0,0)$, and that $f$ is not continuous at $(0,0)$; nevertheless, the restrictions of both $f$ and $g$ to every straight line in $\R^2$ are continuous!
    \begin{proof}
        \underline{$f$ is bounded on $\R^2$}: To prove that $f$ is bounded on $\R^2$, it will suffice to show that there exists a real number $M$ such that $|f(x,y)|\leq M$ for all $(x,y)\in\R^2$. Choose $M=1/2$. Let $(x,y)\in\R^2$ be arbitrary. Then
        \begin{align*}
            0 &\leq (y^2-x)^2\\
            0 &\leq y^4-2xy^2+x^2\\
            2xy^2 &\leq x^2+y^4\\
            \frac{xy^2}{x^2+y^4} &\leq \frac{1}{2}
        \end{align*}
        as desired.\par\smallskip
        \underline{$g$ is unbounded in every neighborhood of $(0,0)$}: Let $N_r(0,0)$ be an arbitrary neighborhood of $(0,0)$. Suppose for the sake of contradiction that there exists a real number $M$ such that $|g(x,y)|\leq M$ for all $(x,y)\in N_r(0,0)$. Let $n_1\in\N$ be such that $10^{n_1}>2M$. Let $n_2\in\N$ be such that $(10^{-3n_2},10^{-n_2})\in N_r(0,0)$. Let $n=\max(n_1,n_2)$; note that this implies that $10^n>2M$ and $(10^{-3n},10^{-n})\in N_r(0,0)$. Then by the latter statement and the fact that
        \begin{align*}
            M &< \frac{1}{2}\cdot 10^n\\
            &= \frac{10^{-5n}}{2\cdot 10^{-6n}}\\
            &= \frac{(10^{-3n})(10^{-n})^2}{(10^{-3n})^2+(10^{-n})^6}\\
            &= g(10^{-3n},10^{-n})\\
            &\leq |g(10^{-3n},10^{-n})|
        \end{align*}
        we have found an $(x,y)\in N_r(0,0)$ such that $|g(x,y)|>M$, a contradiction.\par\smallskip
        \underline{$f$ is not continuous at $(0,0)$}: To prove that $f$ is not continuous at $(0,0)$, it will suffice to show that there exists an $\epsilon>0$ such that for all $\delta>0$, there exists $(x,y)\in\R^2$ such that $\norm{(x,y)-(0,0)}<\delta$ and $|f(x,y)-f(0,0)|\geq\epsilon$. Choose $\epsilon=1/2$. Let $\delta>0$ be arbitrary. Choose $(y^2,y)\in\R^2$ such that $\norm{(y^2,y)-(0,0)}=\norm{(y^2,y)}<\delta$. This combined with the fact that
        \begin{align*}
            |f(y^2,y)-f(0,0)| &= \left| \frac{(y^2)(y)^2}{(y^2)^2+(y)^4} \right|\\
            &= \left| \frac{y^4}{2y^4} \right|\\
            &= \left| \frac{1}{2} \right|\\
            &= \epsilon
        \end{align*}
        completes the proof.\par\smallskip
        \underline{The restriction of $f$ to any straight line in $\R^2$ is continuous}: We divide into two cases (straight lines of the form $y=ax+b$ where $a,b\in\R$, and straight lines of the form $x=c$ where $c\in\R$). In the first case, let $\tilde{f}:\{(x,y):y=ax+b\}\to\R$ be the restriction of $f$ to the arbitrary straight line $y=ax+b$ of the first form. Then
        \begin{equation*}
            \tilde{f}(x,y) = \tilde{f}(x,ax+b) = \frac{x(ax+b)^2}{x^2+(ax+b)^4}
        \end{equation*}
        for every $(x,y)\in\{(x,y):y=ax+b\}$. Since the rightmost function above is the result of sums, products, and quotients of the continuous functions $x\mapsto x$ and $x\mapsto ax+b$, Theorem 4.9 asserts that $\tilde{f}$ is continuous on its domain, except possibly when $x^2+(ax+b)^4=0$. However, this will only happen in the special case when $b=0$ and $(x,y)=(x,ax+0)=(x,ax)=(0,0)$. Thus, to complete the proof, we need only show that for every $\epsilon>0$, there exists a $\delta>0$ such that $|\tilde{f}(x,ax)-\tilde{f}(0,0)|<\epsilon$ if $\norm{(x,ax)-(0,0)}<\delta$. Let $\epsilon>0$ be arbitrary. Choose $\delta=\epsilon/a^2$ if $a\neq 0$ (and $\delta=1$ for the trivial case where $a=0$ and thus $\tilde{f}=0$ as well). Let $(x,ax)$ such that $\norm{(x,ax)-(0,0)}=\norm{(x,ax)}<\delta$ be arbitrary. This importantly implies that $|x|<\delta$. Therefore,
        \begingroup
        \allowdisplaybreaks
        \begin{align*}
            |\tilde{f}(x,ax)-\tilde{f}(0,0)| &= \left| \frac{(x)(ax)^2}{(x)^2+(ax)^4} \right|\\
            &= \left| \frac{a^2x^3}{a^4x^4+x^2} \right|\\
            &= \left| \frac{a^2x}{a^4x^2+1} \right|\\
            &\leq \left| \frac{a^2x}{1} \right|\\
            &= a^2\cdot|x|\\
            &< a^2\cdot\frac{\epsilon}{a^2}\\
            &= \epsilon
        \end{align*}
        \endgroup
        as desired.\par
        In the second case, let $\tilde{f}:\{(c,y)\}\to\R$ be the restriction of $f$ to an arbitrary straight line $x=c$ of the second form. A symmetric argument to the other case completes the proof.\par\smallskip
        \underline{The restriction of $g$ to any straight line in $\R^2$ is continuous}: The proof is symmetric to the above.
    \end{proof}
    \item Let $f$ be a real uniformly continuous function on the bounded set $E$ in $\R^1$. Prove that $f$ is bounded on $E$. Show that the conclusion is false if boundedness of $E$ is omitted from the hypothesis.
    \begin{proof}
        Since $E$ is dense in $\bar{E}$ and $f$ is a uniformly continuous real function on $E$, Exercise 4.13 asserts that $f$ has a continuous extension $g$ from $E$ to $\bar{E}$. Since $E\subset\R^1$ is bounded, $\bar{E}\subset\R^1$ is closed and bounded, and hence compact by the Heine-Borel theorem. This combined with the fact that $g$ is continuous implies by Theorem 4.14 that $g(\bar{E})$ is compact. Thus, $g(\bar{E})$ must be closed and bounded by the Heine-Borel theorem, so $f(E)\subset g(\bar{E})$ must be bounded. It follows trivially that $f$ is bounded.\par
        Let $E=\R^1$ and $f:E\to\R$ be defined by $f(x)=x$ for all $x\in E$. Then $f$ is a real uniformly continuous function on $E$ unbounded for which $f(E)=E$ is naturally unbounded.
    \end{proof}
    \item Show that the requirement in the definition of uniform continuity can be rephrased as follows, in terms of diameters of sets: To every $\epsilon>0$, there exists a $\delta>0$ such that $\diam f(E)<\epsilon$ for all $E\subset X$ with $\diam E<\delta$.
    \begin{proof}
        Suppose first that $f:X\to Y$ is uniformly continuous, where $(X,d_X),(Y,d_Y)$ are metric spaces. We wish to prove that to every $\epsilon>0$, there exists a $\delta>0$ such that $\diam f(E)<\epsilon$ for all $E\subset X$ with $\diam E<\delta$. Let $\epsilon>0$ be arbitrary. Since $f$ is uniformly continuous, Definition 4.18 tells us that there exists $\delta>0$ such that $d_Y(f(p),f(q))<\epsilon/2$ for all $p,q\in X$ for which $d_X(p,q)<\delta$. Choose this $\delta$ to be our $\delta$. Let $E$ be an arbitrary subset of $X$ satisfying $\diam E<\delta$. To prove that $\diam f(E)<\epsilon$, it will suffice to show that
        \begin{equation*}
            \sup\{d_Y(f(p),f(q)):p,q\in E\} < \epsilon
        \end{equation*}
        Since $\diam E<\delta$, $d_X(p,q)<\delta$ for all $p,q\in E$. Thus, for all $p,q\in E\subset X$, $d_Y(f(p),f(q))<\epsilon/2$. It follows that
        \begin{equation*}
            \diam f(E) = \sup\{d_Y(f(p),f(q)):p,q\in E\} \leq \frac{\epsilon}{2} < \epsilon
        \end{equation*}
        as desired.\par
        The proof is symmetric in the other direction.
    \end{proof}
    \item Complete the details of the following alternative proof of Theorem 4.19: If $f$ is not uniformly continuous, then for some $\epsilon>0$, there are sequences $\{p_n\},\{q_n\}$ in $X$ such that $d_X(p_n,q_n)\to 0$ but $d_Y(f(p_n),f(q_n))>\epsilon$. Use Theorem 2.37 to obtain a contradiction.
    \begin{proof}
        Let $(X,d_X),(Y,d_Y)$ where the former is compact, and let $f:X\to Y$ be continuous. Now suppose for the sake of contradiction that $f$ is not uniformly continuous. Then there exists a number $2\epsilon>0$ such that for all $\delta>0$, there exist $p,q\in X$ for which $d_X(p,q)<\delta$ but $d_Y(f(p),f(q))\geq 2\epsilon$. Let $\{\delta_n\}_1^\infty$ be defined by $\delta_n=1/n$. For each $\delta_n$, use the above statement to choose $p_n,q_n\in X$ satisfying $d_X(p_n,q_n)<\delta_n$ and $d_Y(f(p_n),f(q_n))\geq 2\epsilon$. It follows that $\{p_n\},\{q_n\}$ are sequences in $X$ such that $d_X(p_n,q_n)\to 0$ but $d_Y(f(p_n),f(q_n))>\epsilon$. Moreover, the ranges of $\{p_n\},\{q_n\}$ are infinite. (Suppose otherwise. Then there would be a pair of terms $p_n,q_n$ that are the closest together. Let these terms be separated by a distance $r$. We know that $r\neq 0$ since $f(p_n),f(q_n)$ are separated by a nonzero distance, hence are distinct, and $f$, as a function, cannot map the same input to distinct outputs. Moreover, there would then not be a pair of terms corresponding to $\delta_m=1/m<r$, which we know to exist by the Archimedean principle, a contradiction.)\par
        Since $f$ is continuous and $X$ is compact, Theorem 4.14 implies that $f(X)$ is compact. Thus, since the ranges of $\{p_n\},\{q_n\},\{f(p_n)\},\{f(q_n)\}$ are infinite subsets of compact sets, they have limit points. Hence, each of these four sequences converges. Moreover, $\{p_n\},\{q_n\}$ converge to the same point $p=q$ since $d_X(p_n,q_n)\to 0$ while $\{f(p_n)\},\{f(q_n)\}$ converge to different points since $d_Y(f(p_n),f(q_n))>\epsilon$. However, this contradicts the continuity of $f$ at $p=q$ since it proves the existence of points arbitrarily close by that map to separate elements of $Y$.
    \end{proof}
    \item Suppose $f$ is a uniformly continuous mapping of a metric space $X$ into a metric space $Y$ and prove that $\{f(x_n)\}$ is a Cauchy sequence in $Y$ for every Cauchy sequence $\{x_n\}$ in $X$. Use this result to give an alternative proof of the theorem stated in Exercise 4.13.
    \begin{proof}
        Let $\{x_n\}$ be an arbitrary Cauchy sequence in $X$. To prove that $\{f(x_n)\}$ is a Cauchy sequence in $Y$, it will suffice to show that for every $\epsilon>0$, there exists an integer $N$ such that $d_Y(f(x_n),f(x_m))<\epsilon$ if $n,m\geq N$. Let $\epsilon>0$ be arbitrary. Since $f$ is uniformly continuous, there exists a $\delta>0$ such that $d_Y(f(p),f(q))<\epsilon$ for all $p,q\in X$ satisfying $d_X(p,q)<\delta$. Moreover, since $\{x_n\}$ is Cauchy in $X$, there exists an integer $N$ such that $d_X(x_n,x_m)<\delta$ if $n,m\geq N$. Choose this $N$ to be our $N$. Let $n,m\geq N$ be arbitrary. Then since $d_X(x_n,x_m)<\delta$, it follows by the above that $d_Y(f(x_n),f(x_m))<\epsilon$, as desired.\par\medskip
        Suppose $E\subset X$ is dense in $(X,d_X)$ and let $f:E\to\R$ be uniformly continuous. We wish to prove that $f$ has a continuous extension $g:X\to\R$.\par
        We first define an extension $g$ of $f$ as follows. For $x\in E$, let $g(x)=f(x)$. For $x\notin E$, choose a sequence $\{x_n\}$ in $E$ that converges to $x$ (we know that one exists by the density of $E$ in $X$). Since $\{x_n\}$ is convergent, Theorem 3.11a implies that it's Cauchy. Since $\{x_n\}$ is Cauchy, we have by the above that $\{f(x_n)\}$ is Cauchy. It follows by Theorem 3.11c that $\{f(x_n)\}$ converges to a point in $\R$ that we may define to be $g(x)$.\par
        We now prove that $g$ as defined is continuous. Suppose for the sake of contradiction that $g$ is not continuous at some $x\in X$. Then there exists an $\epsilon>0$ such that for all $\delta>0$, there exists $y\in X$ satisfying $d(x,y)<\delta$ and $d_Y(g(x),g(y))\geq\epsilon$. We use this statement to define a sequence $\{y_n\}$ of points in $X$, none of which is equal to $x$, that converges to $x$. First, let $\{\delta_n\}_1^\infty$ be defined by $\delta_n=1/n$. Then let $\{y_n\}_1^\infty$ be a sequence where each $y_n$ satisfies $y_n\in E$, $d(x,y_n)<\delta_n$, and $d_Y(g(x),g(y_n))\geq\epsilon$; we know such a point exists for each $n$ by the above condition and by the density of $E$ in $X$, and that none of the points equals $x$ since there is a nonzero distance between $g(x)$ and $g(y_n)$ and $g$ is a function (i.e., cannot have multiple definitions on one object). Since $\delta_n\to 0$, $y_n\to x$. However, $g(y_n)\nrightarrow g(x)$. If $x\in E$, then this means that we can find points of $E$ arbitrarily close to $x$ that nevertheless map to values isolated from $g(x)=f(x)$, contradicting the continuity of $f$. If $x\notin E$, then $g(x)$ is \emph{defined} to be the limit of $\{y_n\}$, leading to a contradiction with the definition of $g$.
    \end{proof}
    \item A uniformly continuous function of a uniformly continuous function is uniformly continuous. State this more precisely and prove it.
    \begin{proof}
        Let $(X,d_X),(Y,d_Y),(Z,d_Z)$ be metric spaces, and let $E\subset X$. Suppose $f:E\to Y$ and $g:f(E)\to Z$ are uniformly continuous. Then the composition $h:E\to Z$ of $f$ and $g$ is uniformly continuous.\par
        To prove that $h$ is uniformly continuous, it will suffice to show that for every $\epsilon>0$, there exists $\delta>0$ such that $d_Z(h(x),h(x'))<\epsilon$ for all $x,x'\in E$ satisfying $d_X(x,x')<\delta$. Let $\epsilon>0$ be arbitrary. Since $g$ is uniformly continuous, there exists $\eta>0$ such that $d_Z(g(y),g(y'))<\epsilon$ for all $y,y'\in f(E)$ satisfying $d_Y(y,y')<\eta$. Since $f$ is uniformly continuous, there exists $\delta>0$ such that $d_Y(f(x),f(x'))<\eta$ for all $x,x'\in E$ satisfying $d_X(x,x')<\delta$. Let $x,x'$ be arbitrary points of $E$ that satisfy $d_X(x,x')<\delta$. Then $d_Y(f(x),f(x'))<\eta$. It follows that
        \begin{equation*}
            d_Z(h(x),h(x')) = d_Z(g(f(x)),g(f(x'))) < \epsilon
        \end{equation*}
        as desired.
    \end{proof}
    \item Let $E$ be a dense subset of a metric space $X$ and let $f$ be a uniformly continuous \emph{real} function defined on $E$. Prove that $f$ has a continuous extension from $E$ to $X$ (see Exercise 4.5 for terminology). Uniqueness follows from Exercise 4.4. (Hint: For each $p\in X$ and each positive integer $n$, let $V_n(p)$ be the set of all $q\in E$ with $d(p,q)<1/n$. Use Exercise 4.9 to show that the intersection of the closures of the sets $f(V_1(p)),f(V_2(p)),\dots$ consists of a single point, say $g(p)$, of $\R^1$. Prove that the function $g$ so defined on $X$ is the desired extension of $f$.) Could the range space $\R^1$ be replaced by $\R^k$? By any compact metric space? By any complete metric space? By any metric space?
    \begin{proof}
        % ${\color{white}hi}$\\
        % \begin{itemize}
        %     \item Defining $g$.
        %     \begin{itemize}
        %         \item Let $p\in X$ be arbitrary.
        %         \item Define $V_n(p)=\{q\in E:d(p,q)<1/n\}$ for all $n\in\N$.
        %         \item WTS: $\bigcap_1^\infty\overline{f(V_n(p))}=\{g(p)\}$
        %         % Theorem 3.10b: If $K_n$ is a sequence of compact sets in $X$ such that $K_n\supset K_{n+1}$ for all $n\in\N$ and if $\lim_{n\to\infty}\diam K_n=0$, then $\bigcap_1^\infty K_n$ consists of exactly one point.
        %         % Exercise 4.9: If $f$ is uniformly continuous, then to every $\epsilon>0$, there exists a $\delta>0$ such that $\diam f(E)<\epsilon$ for all $E\subset X$ with $\diam E<\delta$.
        %         \begin{itemize}
        %             \item By definition: $V_n(p)\supset V_{n+1}(p)$ for all $n\in\N$.
        %             \item Thus: $f(V_n(p))\supset f(V_{n+1}(p))$ for all $n\in\N$.
        %             \item Thus: $\overline{f(V_n(p))}\supset\overline{f(V_{n+1}(p))}$ for all $n\in\N$.
        %             \item Thus: $\bigcap_{n\in\N}\overline{f(V_n(p))}=\bigcap_{k\in K}\overline{f(V_k(p))}$ for any infinite $K\subset\N$.
        %             \item Let $\epsilon_n=1/n$ for all $n\in\N$.
        %             \item Exercise 4.9 ($f$ is uniformly continuous): There exists $\delta_n>0$ such that $\diam f(E)<\epsilon_n$ for all $E\subset X$ with $\diam E<\delta_n$ for all $n\in\N$.
        %             \item Archimedean principle: To each $\delta_n$, assign a $k_n\in\N$ such that $2/k_n<\delta_{1,\dots,n}$ and $k_n>k_{1,\dots,n-1}$.
        %             \item Let $K=\{k_n:n\in\N\}$.
        %             \item By the construction: $\overline{f(V_{k_n}(p))}\supset\overline{f(V_{k_{n+1}}(p))}$. Additionally, since $\diam V_{k_n}(p)<\delta_n$, $\diam f(V_{k_n}(p))<\epsilon_n$ so $\diam\overline{f(V_{k_n}(p))}\leq\epsilon_n$
        %             \item Comparison test ($\lim_{n\to\infty}\epsilon_n=0$): $\lim_{n\to\infty}\diam\overline{f(V_{k_n}(p))}=0$.
        %             \item (Each $\diam\overline{f(V_{k_n}(p))}\leq\epsilon_n$): $\overline{f(V_{k_n}(p))}$ is bounded for all $n\in\N$.
        %             \item Heine-Borel theorem (Each $\overline{f(V_{k_n}(p))}$ is bounded and closed): $\overline{f(V_{k_n}(p))}$ is compact for all $n\in\N$.
        %             \item Theorem 3.10b ($\{\overline{f(V_{k_n}(p))}\}$ are compact, decreasing sets with $\lim_{n\to\infty}\diam\overline{f(V_{k_n}(p))}=0$): $\bigcap_{k_n\in K}\overline{f(V_{k_n}(p))}=\{g(p)\}$.
        %             \item Thus: $\bigcap_{n\in\N}\overline{f(V_n(p))}=\{g(p)\}$.
        %         \end{itemize}
        %         \item Let $g(p)$ equal the one element in the intersection.
        %     \end{itemize}
        %     \item $g(p)=f(p)$ for all $p\in E$.
        %     \begin{itemize}
        %         \item Suppose (contradiction): $g(p)\neq f(p)$ for some $p\in E$.
        %         \item Let $|g(p)-f(p)|=2\epsilon$.
        %         \item Exercise 4.9 ($f$ is uniformly continuous): There exists $\delta>0$ such that $\diam f(E)<\epsilon$ for all $E\subset X$ with $\diam E<\delta$.
        %         \item Archimedean principle: Choose $2/n<\delta$.
        %         \item Definition of diameter: $\diam V_n(p)<\delta$.
        %         \item Above condition: $\diam\overline{f(V_n(p))}<\epsilon$.
        %         \item ($g(p)\in\bigcap_{n\in\N}\overline{f(V_n(p))}$): $g(p)\in\overline{f(V_n(p))}$.
        %         \item ($p\in V_n(p)$): $f(p)\in\overline{f(V_n(p))}$.
        %         \item Definition of diameter: $|g(p)-f(p)|\leq\epsilon<2\epsilon$, contradiction.
        %     \end{itemize}
        %     \item $g$ is continuous.
        %     \begin{itemize}
        %         \item Argument is symmetric to the above (we can find an $\overline{f(V_n(p))}$ of which $g(p)$ is not a part, a contradiction).
        %     \end{itemize}
        %     \item The only place where we make use of the properties of $\R$ is when we use the Heine-Borel theorem to assert that any closed and bounded set is compact. We can still make this logical step in $\R^k$ (Theorem 2.41), in compact metric spaces (Theorem 2.35), and in complete metric spaces (Definition 3.12), but not in general metric spaces (Exercise 2.16).
        % \end{itemize}


        We begin by defining a function $g$ on $X$. After defining it, we will prove it is a continuous extension of $f$ from $E$ to $X$. Let's begin.\par\smallskip
        Let $p\in X$ be arbitrary. Define the family of sets $\{V_n(p)\}_1^\infty$ by $V_n(p)=\{q\in E:d(q,p)<1/n\}$ for all $n\in\N$. We now show that the intersection of the images of every set in this collection under $f$ contains exactly one point in $\R$ that we may define to be $g(p)$. First off, note that by definition, $V_n(p)\supset V_{n+1}(p)$ for all $n\in\N$. Thus, $f(V_n(p))\supset f(V_{n+1}(p))$ and hence $\overline{f(v_n(p))}\supset\overline{f(V_{n+1}(p))}$ for all $n\in\N$. It follows that $\bigcap_{n\in\N}\overline{f(V_n(p))}=\bigcap_{k\in K}\overline{f(V_k(p))}$ for any infinite $K\subset\N$. Indeed, to prove our desired result that $\bigcap_{n\in\N}\overline{f(V_n(p))}$ is a singleton set, we need only show that the intersection of some infinite subset of $\{\overline{f(V_n(p))}\}$ is a singleton set. We may do so via Theorem 3.10b; the invocation of said result requires that we construct an infinite collection $\{\overline{f(V_{k_n}(p))}\}\subset\{\overline{f(V_n(p))}\}$ of compact, decreasing sets with $\lim_{n\to\infty}\diam\overline{f(V_{k_n}(p))}=0$. We will perform the construction first, and then confirm that it meets the three criteria.\par
        Let $\epsilon_n=1/n$ for each $n\in\N$. Since $f$ is uniformly continuous, Exercise 4.9 tells us that there exists a $\delta_n>0$ such that $\diam f(F)<\epsilon_n$ for all $F\subset E$ with $\diam F<\delta_n$ for each $n\in\N$. By consecutive applications of the Archimedean principle and using strong induction, to each $\delta_n$, assign a $k_n\in\N$ such that $2/k_n<\delta_i$ ($i=1,\dots,n$) and $k_n>k_i$ ($i=1,\dots,n-1$). Let $K=\{k_n:n\in\N\}$. This completes the construction. Now for the check, let $n\in\N$ be arbitrary.\par
        To confirm that $\overline{f(V_{k_n}(p))}$ is compact, the Heine-Borel theorem tells us that it will suffice to demonstrate that $\overline{f(V_{k_n}(p))}$ is closed and bounded. By Theorem 2.27a, $\overline{f(V_{k_n}(p))}$ is closed, as desired. Since $\diam V_{k_n}(p)=2/k_n<\delta_n$, we have by the above that $\diam f(V_{k_n}(p))<\epsilon_n$. Hence $\diam\overline{f(V_{k_n}(p))}\leq\epsilon_n$, verifying that $\overline{f(V_{k_n}(p))}$ is bounded, as desired.\par
        To confirm that $\overline{f(V_{k_n}(p))}\supset\overline{f(V_{k_{n+1}}(p))}$, it will suffice to show that $V_{k_n}(p)\supset V_{k_{n+1}}(p)$. But we know this to be true by the definition of $V_n(p)$ and the fact that $k_n<k_{n+1}$ by the construction, as desired.\par
        To confirm that $\lim_{n\to\infty}\diam\overline{f(V_{k_n}(p))}=0$, we will use the squeeze theorem. In particular, since $\lim_{n\to\infty}0=0$, $\lim_{n\to\infty}\epsilon_n=0$, and $0\leq\diam\overline{f(V_{k_n}(p))}\leq\epsilon_n$ for all $n\in\N$, we must have that $\lim_{n\to\infty}\diam\overline{f(V_{k_n}(p))}=0$, as desired.\par
        Having established that
        \begin{equation*}
            \left| \bigcap_{n\in\N}\overline{f(V_n(p))} \right| = \left| \bigcap_{k_n\in K}\overline{f(V_{k_n}(p))} \right| = 1
        \end{equation*}
        we may define $\{g(p)\}=\bigcap_{n\in\N}\overline{f(V_n(p))}$.\par\smallskip
        We now seek to prove that $g(p)=f(p)$ for all $p\in E$. Suppose for the sake of contradiction that for some $p\in E$, $g(p)\neq f(p)$. Let $|g(p)-f(p)|=2\epsilon>0$. Since $f$ is uniformly continuous, Exercise 4.9 asserts that there exists a $\delta>0$ such that $\diam f(F)<\epsilon$ for all $F\subset E$ with $\diam F<\delta$. Now use the Archimedean principle to choose $2/m<\delta$. It follows by the definition of diameter that $\diam V_m(p)<\delta$. Thus, by the above condition, $\diam\overline{f(V_m(p))}<\epsilon$. Additionally, since $g(p)\in\bigcap_{n\in\N}\overline{f(V_n(p))}$, $g(p)\in\overline{f(V_m(p))}$, and since $p\in V_n(p)$, $f(p)\in\overline{f(V_m(p))}$. But this implies by the definition of diameter that $|g(p)-f(p)|\leq\epsilon<2\epsilon$, a contradiction.\par\smallskip
        A symmetric argument to the above proves that $g$ is continuous (specifically, if there is a discontinuity at $g(p)$, then we can find an $\overline{f(V_m(p))}$ of which $g(p)$ is not an element, contradicting the way $g(p)$ is defined).\par\medskip
        As to the other questions, the only place where we make use of the properties of $\R$ is when we use the Heine-Borel theorem to assert that any closed and bounded set is compact. We can still make this logical step in $\R^k$ (Theorem 2.41), in compact metric spaces (Theorem 2.35), and in complete metric spaces (Definition 3.12), but not in general metric spaces (Exercise 2.16).
    \end{proof}
    \item Let $I=[0,1]$ be the closed unit interval. Suppose $f$ is a continuous mapping of $I$ into $I$. Prove that $f(x)=x$ for at least one $x\in I$.
    \begin{proof}
        Consider the function $g:I\to\R$ defined by $g(x)=f(x)-x$. To prove that $f(x)=x$ for some $x\in I$, it will suffice to show that $g(x)=0$ for some $x\in I$. We divide into two cases ($g(0)=0$ or $g(1)=0$, and $g(0)\neq 0$ and $g(1)\neq 0$). In the first case, we are done immediately. In the second case, we have
        \begin{align*}
            g(0) = f(0)-0 &\neq 0&
            g(1) = f(1)-1 &\neq 0\\
            f(0) &\neq 0&
            f(1) &\neq 1
        \end{align*}
        It follows since $f(0),f(1)\in[0,1]$ that $f(0)>0$ and $f(1)<1$. Thus, $g(0)=f(0)-0>0$ and $g(1)=f(1)-1<0$. Additionally, since $f$ and $x\mapsto x$ are continuous, Theorem 4.9 asserts that the difference of them (i.e., $g$) is continuous as well. Therefore, since $g$ is a continuous real function on the interval $[0,1]$ and $g(0)>0>g(1)$, Theorem 4.23 asserts that there exists a point $x\in(0,1)$ such that $g(x)=0$, as desired.
    \end{proof}
    \item Call a mapping of $X$ into $Y$ open if $f(V)$ is an open set in $Y$ whenever $V$ is an open set in $X$. Prove that every continuous open mapping of $\R^1$ into $\R^1$ is monotonic.
    \begin{proof}
        % ${\color{white}hi}$
        % \begin{itemize}
        %     \item $f$ is 1-1.
        %     \begin{itemize}
        %         \item Suppose $x\neq y$.
        %         \item WLOG let $x<y$.
        %         \item ($f$ is open): $f[(x,y)]$ is open.
        %         \item Theorem 4.22 ($f$ continuous and $(x,y)$ connected): $f((x,y))$ is connected.
        %         \item Theorem 4.19 ($f$ continuous and $[x,y]$ compact): $f$ is uniformly continuous on $[x,y]$.
        %         \item Exercise 4.8 ($f$ real, uniformly continuous on $(x,y)$ bounded): $f$ is bounded on $(x,y)$.
        %         \item ($f((x,y))$ open, connected, and bounded): $f((x,y))=(c,d)$ where $c,d\in\R$ are distinct.
        %         \item Theorem 4.14 ($f$ continuous and $[x,y]$ compact): $f([x,y])$ compact.
        %         \item Theorem 4.22 ($f$ continuous and $[x,y]$ connected): $f([x,y])$ is connected.
        %         \item ($f([x,y])$ compact and connected): $f([x,y])=[c,d]$.
        %         \item Thus, $f(x)=c$ and $f(y)=d$, or vice versa. Either way, $f(x)\neq f(y)$, as desired.
        %     \end{itemize}
        %     \item Suppose (contradiction): $f$ is not monotonic.
        %     \begin{itemize}
        %         \item Thus: There exist $x<y<z$ such that $f(x)<f(y)$ and $f(y)>f(z)$ (or, symmetrically, such that $f(x)>f(y)$ and $f(y)<f(z)$).
        %         \item ($f$ is 1-1, $x\neq z$): $f(x)\neq f(z)$.
        %         \item Case 1: $f(x)<f(z)$.
        %         \begin{itemize}
        %             \item Hypothesis: $f(x)<f(z)<f(y)$.
        %             \item Theorem 4.23 ($f$ continuous, real on $[x,y]$, $f(x)<f(z)<f(y)$): There exists $c\in(x,y)$ such that $f(c)=f(z)$.
        %             \item ($f$ is 1-1): $c=z$.
        %             \item Thus: $x<z<y$, contradiction.
        %         \end{itemize}
        %         \item Case 2: $f(x)>f(z)$.
        %         \begin{itemize}
        %             \item Symmetric.
        %         \end{itemize}
        %     \end{itemize}
        % \end{itemize}


        We will first prove that $f$ is 1-1. Suppose $x\neq y$, and WLOG let $x<y$. We seek to demonstrate that $f((x,y))=(c,d)$ where $c,d\in\R$ are distinct, and that $f([x,y])=[c,d]$; it will follow that $f(x)=c$ and $f(y)=d$ or vice versa, proving either way that $f(x)\neq f(y)$ as desired. Let's begin. To demonstrate the first claim, it will suffice to show that $f((x,y))$ is open, connected, and bounded. Since $f$ is open, $f((x,y))$ is open. Since $f$ is continuous and $(x,y)$ is connected, Theorem 4.22 asserts that $f((x,y))$ is connected. Since $f$ is continuous and $[x,y]$ is compact, Theorem 4.19 asserts that $f$ is uniformly continuous on $[x,y]$; hence $f$ is bounded on $(x,y)$ by Exercise 4.8, as desired. On the other hand, to demonstrate the second claim, it will suffice to show that $f([x,y])$ is compact and connected. Since $f$ is continuous and $[x,y]$ is compact, Theorem 4.14 implies that $f([x,y])$ is compact. Since $f$ is continuous and $[x,y]$ is connected, Theorem 4.22 implies that $f([x,y])$ is connected, as desired.\par
        Now suppose for the sake of contradiction that $f$ is not monotonic. Then there exist $x<y<z$ such that $f(x)<f(y)$ and $f(y)>f(z)$ (or, symmetrically, such that $f(x)>f(y)$ and $f(y)<f(z)$). Since $f$ is 1-1 and $x\neq z$, $f(x)\neq f(z)$. We divide into two cases ($f(x)<f(z)$ and $f(x)>f(z)$). If $f(x)<f(z)$, then $f(x)<f(z)<f(y)$ by hypothesis. Thus, since $f$ is a continuous real function on $[x,y]$ and $f(x)<f(z)<f(y)$, Theorem 4.23 asserts that there exists a $c\in(x,y)$ such that $f(c)=f(z)$. But since $f$ is 1-1, this implies that $c=z$, meaning that $x<z<y$, a contradiction. The proof of the other case is symmetric.
    \end{proof}
    \item Let $[x]$ denote the largest integer contained in $x$, that is, $[x]$ is the integer such that $x-1<[x]\leq x$; and let $(x)=x-[x]$ denote the fractional part of $x$. What discontinuities do the functions $[x]$ and $(x)$ have?
    \begin{proof}
        $[x]$ has a simple discontinuity at each $z\in\Z$. At each $z\in\Z$, $f(z+)=z$ and $f(z-)=z-1$. At each $x\notin\Z$, $f(x+)=f(x-)=[x]$.\par
        $(x)$ also has a simple discontinuity at each $z\in\Z$. At each $z\in\Z$, $f(z+)=0$ and $f(z-)=1$. At each $x\notin\Z$, $f(x+)=f(x-)=(x)$.
    \end{proof}
    \item Let $f$ be a real function defined on $(a,b)$. Prove that the set of points at which $f$ has a simple discontinuity is at most countable. (Hint: Let $E$ be the set on which $f(x-)<f(x+)$. With each point $x$ of $E$, associate a triple $(p,q,r)$ of rational numbers such that
    \begin{enumerate}
        \item $f(x-)<p<f(x+)$;
        \item $a<q<t<x$ implies $f(t)<p$;
        \item $x<t<r<b$ implies $f(t)>p$.
    \end{enumerate}
    The set of all such triples is countable. Show that each triple is associated with at most one point of $E$. Deal similarly with the other possible types of simple discontinuities.)
    \begin{proof}
        Let $D$ be the set of points at which $f$ has a simple discontinuity. Let $E$ be the set of all $x\in D$ such that $f(x-)<f(x+)$. Let $x\in E$ be arbitrary. We now show that we can choose a $(p,q,r)$ pertaining to $x$ as described in the hint. By the density of $\Q\subset\R$, choose $p\in\Q$ such that $f(x-)<p<f(x+)$. Let $\epsilon=p-f(x-)$. Since $\lim_{t\to x^-}f(t)=f(x-)$, there exists a $\delta>0$ such that if $t\in(a,b)$ and $0<x-t<\delta$, then $|f(t)-f(x-)|<\epsilon=p-f(x-)$; in particular, $f(t)<p$. Choose $q\in\Q$ such that $x-\delta<q<x$. Choose $r$ symmetrically, as per the hint.\par
        We now show that if $y\in E$ such that $(p,q,r)$ pertain to $y$, then $y=x$. Suppose for the sake of contradiction that $y\neq x$. WLOG let $x<y$. Choose $q<x<t<y<r$. It follows since $q<t<y$ that $f(t)<p$. It follows since $x<t<r$ that $f(t)>p$, a contradiction.\par
        We can treat the set $F$ of all $x\in D$ such that $f(x-)>f(x+)$ symmetrically.\par
        Having defined an injective function from both $E$ and $F$ to $\Q^3$, we know that $E$ and $F$ are at most countable. Thus, their union ($D$) is also at most countable, as desired.
    \end{proof}
    \item Every rational $x$ can be written in the form $x=m/n$, where $n>0$ and $m$ and $n$ are integers without any common divisors. When $x=0$, we take $n=1$. Consider the function $f$ defined on $\R^1$ by
    \begin{equation*}
        f(x) =
        \begin{cases}
            0 & x\notin\Q\\
            \frac{1}{n} & x=\frac{m}{n}
        \end{cases}
    \end{equation*}
    Prove that $f$ is continuous at every irrational point, and that $f$ has a simple discontinuity at every rational point.
    \item Suppose $f$ is a real function with domain $\R^1$ which has the intermediate value property: If $f(a)<c<f(b)$, then $f(x)=c$ for some $x$ between $a$ and $b$. Suppose also, for every rational $r$, that the set of all $x$ with $f(x)=r$ is closed. Prove that $f$ is continuous. (Hint: If $x_n\to x_0$, but $f(x_n)>r>f(x_0)$ for some $r$ and all $n$, then $f(t_n)=r$ for some $t_n$ between $x_0$ and $x_n$; thus, $t_n\to x_0$. Find a contradiction. \parencite{bib:Fine}.)
    \item If $E$ is a nonempty subset of a metric space $X$, define the \textbf{distance} from $x\in X$ to $E$ by
    \begin{equation*}
        \rho_E(x) = \inf_{z\in E}d(x,z)
    \end{equation*}
    \begin{enumerate}
        \item Prove that $\rho_E(x)=0$ if and only if $x\in\bar{E}$.
        \begin{proof}
            Suppose first that $\rho_E(x)=0$. Let $N_r(x)$ be an arbitrary neighborhood of $x$. Since $\inf_{z\in E}d(x,z)=0$, there exists a $z\in E$ such that $0\leq d(x,z)<r$. This $z\in E$ will therefore be an element of $N_r(x)$, proving that $x\in\bar{E}$, as desired.
            Now suppose that $x\in\bar{E}$. Then there is some point of $E$ in every neighborhood of $x$. Thus, since there are elements $z\in E$ arbitrarily close to $x$, there are elements $z\in E$ that make $d(x,z)$ arbitrarily small. Thus, $\rho_E(x)=\inf_{z\in E}d(x,z)=0$, as desired.
        \end{proof}
        \item Prove that $\rho_E$ is a uniformly continuous function on $X$, by showing that
        \begin{equation*}
            |\rho_E(x)-\rho_E(y)| \leq d(x,y)
        \end{equation*}
        for all $x,y\in X$. (Hint: $\rho_E(x)\leq d(x,z)\leq d(x,y)+d(y,z)$, so that $\rho_E(x)\leq d(x,y)+\rho_E(y)$.)
        \begin{proof}
            Let $x,y\in X$ be arbitrary. We have that
            \begin{equation*}
                \rho_E(x) \leq d(x,z) \leq d(x,y)+d(y,z)
            \end{equation*}
            for all $z\in E$ by the definition of $\rho_E$. In particular, considering a sequence $\{z_n\}$ in $E$ such that $d(y,z_n)\to\inf_{z\in E}d(y,z)$ yields
            \begin{align*}
                \rho_E(x) &\leq d(x,y)+\rho_E(y)\\
                \rho_E(x)-\rho_E(y) &\leq d(x,y)
            \end{align*}
            Interchanging the roles of $x$ and $y$ in the above algebra yields
            \begin{equation*}
                \rho_E(y)-\rho_E(x) \leq d(y,x) = d(x,y)
            \end{equation*}
            Therefore, we have that
            \begin{equation*}
                |\rho_E(x)-\rho_E(y)| \leq d(x,y)
            \end{equation*}
            It follows that if we want $|\rho_E(x)-\rho_E(y)|<\epsilon$, we need only require that $d(x,y)<\delta=\epsilon$, so $\rho_E$ is uniformly continuous, as desired.
        \end{proof}
    \end{enumerate}
    \item Suppose $K$ compact and $F$ closed are disjoint sets in a metric space $X$. Prove that there exists $\delta>0$ such that $d(p,q)>\delta$ if $p\in K$, $q\in F$. (Hint: $\rho_F$ is a continuous positive function on $K$.) Show that the conclusion may fail for two disjoint closed sets if neither is compact.
    \begin{proof}
        Since $F$ is closed, $F=\bar{F}$. Thus, by Exercise 4.20a, $\rho_F(x)=0$ if and only if $x\in F$. It follows since $K$ is disjoint from $F$ that $\rho_F(x)\neq 0$ for all $x\in K$. In particular, since the distance function is strictly nonnegative, $\rho$ must be strictly nonnegative, meaning that $\rho_F(x)>0$ for all $x\in K$. Additionally, we have by Exercise 4.20b that $\rho_F$ is uniformly continuous on $X$. Thus, since $\rho_F$ is continuous and $K$ is compact, Theorem 4.14 asserts that $\rho_F(K)$ is compact. This combined with the previous result implies that $0\notin\rho_F(K)$ and, since $\rho_F(K)$ is closed as a compact set, 0 is isolated from $\rho_F(K)$. Consequently, there exists $2\delta>0$ such that $N_{2\delta}(0)\cap\rho_F(K)=\emptyset$. It follows that $\rho_F(p)\geq 2\delta>\delta$ for all $p\in K$. Therefore, if $p\in K$ and $q\in F$, then
        \begin{equation*}
            d(p,q) \geq \rho_F(p) >\delta
        \end{equation*}
        as desired.\par
        As a counterexample, consider the sets $A,B$ of all rational numbers less than $\sqrt{2}$ and all rational numbers greater than $\sqrt{2}$, respectively. $A$ and $B$ are both closed and disjoint, but since we can find rational numbers arbitrarily close to $\sqrt{2}$ from both sides, the minimum distance between two points in the sets converges to zero.
    \end{proof}
    \item Let $A$ and $B$ be disjoint nonempty closed sets in a metric space $X$, and define
    \begin{equation*}
        f(p) = \frac{\rho_A(p)}{\rho_A(p)+\rho_B(p)}
    \end{equation*}
    for all $p\in X$. Show that $f$ is a continuous function on $X$ whose range lies in $[0,1]$, that $f(p)=0$ precisely on $A$, and that $f(p)=1$ precisely on $B$. This establishes a converse of Exercise 4.3: Every closed set $A\subset X$ is $Z(f)$ for some continuous real $f$ on $X$. Setting
    \begin{align*}
        V &= f^{-1}([0,\tfrac{1}{2}))&
        W &= f^{-1}((\tfrac{1}{2},1])
    \end{align*}
    show that $V$ and $W$ are open and disjoint, and that $A\subset V$, $B\subset W$. (Thus pairs of disjoint closed sets in a metric space can be covered by pairs of disjoint open sets. This property of metric spaces is called \textbf{normality}.)
    \begin{proof}
        Since $f$ is the result of sums and quotients of continuous (Exercise 4.20b) functions, Theorem 4.9 asserts that $f$ is continuous, except possibly where $\rho_A(p)+\rho_B(p)=0$. However, this will never be the case: Since both functions in the sum are nonnegative, the sum can only be zero if both functions are equal to zero. But if $\rho_A(p)=0$ and $\rho_B(p)=0$, then $p\in A$ and $\rho_B(p)=0$ by Exercise 4.20a, contradicting the fact that $A,B$ are disjoint. Thus, $f$ is everywhere continuous, as desired.\par
        Since $\rho_A,\rho_B$ are nonnegative, $0\leq\rho_A(p)\leq\rho_A(p)+\rho_B(p)$ for all $p\in X$. Thus, 
        \begin{equation*}
            0 \leq \frac{\rho_A(p)}{\rho_A(p)+\rho_B(p)} = f(p) \leq 1
        \end{equation*}
        for all $p\in X$, as desired.\par
        If $p\in A$, then
        \begin{align*}
            f(p) &= \frac{\rho_A(p)}{\rho_A(p)+\rho_B(p)}\\
            &= \frac{0}{0+\rho_B(p)}\tag*{Exercise 4.20a}\\
            &= 0
        \end{align*}
        as desired. On the other hand, if $f(p)=0$, then $\rho_A(p)=0$, so Exercise 4.20a implies that $p\in A$, as desired.\par
        If $p\in B$, then
        \begin{align*}
            f(p) &= \frac{\rho_A(p)}{\rho_A(p)+\rho_B(p)}\\
            &= \frac{\rho_A(p)}{\rho_A(p)+0}\tag*{Exercise 4.20a}\\
            &= 1
        \end{align*}
        as desired. On the other hand, if $f(p)=1$, then $\rho_B(p)=0$, so Exercise 4.20a implies that $p\in B$, as desired.\par\smallskip
        Since $[0,\frac{1}{2})$ is open in $[0,1]$ and $f$ is continuous, Theorem 4.8 asserts that $V=f^{-1}([0,\frac{1}{2}))$ is open in $X$. Similarly, $W$ is open in $X$. Additionally, $V,W$ are disjoint since if $x\in V\cap W$, then $f(x)<1/2$ and $f(x)>1/2$, a contradiction. Lastly, if $p\in A$, the $f(p)=0$, so $p\in V$. Similarly, $B\subset W$.
    \end{proof}
    \item A real-valued function $f$ defined in $(a,b)$ is said to be \textbf{convex} if
    \begin{equation*}
        f(\lambda x+(1-\lambda)y) \leq \lambda f(x)+(1-\lambda)f(y)
    \end{equation*}
    whenever $a<x<b$, $a<y<b$, and $0<\lambda<1$. Prove that every convex function is continuous. Prove that every increasing convex function of a convex function is convex. (For example, if $f$ is convex, so is $\e[f]$.) If $f$ is convex in $(a,b)$ and if $a<s<t<u<b$, show that
    \begin{equation*}
        \frac{f(t)-f(s)}{t-s} \leq \frac{f(u)-f(s)}{u-s} \leq \frac{f(u)-f(t)}{u-t}
    \end{equation*}
    \item Assume that $f$ is a continuous real function defined in $(a,b)$ such that
    \begin{equation*}
        f\left( \frac{x+y}{2} \right) \leq \frac{f(x)+f(y)}{2}
    \end{equation*}
    for all $x,y\in(a,b)$. Prove that $f$ is convex.
    \begin{proof}
        Suppose for the sake of contradiction that $f$ is not convex. Then there exist $c,d\in(a,b)$ and $\lambda\in(0,1)$ such that $f(\lambda c+(1-\lambda)d)>\lambda f(c)+(1-\lambda)f(d)$. WLOG let $c<d$ (we know $c\neq d$ since the two sides of the convexity condition would be equal were $c$ equal to $d$). Consider $g:(a,b)\to\R$ defined by
        \begin{equation*}
            g(x) = f(x)-\left[ f(c)+\frac{f(d)-f(c)}{d-c}(x-c) \right]
        \end{equation*}
        Analogously to $f$, we have that
        \begin{align*}
            g\left( \frac{x+y}{2} \right) &= f\left( \frac{x+y}{2} \right)-\left[ f(c)+\frac{f(d)-f(c)}{d-c}\left( \frac{x+y}{2}-c \right) \right]\\
            &= f\left( \frac{x+y}{2} \right)-\left[ \frac{f(c)+f(c)}{2}+\frac{f(d)-f(c)}{d-c}\left( \frac{x-c+y-c}{2} \right) \right]\\
            &= f\left( \frac{x+y}{2} \right)-\frac{1}{2}\left[ \left( f(c)+\frac{f(d)-f(c)}{d-c}(x-c) \right)+\left( f(c)+\frac{f(d)-f(c)}{d-c}(y-c) \right) \right]\\
            &\leq \frac{f(x)+f(y)}{2}-\frac{1}{2}\left[ \left( f(c)+\frac{f(d)-f(c)}{d-c}(x-c) \right)+\left( f(c)+\frac{f(d)-f(c)}{d-c}(y-c) \right) \right]\\
            &= \frac{1}{2}\left[ \left( f(x)-\left( f(c)+\frac{f(d)-f(c)}{d-c}(x-c) \right) \right)+\left( f(y)-\left( f(c)+\frac{f(d)-f(c)}{d-c}(y-c) \right) \right) \right]\\
            &= \frac{g(x)+g(y)}{2}
        \end{align*}
        for all $x,y\in(a,b)$, and that $g$ is continuous as the result of sums and products of continuous functions (Theorem 4.9). Additionally, we know that $g(c)=g(d)=0$ (by definition) and that
        \begin{align*}
            g(\lambda c+(1-\lambda)d) &= f(\lambda c+(1-\lambda)d)-\left[ f(c)+\frac{f(d)-f(c)}{d-c}([\lambda c+(1-\lambda)d]-c) \right]\\
            &= f(\lambda c+(1-\lambda)d)-\left[ f(c)+\frac{f(d)-f(c)}{d-c}((1-\lambda)d-(1-\lambda)c) \right]\\
            &= f(\lambda c+(1-\lambda)d)-[f(c)+(1-\lambda)(f(d)-f(c))]\\
            &= f(\lambda c+(1-\lambda)d)-f(c)-(1-\lambda)f(d)+(1-\lambda)f(c)\\
            &> \lambda f(c)+(1-\lambda)f(d)-f(c)-(1-\lambda)f(d)+(1-\lambda)f(c)\\
            &= 0
        \end{align*}
        Since $g$ is continuous and $[c,d]$ is compact, Theorem 4.16 asserts that $g$ attains its maximum, say of $g(e)$ at $e\in[c,d]$. It follows since $g(\lambda c+(1-\lambda)d)>0$ that $g(e)\geq g(\lambda c+(1-\lambda)d)>0$. We now divide into two cases ($d(c,e)\leq d(e,d)$ and $d(c,e)>d(e,d)$). In the first case, let $\delta=d(c,e)$. Since $g(e)\geq g(x)$ for all $x\in[c,d]$, we know that $g(c+2\delta)\leq g(e)$. It follows since $g(e)>0$ that
        \begin{equation*}
            g(e) = g(c+\delta)
            = g\left( \frac{c+(c+2\delta)}{2} \right)
            \leq \frac{g(c)+g(c+2\delta)}{2}
            = \frac{g(c+2\delta)}{2}
            < g(c+2\delta)
            \leq g(e)
        \end{equation*}
        a contradiction. The proof is symmetric in the other case.
    \end{proof}
    \item If $A,B\subset\R^k$, define $A+B$ to be the set of all sums $\x+\y$ with $\x\in A$, $\y\in B$.
    \begin{enumerate}
        \item If $K$ is compact and $C$ is closed in $\R^k$, prove that $K+C$ is closed. (Hint: Take $\z\notin K+C$, put $F=\z-C$, the set of all $\z-\y$ with $\y\in C$. Then $K$ and $F$ are disjoint. Choose $\delta$ as in Exercise 4.21. Show that the open ball with center $\z$ and radius $\delta$ does not intersect $K+C$.)
        \begin{proof}
            Suppose first that $K+C=\R^k$. Then $K+C$ is closed.\par
            Having dealt with the trivial case, we now seek to prove that $K+C\subsetneq\R^k$ is closed. We will do so by proving that $(K+C)^c$ is open. To do so, it will suffice to show that to every $\z\in(K+C)^c$ there corresponds a $N_\delta(\z)\subset(K+C)^c$. Let $\z\in(K+C)^c$ be arbitrary. Define $F=\z-C$. Now suppose for the sake of contradiction that $\ab\in K\cap F$. Then $\ab=\z-\bb$ for some $\bb\in C$. Additionally, since $\z\notin K+C$, $\z\neq\x+\y$ for any $\x\in K$, $\y\in C$. Thus, $\ab\neq(\x+\bb)-\bb=\x$ for any $\x\in K$, so $\ab\notin K$, a contradiction. Therefore, $K,F$ are disjoint. It follows since $K$ compact and $F$ closed are disjoint subsets of the metric space $\R^k$ by Exercise 4.21 that there exists $\delta>0$ such that $\norm{\x-\y}>\delta$ for all $\x\in K$, $\y\in F$. Now suppose there exists $\x+\y\in N_\delta(\z)$ where $\x\in K$ and $\y\in C$. Then $\norm{\x+\y-\z}=\norm{\x-(\z-\y)}<\delta$, a contradiction.
        \end{proof}
        \item Let $\alpha$ be an irrational real number. Let $C_1$ be the set of all integers, and let $C_2$ be the set of all $n\alpha$ with $n\in C_1$. Show that $C_1$ and $C_2$ are closed subsets of $\R^1$ whose sum $C_1+C_2$ is \emph{not} closed, by showing that $C_1+C_2$ is a countable dense subset of $\R^1$.
        \begin{proof}
            $C_1$ and $C_2$ are both closed since every point in each is isolated.\par
            $C_1+C_2$ is countable since we can define a natural injection from it to $C_1\times C_2$ and we know that cross products of countable sets are countable. $C_1+C_2$ is dense in $\R^1$ since the set of all fractional parts of all $n\alpha\in C_2$ is dense in $[0,1]$ because $\alpha$ is irrational (and thus there is no repeating cycle), and we can shift this dense segment using values in $C_1$. Thus, since $C_1+C_2$ is a countable dense subset of $\R^1$, $\overline{C_1+C_2}=\R^1$ is uncountable, i.e., contains more elements than $C_1+C_2$, showing that $C_1+C_2$ is not closed.
        \end{proof}
    \end{enumerate}
    \item Suppose $X,Y,Z$ are metric spaces, and $Y$ is compact. Let $f:X\to Y$, let $g:Y\to Z$ be continuous and 1-1, and let $h(x)=g(f(x))$ for all $x\in X$. Prove that $f$ is uniformly continuous if $h$ is uniformly continuous. (Hint: $g^{-1}$ has compact domain $g(Y)$, and $f(x)=g^{-1}(h(x))$.) Prove also that $f$ is continuous if $h$ is continuous. Show (by modifying Example 4.21, or by finding a different example) that the compactness of $Y$ cannot be omitted from the hypotheses, even when $X$ and $Z$ are compact.
    \begin{proof}
        To prove that $f$ is uniformly continuous, it will suffice to show that for every $\epsilon>0$, there exists a $\delta>0$ such that if $x,y\in X$ and $d_X(x,y)<\delta$, then $d_Y(f(x),f(y))<\epsilon$. Let $\epsilon>0$ be arbitrary. Since $g$ is continuous and 1-1 on $Y$ compact, Theorem 4.17 implies that $g^{-1}:g(Y)\to Y$ is continuous. Additionally, since $g$ is continuous and $Y$ is compact, $g(Y)$ is compact. The last two results imply by Theorem 4.19 that $g^{-1}$ is uniformly continuous. Therefore, since $f=g^{-1}\circ h$ where $g^{-1},h$ are uniformly continuous, Exercise 4.12 implies that $f$ is uniformly continuous.\par
        The proof that $f$ is continuous given that $h$ is continuous is symmetric to the above, except that it uses Theorem 4.7.\par
        Let $X=[0,2\pi]$, $Y=[0,2\pi)$, and $Z=\{(r,\theta)\in\R^2:r=1,\theta\in\R\}$ be metric spaces under the normal Euclidean metric. Clearly $X$ and $Z$ are compact while $Y$ is not. Let $f:X\to Y$ be defined by
        \begin{equation*}
            f(x) =
            \begin{cases}
                \pi-x & x\in[0,\pi]\\
                3\pi-x & x\in(\pi,2\pi]
            \end{cases}
        \end{equation*}
        Let $g:Y\to Z$ be defined by
        \begin{equation*}
            g(x) = (\cos x,\sin x)
        \end{equation*}
        By Example 4.21, $g$ is continuous and 1-1. Clearly $h=g\circ f$ is uniformly continuous while $f$ is not.
    \end{proof}
\end{enumerate}




\end{document}