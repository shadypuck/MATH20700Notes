\documentclass[../psets.tex]{subfiles}

\pagestyle{main}
\renewcommand{\leftmark}{Problem Set \thesection}

\begin{document}




\section{Matrix Basics and Linear Systems}
\emph{From \textcite{bib:Treil}.}
\subsection*{Chapter 1}
\begin{enumerate}[label={\textbf{1.\arabic*.}}]
    \stepcounter{enumi}
    \item \marginnote{10/4:}Which of the following sets (with natural addition and multiplication by a scalar) are vector spaces? Justify your answer.
    \begin{enumerate}
        \item The set of all continuous functions on the interval $[0,1]$.
        \item The set of all non-negative functions on the interval $[0,1]$.
        \item The set of all polynomials of degree \emph{exactly} $n$.
        \item The set of all symmetric $n\times n$ matrices, i.e., the set of matrices $A=\{a_{j,k}\}_{j,k=1}^n$ such that $A^T=A$.
    \end{enumerate}
    \item True or false:
    \begin{enumerate}
        \item Every vector space contains a zero vector.
        \item A vector space can have more than one zero vector.
        \item An $m\times n$ matrix has $m$ rows and $n$ columns.
        \item If $f$ and $g$ are polynomials of degree $n$, then $f+g$ is also a polynomial of degree $n$.
        \item If $f$ and $g$ are polynomials of degree at most $n$, then $f+g$ is also a polynomial of degree at most $n$.
    \end{enumerate}
\end{enumerate}

\begin{enumerate}[label={\textbf{2.\arabic*.}}]
    \stepcounter{enumi}
    \item True or false:
    \begin{enumerate}
        \item Any set containing a zero vector is linearly dependent.
        \item A basis must contain $\bm{0}$.
        \item Subsets of linearly dependent sets are linearly dependent.
        \item Subsets of linearly independent sets are linearly independent.
        \item If $\alpha_1\vm_1+\alpha_2\vm_2+\cdots+\alpha_n\vm_n=\mathbf{0}$, then all scalars $\alpha_k$ are zero.
    \end{enumerate}
    \setcounter{enumi}{4}
    \item Let a system of vectors $\vm_1,\vm_2,\dots,\vm_r$ be linearly independent but not generating. Show that it is possible to find a vector $\vm_{r+1}$ such that the system $\vm_1,\vm_2,\dots,\vm_r,\vm_{r+1}$ is linearly independent. (Hint: Take for $\vm_{r+1}$ any vector that cannot be represented as a linear combination $\sum_{k=1}^r\alpha_k\vm_k$ and show that the system $\vm_1,\vm_2,\dots,\vm_r,\vm_{r+1}$ is linearly independent.)
    \item Is it possible that vectors $\vm_1,\vm_2,\vm_3$ are linearly dependent, but the vectors $\wm_1=\vm_1+\vm_2,\wm_2=\vm_2+\vm_3,\wm_3=\vm_3+\vm_1$ are linearly \emph{independent}?
\end{enumerate}

\begin{enumerate}[label={\textbf{3.\arabic*.}}]
    \setcounter{enumi}{2}
    \item For each linear transformation below, find its matrix.
    \begin{enumerate}
        \item $T:\R^2\to\R^3$ defined by $T(x,y)^T=(x+2y,2x-5y,7y)^T$.
        \item $T:\R^4\to\R^3$ defined by $T(x_1,x_2,x_3,x_4)^T=(x_1+x_2+x_3+x_4,x_2-x_4,x_1+3x_2+6x_4)^T$.
        \item $T:\Pm_n\to\Pm_n$ defined by $Tf(t)=f'(t)$ (find the matrix with respect to the standard basis $1,t,t^2,\dots,t^n$).
        \item $T:\Pm_n\to\Pm_n$ defined by $Tf(t)=2f(t)+3f'(t)-4f''(t)$ (again with respect to the standard basis $1,t,t^2,\dots,t^n$).
    \end{enumerate}
    \setcounter{enumi}{5}
    \item The set $\C$ of complex numbers can be canonically identified with the space $\R^2$ by treating each $z=x+iy\in\C$ as a column $(x,y)^T\in\R^2$.
    \begin{enumerate}
        \item Treating $\C$ as a complex vector space, show that the multiplication by $\alpha=a+ib\in\C$ is a linear transformation in $\C$. What is its matrix?
        \item Treating $\C$ as the real vector space $\R^2$, show that the multiplication by $\alpha=a+ib$ defines a linear transformation there. What is its matrix?
        \item Define $T(x+iy)=2x-y+i(x-3y)$. Show that this transformation is not a linear transformation in the complex vector space $\C$, but if we treat $\C$ as the real vector space $\R^2$, then it is a linear transformation there (i.e., that $T$ is a \emph{real linear} but not a \emph{complex linear} transformation). Find the matrix of the real linear transformation $T$.
    \end{enumerate}
\end{enumerate}

\begin{enumerate}[label={\textbf{5.\arabic*.}}]
    \setcounter{enumi}{2}
    \item Multiply two rotation matrices $T_\alpha$ and $T_\beta$ (it is a rare case when the multiplication is commutative, i.e., $T_\alpha T_\beta=T_\beta T_\alpha$, so the order is not essential). Deduce formulas for $\sin(\alpha+\beta)$ and $\cos(\alpha+\beta)$ from here.
    \setcounter{enumi}{4}
    \item Find linear transformations $A,B:\R^2\to\R^2$ such that $AB=\bm{0}$ but $BA\neq\bm{0}$.
    \setcounter{enumi}{7}
    \item Find the matrix of the reflection through the line $y=-2x/3$. Perform all the multiplications.
\end{enumerate}

\begin{enumerate}[label={\textbf{6.\arabic*.}}]
    \setcounter{enumi}{2}
    \item Find all left inverses of the column $(1,2,3)^T$.
    \setcounter{enumi}{5}
    \item Suppose the product $AB$ is invertible. Show that $A$ is right invertible and $B$ is left invertible. (Hint: You can just write formulas for right and left inverses.)
    \stepcounter{enumi}
    \item Let $A$ be an $n\times n$ matrix. Prove that if $A^2=\bm{0}$, then $A$ is not invertible.
    \stepcounter{enumi}
    \item Write matrices of the linear transformations $T_1$ and $T_2$ in $\F^5$, defined as follows: $T_1$ interchanges the coordinates $x_2$ and $x_4$ of the vector $\x$, and $T_2$ just adds to the coordinate $x_2$ the quantity $a$ times the coordinate $x_4$, and does not change other coordinates, i.e.,
    \begin{align*}
        T_1
        \begin{pmatrix}
            x_1\\
            x_2\\
            x_3\\
            x_4\\
            x_5\\
        \end{pmatrix}
        &=
        \begin{pmatrix}
            x_1\\
            x_4\\
            x_3\\
            x_2\\
            x_5\\
        \end{pmatrix}&
        T_2
        \begin{pmatrix}
            x_1\\
            x_2\\
            x_3\\
            x_4\\
            x_5\\
        \end{pmatrix}
        &=
        \begin{pmatrix}
            x_1\\
            x_2+ax_4\\
            x_3\\
            x_4\\
            x_5\\
        \end{pmatrix}
    \end{align*}
    where $a$ is some fixed number. Show that $T_1$ and $T_2$ are invertible transformations, and write the matrices of the inverses. (Hint: It may be simpler, if you first describe the inverse transformation, and then find its matrix, rather than trying to guess [or compute] the inverses of the matrices $T_1,T_2$.)
    \setcounter{enumi}{12}
    \item Let $A$ be an invertible symmetric ($A^T=A$) matrix. Is the inverse of $A$ symmetric? Justify.
\end{enumerate}

\begin{enumerate}[label={\textbf{7.\arabic*.}}]
    \setcounter{enumi}{2}
    \item Let $X$ be a subspace of a vector space $V$, and let $\vm\in V$, $\vm\notin X$. Prove that if $\x\in X$, then $\x+\vm\notin X$.
    \item Let $X$ and $Y$ be subspaces of a vector space $V$. Using the previous exercise, show that $X\cup Y$ is a subspace if and only if $X\subset Y$ or $Y\subset X$.
    \item What is the smallest subspace of the space of $4\times 4$ matrices which contains all upper triangular matrices ($a_{j,k}=0$ for all $j>k$), and all symmetric matrices $(A=A^T)$? What is the largest subspace contained in both of those subspaces?
\end{enumerate}


\subsection*{Chapter 2}
\begin{enumerate}[label={\textbf{3.\arabic*.}}]
    \setcounter{enumi}{3}
    \item Do the polynomials $x^3+2x$, $x^2+x+1$, $x^3+5$ generate (span) $\Pm_3$? Justify your answer.
    \item Can 5 vectors in $\F^4$ be linearly independent? Justify your answer.
    \stepcounter{enumi}
    \item Prove or disprove: If the columns of a square ($n\times n$) matrix $A$ are linearly independent, so are the rows of $A^3=AAA$.
\end{enumerate}

\begin{enumerate}[label={\textbf{5.\arabic*.}}]
    \item True or false:
    \begin{enumerate}
        \item Every vector space that is generated by a finite set has a basis.
        \item Every vector space has a (finite) basis.
        \item A vector space cannot have more than one basis.
        \item If a vector space has a finite basis, then the number of vectors in every basis is the same.
        \item The dimension of $\Pm_n$ is $n$.
        \item The dimension on $M_{m\times n}$ is $m+n$.
        \item If vectors $\vm_1,\vm_2,\dots,\vm_n$ generate (span) the vector space $V$, then every vector in $V$ can be written as a linear combination of vectors $\vm_1,\vm_2,\dots,\vm_n$ in only one way.
        \item Every subspace of a finite-dimensional space is finite-dimensional.
        \item If $V$ is a vector space having dimension $n$, then $V$ has exactly one subspace of dimension 0 and exactly one subspace of dimension $n$.
    \end{enumerate}
    \item Prove that if $V$ is a vector space having dimension $n$, then a system of vectors $\vm_1,\vm_2,\dots,\vm_n$ in $V$ is linearly independent if and only if it spans $V$.
    \setcounter{enumi}{5}
    \item Consider in the space $\R^5$ vectors $\vm_1=(2,-1,1,5,-3)^T$, $\vm_2=(3,-2,0,0,0)^T$, $\vm_3=(1,1,50,-921,0)^T$. (Hint: If you do part (b) first, you can do everything without any computations.)
    \begin{enumerate}
        \item Prove that these vectors are linearly independent.
        \item Complete the system of vectors to a basis.
    \end{enumerate}
\end{enumerate}

\begin{enumerate}[label={\textbf{6.\arabic*.}}]
    \item True or false:
    \begin{enumerate}
        \item Any system of linear equations has at least one solution.
        \item Any system of linear equations has at most one solution.
        \item Any homogeneous system of linear equations has at least one solution.
        \item Any system of $n$ linear equations in $n$ unknowns has at least one solution.
        \item Any system of $n$ linear equations in $n$ unknowns has at most one solution.
        \item If the homogeneous system corresponding to a given system of linear equations has a solution, then the given system has a solution.
        \item If the coefficient matrix of a homogeneous system of $n$ linear equations in $n$ unknowns is invertible, then the system has no non-zero solutions.
        \item The solution set of any system of $m$ equations in $n$ unknowns is a subspace of $\R^n$.
        \item The solution set of any homogeneous system of $m$ equations in $n$ unknowns is a subspace of $\R^n$.
    \end{enumerate}
\end{enumerate}

\begin{enumerate}[label={\textbf{7.\arabic*.}}]
    \item True or false:
    \begin{enumerate}
        \item The rank of a matrix is equal to the number of its non-zero columns.
        \item The $m\times n$ zero matrix is the only $m\times n$ matrix having rank 0.
        \item Elementary row operations preserve rank.
        \item Elementary column operations do not necessarily preserve rank.
        \item The rank of a matrix is equal to the maximum number of linearly independent columns in the matrix.
        \item The rank of a matrix is equal to the maximum number of linearly independent rows in the matrix.
        \item The rank of an $n\times n$ matrix is at most $n$.
        \item An $n\times n$ matrix having rank $n$ is invertible.
    \end{enumerate}
    \setcounter{enumi}{3}
    \item Prove that if $A:X\to Y$ and $V$ is a subspace of $X$, then $\dim AV\leq\rank A$. ($AV$ here means the subspace $V$ transformed by the transformation $A$, i.e., any vector in $AV$ can be represented as $A\vm$, $\vm\in V$.) Deduce from here that $\rank AB\leq\rank A$. (Remark: Here, one can use the fact that if $V\subset W$, then $\dim V\leq\dim W$. Do you understand why it is true?)
    \stepcounter{enumi}
    \item Prove that if the product $AB$ of two $n\times n$ matrices is invertible, then both $A$ and $B$ are invertible. Even if you know about determinants, do not use them (we did not cover them yet). (Hint: Use the previous 2 problems.)
    \setcounter{enumi}{8}
    \item If $A$ has the same four fundamental subspaces as $B$, does $A=B$?
    \setcounter{enumi}{13}
    \item Is it possible for a real matrix $A$ that $\Ran A=\Ker A^T$? Is it possible for a complex $A$?
\end{enumerate}

\begin{enumerate}[label={\textbf{8.\arabic*.}}]
    \setcounter{enumi}{2}
    \item Find the change of coordinates matrix that changes the coordinates in the basis $1,1+t$ in $\Pm_1$ to the coordinates in the basis $1-t,2t$.
    \setcounter{enumi}{5}
    \item Are the matrices $
        \left(
        \begin{smallmatrix}
            1 & 3\\
            2 & 2\\
        \end{smallmatrix}
        \right)
    $ and $
        \left(
        \begin{smallmatrix}
            0 & 2\\
            4 & 2\\
        \end{smallmatrix}
        \right)
    $ similar? Justify.
\end{enumerate}




\end{document}