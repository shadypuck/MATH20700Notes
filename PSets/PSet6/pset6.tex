\documentclass[../psets.tex]{subfiles}

\pagestyle{main}
\renewcommand{\leftmark}{Problem Set \thesection}
\setcounter{section}{5}

\begin{document}




\section{Basic Topology}
\emph{From \textcite{bib:Rudin}.}
\subsection*{Chapter 2}
\begin{enumerate}[label={\textbf{\arabic*.}}]
    \item \marginnote{11/8:}Prove that the empty set is a subset of every set.
    \item A complex number $z$ is said to be \textbf{algebraic} if there are integers $a_0,\dots,a_n$, not all zero, such that
    \begin{equation*}
        a_0z^n+a_1z^{n-1}+\cdots+a_{n-1}z+a_n = 0
    \end{equation*}
    Prove that the set of all algebraic numbers is countable. (Hint: For every positive integer $N$, there are only finitely many equations with $n+|a_0|+|a_1|+\cdots+|a_n|=N$.)
    \item Prove that there exist real number which are not algebraic.
    \item Is the set of all irrational real numbers countable?
    \item Construct a bounded set of real numbers with exactly three limit points.
    \item Let $E'$ be the set of all limit points of a set $E$. Prove that $E'$ is closed. Prove that $E$ and $\bar{E}$ have the same limit points (recall that $\bar{E}=E\cup E'$). Do $E$ and $E'$ always have the same limit points?
    \item Let $A_1,A_2,\dots$ be subsets of a metric space.
    \begin{enumerate}
        \item If $B_n=\bigcup_{i=1}^nA_i$, prove that $\bar{B}_n=\bigcup_{i=1}^n\bar{A}_i$ for $n=1,2,3,\dots$.
        \item If $B=\bigcup_{i=1}^\infty A_i$, prove that $\bar{B}\supset\bigcup_{i=1}^\infty\bar{A}_i$. Show, by an example, that this inclusion can be proper.
    \end{enumerate}
    \item Is every point of every open set $E\subset\R^2$ a limit point of $E$? Answer the same question for closed sets in $\R^2$.
    \item Let $E^\circ$ denote the set of all interior points of a set $E$ (see Definition 2.18e; $E^\circ$ is called the \textbf{interior} of $E$).
    \begin{enumerate}
        \item Prove that $E^\circ$ is always open.
        \item Prove that $E$ is open if and only if $E^\circ=E$.
        \item If $G\subset E$ and $G$ is open, prove that $G\subset E^\circ$.
        \item Prove that the complement of $E^\circ$ is the closure of the complement of $E$.
        \item Do $E$ and $\bar{E}$ always have the same interiors?
        \item Do $E$ and $E^\circ$ always have the same closures?
    \end{enumerate}
    \item Let $X$ be an infinite set. For $p\in X$ and $q\in X$, define
    \begin{equation*}
        d(p,q) =
        \begin{cases}
            1 & p\neq q\\
            0 & p=q
        \end{cases}
    \end{equation*}
    Prove that this is a metric. Which subsets of the resulting metric space are open? Which are closed? Which are compact?
    \item For $x\in\R^1$ and $y\in\R^1$, define
    \begin{align*}
        d_1(x,y) &= (x-y)^2\\
        d_2(x,y) &= \sqrt{|x-y|}\\
        d_3(x,y) &= |x^2-y^2|\\
        d_4(x,y) &= |x-2y|\\
        d_5(x,y) &= \frac{|x-y|}{1+|x-y|}
    \end{align*}
    Determine, for each of these, whether it is a metric or not.
    \item Let $K\subset\R^1$ consist of 0 and the numbers $1/n$ for $n=1,2,3,\dots$. Prove that $K$ is compact directly from the definition (without using the Heine-Borel theorem).
    \item Construct a compact set of real numbers whose limit points form a countable set.
    \item Give an exmaple of an open cover of the segment $(0,1)$ which has no finite subcover.
    \item Show that Theorem 2.36 and its Corollary become false (in $\R^1$, for example) if the word "compact" is replaced by "closed" or by "bounded."
\end{enumerate}




\end{document}